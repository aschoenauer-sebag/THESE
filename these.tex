%%%%%%%%%%%%%%%%%%%%%%%%%%%%%%%%%%%%%%%%%
% Masters/Doctoral Thesis 
% LaTeX Template
% Version 1.43 (17/5/14)
%
% This template has been downloaded from:
% http://www.LaTeXTemplates.com
%
% Original authors:
% Steven Gunn 
% http://users.ecs.soton.ac.uk/srg/softwaretools/document/templates/
% and
% Sunil Patel
% http://www.sunilpatel.co.uk/thesis-template/
%
% License:
% CC BY-NC-SA 3.0 (http://creativecommons.org/licenses/by-nc-sa/3.0/)
%
% Note:
% Make sure to edit document variables in the Thesis.cls file
%
%%%%%%%%%%%%%%%%%%%%%%%%%%%%%%%%%%%%%%%%%


%----------------------------------------------------------------------------------------
%	PACKAGES AND OTHER DOCUMENT CONFIGURATIONS
%----------------------------------------------------------------------------------------

\documentclass[11pt, oneside]{Thesis} % The default font size and one-sided printing (no margin offsets)
% oneside : better for pdf reading
% twoside : better if printed in book format
\usepackage{cite} % for multiple quotes
%\usepackage[]{algorithm2e} % for algorithm environment
%\usepackage[square, comma, numbers, sort&compress]{natbib} % Use the natbib reference package - read up on this to edit the reference style; if you want text (e.g. Smith et al., 2012) for the in-text references (instead of numbers), remove 'numbers' 

\usepackage[french]{babel}
\usepackage[T1]{fontenc}
\usepackage[utf8]{inputenc}
\usepackage{amsmath}

\usepackage{graphicx}
%\usepackage{graphics}
\usepackage{color}
\usepackage{floatflt}
%usepackage[frenchb]{babel}
\usepackage{amsfonts}
\usepackage{multirow}
\usepackage{calc}
%\linespread{1.3}
\usepackage{setspace}
\usepackage{hyperref}
\usepackage{longtable}
\hypersetup{bookmarks=true,
colorlinks,%
citecolor=black,%
filecolor=black,%
linkcolor=black,%
urlcolor=blue,%
pdftex}
\usepackage{array}
\newcolumntype{L}[1]{>{\raggedright\let\newline\\\arraybackslash\hspace{0pt}}m{#1}}
\newcolumntype{C}[1]{>{\center\let\newline\\\arraybackslash\hspace{0pt}}m{#1}}
\DeclareMathOperator*{\argmax}{arg\,max}
\DeclareMathOperator*{\argmin}{arg\,min}
\title{\ttitle} % Defines the thesis title - don't touch this

%\setlength{\unitlength}{2mm}
%\pagestyle{plain}
%\textwidth 17truecm
%\textheight 23truecm
%\voffset -2.5truecm
%\hoffset -2.0truecm
%\sloppy
%\hbadness = 10000
%\vbadness = 10000
%\def\RR{{\rm I\hspace{-0.50ex}R} }

\begin{document}
\frontmatter % Use roman page numbering style (i, ii, iii, iv...) for the pre-content pages

\setstretch{1.3} % Line spacing of 1.3

% Define the page headers using the FancyHdr package and set up for one-sided printing
\fancyhead{} % Clears all page headers and footers
\rhead{\thepage} % Sets the right side header to show the page number
\lhead{} % Clears the left side page header

\pagestyle{fancy} % Finally, use the "fancy" page style to implement the FancyHdr headers

\newcommand{\HRule}{\rule{\linewidth}{0.5mm}} % New command to make the lines in the title page

% PDF meta-data
\hypersetup{pdftitle={\ttitle}}
\hypersetup{pdfsubject=\subjectname}
\hypersetup{pdfauthor=\authornames}
\hypersetup{pdfkeywords=\keywordnames}

%%%%%%%%%%%%%%%%%%%%%%%%%%%%%%%%%%%
% Personal Commands
%%%%%%%%%%%%%%%%%%%%%%%%%%%%%%%%%%%


%----------------------------------------------------------------------------------------
%	TITLE PAGE
%----------------------------------------------------------------------------------------

%\begin{titlepage}
%\begin{center}
%Titre
%\end{center}
%
%\end{titlepage}

%----------------------------------------------------------------------------------------
%	DECLARATION PAGE
%	Your institution may give you a different text to place here
%----------------------------------------------------------------------------------------

\Declaration{

\addtocontents{toc}{\vspace{1em}} % Add a gap in the Contents, for aesthetics

I, \authornames, declare that this thesis titled, '\ttitle' and the work presented in it are my own. I confirm that:

\begin{itemize} 
\item[\tiny{$\blacksquare$}] This work was done wholly or mainly while in candidature for a research degree at this University.
\item[\tiny{$\blacksquare$}] Where any part of this thesis has previously been submitted for a degree or any other qualification at this University or any other institution, this has been clearly stated.
\item[\tiny{$\blacksquare$}] Where I have consulted the published work of others, this is always clearly attributed.
\item[\tiny{$\blacksquare$}] Where I have quoted from the work of others, the source is always given. With the exception of such quotations, this thesis is entirely my own work.
\item[\tiny{$\blacksquare$}] I have acknowledged all main sources of help.
\item[\tiny{$\blacksquare$}] Where the thesis is based on work done by myself jointly with others, I have made clear exactly what was done by others and what I have contributed myself.\\
\end{itemize}
 
Signed:\\
\rule[1em]{25em}{0.5pt} % This prints a line for the signature
 
Date:\\
\rule[1em]{25em}{0.5pt} % This prints a line to write the date
}

\clearpage % Start a new page

%----------------------------------------------------------------------------------------
%	QUOTATION PAGE
%----------------------------------------------------------------------------------------

%----------------------------------------------------------------------------------------
%	RESUME PAGE
%----------------------------------------------------------------------------------------

\addtotoc{Résumé} % Add the "Abstract" page entry to the Contents


\clearpage % Start a new page

%----------------------------------------------------------------------------------------
%	ABSTRACT PAGE
%----------------------------------------------------------------------------------------

\addtotoc{Abstract} % Add the "Abstract" page entry to the Contents
\begin{abstract}
Biological screens are experiments in which a set of compounds is tested for a specific biological effect on a living entity. Recent progresses in hardware automation, software, biomolecular engineering and organic chemistry have simultaneously made it possible to biologically test hundreds to thousands of chemical compounds in parallel. Molecules such as small interfering RNAs or custom plasmids enable to systematically assess the consequences of gene silencing or overexpression, whereas putative drugs and small molecules are efficiently toxicologically and functionally tested.

Time-lapse microscopy screening experiments are screening experiments in which images are acquired over time. This specific type of screening experiments enables to gather even more and more precise data regarding the consequences of chemical perturbation on a given biological process. Nevertheless, on top of specific experimental constraints, the quantity and specific structure of information which such experiments produce make them difficult to analyze. They demand the combination of robust computer vision methods and efficient statistical algorithms for significant effect detection, on top of reliable quality controls.
\end{abstract}
Through examples, this thesis answers the question to know how to optimally develop methods for analyzing and re-analyzing high throughput (HT) time-lapse microscopy screening data. A generic methodological workflow for the study of single cell motility in such data is detailed in Chapter 2. It constitutes the first multivariate HT workflow to study single cell motility. Chapter 3 presents the application of this workflow to published data, and the development of a new distance for drug target inference by \textit{in silico} comparison of existing siRNA experiments to an unpublished drug screen. Therefore, it demonstrates the significant insights which one can gain by developing generic methods for HT time-lapse screens, and applying them to published data. Finally, chapter 4 presents a complete methodological pipeline for performing HT time-lapse screens in Environmental Toxicology, together with its visualization Web-interface.


\clearpage % Start a new page
%----------------------------------------------------------------------------------------
%	ACKNOWLEDGEMENTS
%----------------------------------------------------------------------------------------

% \setstretch{1.3} % Reset the line-spacing to 1.3 for body text (if it has changed)
 
 \acknowledgements{\addtocontents{toc}{\vspace{1em}} % Add a gap in the Contents, for aesthetics
 hidden for now
% Thomas JP Nelle equipe + assistance / Robert Celine RT Olivier Celine D Martine +equipe Luc
% 
% Papa Maman Jean Mathilde
% 
% Nico Ge Loulou Fred / coupains, Laure 
% 
%Vincent Max Anna Charles
%
% Olivia Greg Noemie 
% 
% Sandra
% 
% Andrei
 }
 \clearpage % Start a new page

%----------------------------------------------------------------------------------------
%	LIST OF CONTENTS/FIGURES/TABLES PAGES
%----------------------------------------------------------------------------------------

\pagestyle{fancy} % The page style headers have been "empty" all this time, now use the "fancy" headers as defined before to bring them back

\lhead{\emph{Contents}} % Set the left side page header to "Contents"
\tableofcontents % Write out the Table of Contents

\lhead{\emph{List of Figures}} % Set the left side page header to "List of Figures"
\listoffigures % Write out the List of Figures

\lhead{\emph{List of Tables}} % Set the left side page header to "List of Tables"
\listoftables % Write out the List of Tables

%----------------------------------------------------------------------------------------
%	ABBREVIATIONS
%----------------------------------------------------------------------------------------

\clearpage % Start a new page
\lhead{\emph{List of abbreviations and main definitions}} % Set the left side page header to "List of Figures"
\begin{table}
\caption{Abbreviations}
\begin{tabular}{ll}
ADCCM &Asymmetric Distribution of Condensed Chromosome Masses\\
AURKA,B & Aurora kinase A, B\\
BPA & bisphenol-A \\
DMSO & dimethyl sulfoxide\\
DNA & deoxyribonucleic acid\\
ds & double stranded \\
Endo & $\alpha$-endosulfan \\
FACS & fluorescence-activated cell sorter\\
FCS & fetal calf serum\\
H2B & histone 2B\\
HC & high-content\\
HCS & high-content screening\\
HT & high-throughput\\
MeHg & methylmercury\\
MSD & mean squared displacement\\
PCB153 & 2,2',4,4',5,5'-Hexachlorobiphenyl\\
PCNA & proliferating cell nuclear antigen\\
RNA & ribonucleic acid\\
RT-qPCR & real-time quantitative polymerase chain reaction\\
TCDD & 2,3,7,8-Tetrachlorodibenzo-p-dioxin\\
TOP1 & topoisomerase (DNA) I\\
\end{tabular}
\end{table}
\begin{table}
\caption{Main definitions}
\begin{tabular}{lL{10cm}}
Condition & Combination of a dose and a chemical\\
Xenobiotic & Any chemical which is foreign to an organism, i.e. is neither produced nor expected to be found in it.
\end{tabular}
\end{table}
\clearpage % Start a new page

%----------------------------------------------------------------------------------------
%	DEDICATION
%----------------------------------------------------------------------------------------

\setstretch{1.3} % Return the line spacing back to 1.3

\pagestyle{empty} % Page style needs to be empty for this page

\dedicatory{A Mine et Baba, Tita et Pigeo} % Dedication text

\addtocontents{toc}{\vspace{2em}} % Add a gap in the Contents, for aesthetics

%----------------------------------------------------------------------------------------
%	THESIS CONTENT - CHAPTERS
%----------------------------------------------------------------------------------------

\mainmatter % Begin numeric (1,2,3...) page numbering

\pagestyle{fancy} % Return the page headers back to the "fancy" style

\lhead{\emph{Introduction}} 
\chapter{Introduction}
\paragraph{Biological screens}~\\
Progress in organic chemistry and molecular biology have led to the constitution of giant libraries, respectively of putative drugs and potential biologically active small molecules, and engineered organisms or proteins for gene silencing or overexpression. As an example, the Biomolecular Screening Facility of the EPFL (Lausanne, Switzerland) has a collection of 65,000 compounds and 130,000 small interfering RNAs (siRNAs\footnote{Small interfering RNAs are short double-stranded RNA molecules which interfers with the expression of genes that present complementary nucleotide sequences.}), while most bio-technological companies offer to ship custom genome-wide siRNA libraries over a fortnight. Pharmaceutical industry libraries are impressive as well, most containing more than one million compounds.

Putative drugs need to be tested for the expected biological effect and against undesirable secondary effects. Understanding their mode-of-action, which screening experiments can help to do, is also a major concern during drug discovery, and often its rate-limiting step~\cite{pmid15547975}. Screens have therefore become a major component of drug discovery processes~\cite{pmid22155864}. On the other hand, the development of biomolecular engineering, and small molecule and siRNA libraries at (more) affordable costs, has led to a significant increase in functional genomic screens, which test for gene function and gene relations. For example, \cite{pmid12140549} exhaustively engineered gene-deletion mutants of the yeast \textit{Saccharomyces cerevisiae}, while \cite{pmid16511445} performed a genome-wide siRNA screen in the fruit fly \textit{Drosophila melanogaster}.

Finally, progress in organic chemistry has also led to the explosion of the number of new molecules which are synthesized each year for industrial purposes (e.g. pesticides, plastics, food additives). This calls for the development of systematic testing experiments, both for the desired action and against undesirable effects on living organisms. Hence a third major field of application for biological screens is Environmental Toxicology.

\textbf{Biological screens} are experiments which are designed for testing a set of compounds for a specific biological action in a given organism. The latter can be any organism in which the action is easily detected, such as a fish (\textit{Danio rerio}), a fly (\textit{Drosophila melanogaster}), a worm (\textit{Caenorhabditis elegans}) or a human cell (\textit{Homo sapiens}). The biological action which is tested can go from a simple univariate assessment of cell death, to the multivariate quantification of an effect on a complex cellular phenotype such as cell division or motility. This will determine the screen \textbf{content}, which would respectively be \textbf{low} and \textbf{high}. Another important parameter of biological screens are their \textbf{throughput}. It ranges from \textbf{low}, for example when testing a dozen of carefully selected compounds, to \textbf{high} for hundreds of thousands of siRNAs in the case of a genome-wide screen. 

\paragraph{Time-lapse microscopy}~\\
Time-lapse microscopy experiments are experiments in which images are regularly acquired over time. Hence they produce rich 3 or 4-dimensional datasets: they are high-content (HC) almost by definition. Most time-lapse data comes from fluorescent samples. Fluorescent labelling has the advantage of being presently very affordable, (mostly) non-toxic to living organisms and flexible. It makes it possible to easily follow a single protein of interest, which would have been delicate in bright-field images. Furthermore, although it demands either the addition of some reagent when acquiring for a few minutes or hours, or cell genetic modification when acquiring repeatedly for a few days, fluorescence images are easier to segment and analyze than bright field images. This generalized use of fluorescent proteins was permitted by their recent discovery, which started by the green fluorescent protein (GFP)~\cite{pmid8303295},~\cite{pmid19575655}.

The use of time-lapse microscopy rather complicates data acquisition: samples need to be maintained in the appropriate atmosphere at the appropriate temperature, and one should be certain to avoid phototoxicity from repeated light exposure. Nevertheless, they provide a wealth of information which it is not possible to access otherwise. First of all, time-lapse microscopy experiments seem natural for studying dynamic processes such as cell division or cell motility. Furthermore, they enable to visualize very transient events, which are barely observed in endpoints assays, such as early anaphases~\cite{pmid20360735}. Time-lapse microscopy experiments also permit to establish causality links between phenotypes~\cite{pmid15539606}. Indeed, it makes it possible to observe the order in which phenotypes occur, therefore enabling to determine which one leads to the other. It also makes it possible to study cell population heterogeneity in response to gene silencing or chemical exposure: tracking cells over time permits to perform \textit{in silico} cell alignment, and therefore to determine if there exists cell subpopulations with different phenotypic stories. This is not surprising given the stochastic nature of gene expression~\cite{pmid18957198}, and can already been found in the Event Order Maps of~\cite{pmid20360735} (although it was not formalized in that direction in the latter). 

Despite more complex experimental procedures, time-lapse microscopy therefore has a real advantage over endpoint assays when studying complex phenotypes: they truly permit to functionally dig into the consequences of either gene silencing or chemical exposure. As such, it started being used approximately 15 years ago, although some pioneering studies date back from the 1980s (e.g.~\cite{pmid6684600}, see~\cite{pmid19575655} for a review on quantitative time-lapse fluorescence microscopy in single cells). We hereafter refer to time-lapse fluorescence microscopy data by time-lapse data.

\paragraph{Analysis of high-throughput time-lapse screening data}
~\\ High-throughput (HT) screening experiments have only been made recently possible by the development of screening robots, automated microscopes and measurement devices, and relevant software. These tools are necessary for preparing and performing the experiments. In the case of low-content experiments, result analysis remains simple, although it should not be forgotten that statistics on large datasets should not be done as on small ones (see paragraph on the control of false discovery rate in section~\ref{sec:fdr}). Quality control procedures shall be included in the pipeline as well.

On top of quality control and large dataset statistics, robust and efficient multivariate analytical methods are necessary to deal with HC HT screening experiments. This applies to HT time-lapse screening data as well. It is indeed a specific type of HC HT screening data, in which one data dimension is time. Most of the time, analyzing such datasets demands to combine computer vision methods to multivariate statistical algorithms for significant effect detection. These methods should be tailored to the biological process which is studied, but they all have in common the challenges to be robust when faced with noise, and scalable as dataset sizes increase. The question is to know how to develop analytical methods for optimally exploiting such datasets.

A first biological process which seems natural to be studied with time-lapse microscopy is single cell motility. A systematic functional genomics approach to cell motility is all the more needed since all the involved proteins and pathways are not yet known. Nevertheless, it was never studied using multivariate statistical tests in HT settings. During this thesis, we therefore developed a generic methodological workflow for studying single cell motility in HT time-lapse screening data, which will be presented in chapter 2. As will be detailed, an \textit{ad-hoc} statistical procedure indeed had to be developed.

The generic quality of this workflow enabled it to be applied to an existing dataset, the Mitocheck dataset (see section~\ref{sec:mito_prez} for a presentation of this dataset). This revealed the quantity of unexploited information in this dataset, and more generally the wealth of existing HT time-lapse screening data which can be re-used to different purposes than the original experimental design. Proofs of this constitutes chapter 3, which presents how an ontology of single cell trajectories could be extracted from the Mitocheck dataset (section~\ref{sec:ontology}). The latter was also used for detecting cell cycle genes (section~\ref{sec:cellcycle}). In section~\ref{target_inference}, it also permitted to perform drug target inference on an unpublished time-lapse drug screen, which made it necessary to develop a new distance.

Finally, as was mentioned in the beginning of this introduction, one of the most important applications of screening is in Toxicology. Moreover, HT time-lapse screening experiments have never been performed in Environmental Toxicology. We therefore developed a robust methodological workflow and its visualization Web-interface, for conducting and analyzing HT time-lapse screening data in Environmental Toxicology. This composes chapter 4.

Before diving into the main matters, all the datasets which were analyzed in the course of this thesis, as well as the software which we used to this end, are briefly described in the following section.
%What is the topic and why is it important? State the problem(s) as simply as you can. How does it fit into the broader world of your discipline? 
%Where did the problem come from? What is already known about this problem? What other methods have been tried to solve it? 
%mentionner la toxicologie environnementale. Ce qu'on fait c'est de la mise en place d'outils. Etre un peu global sur le time-lapse en criblage. Il y a toxicology(envtal pharmaco)/genetic. Explain what a screen is, give examples
%
%What are the challenges and questions that come up in such a setting => pour que ca nous arrange
%
%Not necessarily very long. Just prepare the reader to know what screens are, their scope. Also give a reasoning fil rouge for the rest, for it to be clear for the reader
\section{Data sets}
Four time-lapse datasets were used in our work, which will briefly be presented in this section: the Mitocheck dataset, which is the first genome-wide siRNA time-lapse screen, a PCNA dataset for the study of cell cycle phases, an unpublished drug screen in similar settings to those of the Mitocheck dataset, and an unpublished xenobiotic screen.

\subsection{Mitocheck data set}
\label{sec:mito_prez}
The main dataset which we used is a previously published genome-wide data set of time-resolved
records of cellular phenotype responses to gene silencing, which
were generated for virtually all protein-coding genes~\cite{pmid20360735}. It is publicly available at
\href{http://www.mitocheck.org}{mitocheck.org}. 

For this, arrays of transfection cocktails containing small interfering RNA (siRNA) were spotted directly into live cell-imaging chambers in a 384 format. HeLa cells (ATCC\up{\textregistered} CCL-2\texttrademark) stably expressing the core histone 2B tagged with GFP were seeded on top of the arrays, and imaged 18 h after the transfection for 48 h with a time-lapse of 30 min (Plan10x, NA 0.4; Olympus). Imaging chambers were sealed during imaging. Each microarray contained 8 negative controls (scrambled: not targeting any gene) and 12 positive controls showing different phenotypes. 
22,612 protein-coding genes have been targeted by at least 2 siRNAs each, in total 51,767 siRNAs. For each siRNA, there is data from at least 3 technical replicates, which created 182,191 quality controlled time-lapse experiments in total. Due to updates in the genome annotation, some reagents could not be mapped to the current ENSEMBL version. In total, the data set contains data for 17,816 protein-coding genes in 144,909 quality controlled time-lapse
experiments.  

HeLa cells are epithelial cancer cells which were derived from the adenocarcinoma of Henrietta Lacks in 1951. Being the first human cell line to survive \textit{ex vivo} for more than a few days, they are very appreciated from cell biologists as they are easy to grow and transfect. Indeed, this cell line is mentioned in approximately 0.3\% of PubMed abstracts although 64\% of its genome has a copy number greater than three~\cite{pmid23925245}. It is not motile as can be the case of other cell lines which are widely used in migration studies such as the epithelial metastatic breast-cancer derived MDA-MB-231 cell line (ATCC\up{\textregistered} HTB-26\texttrademark), or the epithelial metastatic lung-cancer derived NCI-H1299 (ATCC\up{\textregistered} CRL-5803\texttrademark). Indeed, it was the 16\up{th} slowest out of 54 in the first World Cell Race~\cite{pmid22974990}. Gene silencing in this background therefore makes it easier to identify migration suppressors, that is, genes whose silencing will enhance cell motility, rather than migration enhancers, given that HeLa basal cell motility is rather low.

\subsection{PCNA data set}
In section~\ref{sec:cellcycle}, another published dataset is mentioned, which is related to the study of cell cycle phases. It was published with~\cite{cellcognition} and is publicly available with annotations\footnote{\href{http://www.cellcognition.org/downloads/data}{http://www.cellcognition.org/downloads/data}}.

Briefly, HeLa cells were stably transfected for a red fluorescent chromatin protein (histone 2B fused to mCherry, H2B-mCherry) and a green fluorescent DNA replication factory (proliferating nuclear antigen fused to GFP, PCNA-mEGFP). Cells were seeded on LabTek chambered coverslips for live microscopy, and imaged for 48h with a time-lapse of 6 min (Plan10x, NA 0.5; Nikon). Cells were maintained at $37^\circ$C in humidified atmosphere of 5\% CO2 during imaging. Following this, cell nuclei were segmented using local adaptative thresholding, improved by a split-and-merge approach as described, and samples for the different cell cycle phases were manually annotated using the open-source software Cell Cognition~\cite{cellcognition}.
%
%Automated microscopy with reflection-based laser auto focus was performed on a Molecular Devices ImageXpressMicro screening microscope equipped with a 10× 0.5 numerical aperture (NA) and 20× 0.8 NA S Fluor dry objectives (Nikon) and recorded as two-dimensional time series. The microscope was controlled by in-house–developed Metamorph macros (PlateScan software package, available at http://www.bc.biol.ethz.ch/people/groups/gerlichd). Cells were maintained in a microscope stage incubator at 37 °C in humidified atmosphere of 5% CO2 throughout the entire experiment. 
\subsection{Drug screen}
In section~\ref{target_inference}, we analyze an unpublished time-lapse drug screen\footnote{Manuscript in preparation}. In this dataset, 25 drugs were screened for their effect on HeLa cells in similar experimental settings to that of the Mitocheck dataset.

Experiments were not conducted in the context of this PhD. They were performed at the Advanced Light Microscopy facility of the EMBL (Heidelberg, Germany) by Beate Neumann, Jutta Bulkescher and Thomas Walter. Briefly, HeLa cells were stably transfected for a green fluorescent chromatin protein (H2B-GFP). Cells were seeded on 384-well plates for live microscopy **h prior to imaging. Drug exposure occurred ***h prior to imaging. Finally, cells were imaged for 48h with a time-lapse of 30 min (Plan10x, ** NA; [MICROSCOPE BRAND]). Cells were maintained at $37^\circ$C in humidified atmosphere of 5\% CO2 during imaging.

\subsection{Xenobiotic screen}
In chapter 4, we analyze an unpublished time-lapse xenobiotic screen, for which we performed the experiments. In this dataset, 5 xenobiotics were screened for their effect on MCF-7 cells.

Briefly, MCF-7 cells (ATCC\up{\textregistered} Catalog N\up{o}HTB-22\texttrademark) were stably transfected for a red fluorescent chromatin protein (H2B-mCherry) and a green fluorescent membrane protein (myrPalm fused to GFP, myrPalm-GFP). Cells were seeded on 96-well plates 24h prior to exposure, and they were exposed to xenobiotics 24h prior to imaging. Finally, cells were imaged for 48h with a time-lapse of 15 min (Plan10x, 0.3 M27; Zeiss). Imaging plates were sealed during imaging. Our experimental procedures will be detailed in section~\ref{protocoles}.

\section{Software}
We use CellCognition \cite{cellcognition}\footnote{\href{http://cellcognition.org}{http://cellcognition.org}} for segmentation and object feature extraction in all projects. To store, manage and access
screening data, we use a previously published data format CellH5~\cite{Sommer2013}. All scripts are written in the programming language Python 2.7\footnote{\href{http://www.python.org}{http://www.python.org}}
using scipy~\cite{scipy}, numpy, scikit-learn, fastcluster~\cite{fastcluster}, rpy2 and statsmodels, and all plots were generated by matplotlib~\cite{matplotlib}. The Web-based user interface which was used for data visualization and sharing is based on Django\footnote{\href{https://www.djangoproject.com/}{https://www.djangoproject.com/}}, Linux-Apache web-server, mod\_wsgi and SQLite\footnote{\href{https://sqlite.org/}{https://sqlite.org/}}.
The R statistical function stats.p\_adjust was used for adjusting p-values according to the Benjamini-Hochberg procedure~\cite{Benjamini1}. Finally, CPlex\footnote{\href{http://www-01.ibm.com/software/commerce/optimization/cplex-optimizer/}{http://www-01.ibm.com/software/commerce/optimization/cplex-optimizer/}} was used for optimization in the tracking procedure.%introduction
\clearpage
%\newcommand{\tw}[1]{\textcolor{red}{\bf [TW:~#1]}}

\lhead{Chapter 2 - \emph{A generic methodological framework for studying single cell motility}} % Set the left side page header to "List of Figures"
\chapter{A generic methodological framework for studying single cell motility}
\label{chap:motiw}
This chapter was published in~\cite{motiw}.
\begin{table}[!ht]
\begin{tabular}{|l|}
\hline
~\\

\parbox{15cm}{\textbf{Résumé - Un cadre méthodologique général pour l'étude de la motilité cellulaire individuelle~}(see \textit{infra} for English text)}\\
~\\
\parbox{15cm}{Il existe de nombreux tests de motilité cellulaire, dont la majorité fournissent des informations à l'échelle d'une population de cellules (comme le test de la blessure). Toutefois, l'information à l'échelle individuelle est cruciale, car elle permet de détecter l'existence de sous-populations de cellules en terme comportemental.

Une seule autre étude de motilité cellulaire individuelle à haut débit a par conséquent été publiée, qui se base sur les empreintes des cellules sur un tapis de polymère~\cite{pmid25774502}. Dans ce chapitre, nous présentons la première approche méthodologique pour étudier la motilité cellulaire individuelle dans des données de vidéomicroscopie d'un crible à grande échelle.

Ce workflow, MotIW (pour \textit{Motility Integrated Workflow}), est constitué des étapes suivantes. Après l'acquisition des données, la segmentation et la description des objets sont réalisées à l'aide du logiciel libre \textit{Cell Cognition}~\cite{cellcognition}. Le suivi cellulaire est ensuite réalisé. Pour ce faire, nous nous sommes inspirés de~\cite{lou}, qui formule le suivi de cellules entre deux images consécutives comme un problème d'apprentissage structuré. Le suivi cellulaire permet de résumer chaque expérience comme un ensemble de trajectoires de cellules, qui sont décrites à l'aide d'un ensemble original de 15 descripteurs. Enfin, les distributions de ces descripteurs permettent de caractériser une expérience : elles sont utilisées dans un test statistique multi-varié que nous avons conçu afin de déterminer si la motilité cellulaire individuelle y est différente de celle des expériences contrôles. Le workflow permet donc d'aller d'un ensemble de molécules test à une liste des molécules modifiant significativement la motilité cellulaire. Enfin, nous montrons dans la dernière partie l'intérêt et le pouvoir de notre méthode en l'appliquant à un jeu de données simulé.}\\
~\\
\hline
\end{tabular}
\end{table}
\clearpage
\section{Studying single cell motility in a HT setup}
% Introduction: cell motility

Cell \textit{migration} describes "any directed cell movement within the body", as according to~\cite{pmid22940039}. On the other hand, cell \textit{motility} more broadly encompasses any cell movement which is active, i.e. energy-consuming. Cell motility plays a key role in many physiological processes including embryonic development or immune response~\cite{pmid18711433}, and is also involved in pathological processes such as fibrosis and metastasis. The latter is dependent on the ability of cancer cells to migrate, both as single cells and collectively~\cite{pmid16888756},~\cite{pmid20460404}, which highlights the need to understand the molecular basis of both processes.
%\textcolor{blue}{Also say something about cell motility at different time scales?}

% Assays for cell motility
\paragraph*{Cell motility assays\\}
Many \textit{in vitro} assays have been specifically designed to study cell motility~\cite{pmid16888756},~\cite{pmid22940039}. The most classic methods are wound healing assay, cell exclusion zone assay, and trans-well migration assay. They measure the ability of cells to migrate into some free space (the wound, the exclusion zone) or a new chamber, in a limited amount of time. These assays have the advantage of being widely known; as an example, the Boyden assay is a trans-well migration assay which was introduced in 1962~\cite{pmid13872176}. More recently, particle-coated plates were developed, in which particles are phagocyted by cells as they move~\cite{pmid329998}.
In this assay, a single picture is taken at the end of the experiment,
showing the path that was cleared by the moving cell. Analysis of
these images is therefore equivalent to the analysis of the temporal
projection of single cell trajectories.  
% The analysis of cell imprint is equivalent to that of its trajectory. 
%This method was recently adapted to be used in a HT setup~\cite{pmid18213366}. STUDY ONLY ON 55 GENES

\paragraph*{}
Live-imaging data can be obtained from experiments using any of these methods (except the trans-well migration assay in its classical version). Data at single cell level could therefore theoretically be obtained. But the classic assays are most of the time used as endpoint assays: the experimenter is focused on getting aggregated data at the level of the cell population (e.g. wound closure time or percentage of cells staying in the upper compartment of the well).

However, single cell characteristics are relevant: they enable to detect patterns which are not visible at the population level, such as the existence of cell subpopulations with regard to their motility behaviours~\cite{pmid24324630},~\cite{pmid25129619},~\cite{pmid25799384},~\cite{motiw}. This finds an application in drug design, as one might want to target specific populations~\cite{pmid15539606},~\cite{pmid20461076}. The little use of single cell motility assays is probably due to both cultural and technical reasons: people are used to proceeding at population level, and univariate analysis is easier than multivariate analysis. Furthermore, single cell motility studies in live cell imaging data have so far been limited to low-to-medium throughput~\cite{pmid21423205},~\cite{pmid25799384}. While this limitation has in principle be alleviated recently~\cite{pmid20360735}, live cell imaging still remains a relatively expensive technique and produces large amounts of data, thus requiring an appropriate infrastructure, both for imaging and IT. 
%On top of the technical and cultural reasons which were mentioned supra, a third bottleneck can be added: acquiring large-scale data is expensive. 
% Neumann ~\cite{pmid20360735}

\paragraph*{HT motility studies\\}
As a consequence, there are only few automatic workflows for the comprehensive analysis of single cell trajectories including tracking, statistical analysis and data mining, applicable to HCS data: ~\cite{pmid25774502} analyses temporal projections of single cell trajectories, as observed by cell imprints (see above). Information on membrane dynamics is indirectly inferred from these data, but the data is not informative about direct movement features, such as instantaneous speed, curvature or, more importantly, resting time and speed variations. The methods published in \cite{pmid24324630} are based on manual tracking and do therefore not scale easily to HCS. 

%very little completely automatic workflow, from tracking to statistical analysis, were developed for the analysis of single cell trajectories in 2 dimensions: only ours and~\cite{pmid25774502} could be found\footnote{\cite{pmid24324630} is based on manual tracking.}. The latter approach works on cell imprints as described earlier. It enables them to study membrane dynamics which are recorded in cell imprints; however they do not get any detailed temporal information such as speed, curvature, or more importantly resting time and speed variations.

In this chapter, we present MotIW (\textbf{Mot}ility study \textbf{I}ntegrated \textbf{W}orkflow). A generic methodological framework, MotIW enables to quantitatively study cell motility at single cell resolution in HT time-lapse data in an unsupervised way. It consists of cell tracking, cell trajectory mapping to an original feature space, and outlier experiment detection according to a new statistical procedure (cf figure~\ref{workflow}). We show the power of our method in section~\ref{sec:simulation} by applying MotIW to simulated data, which allows us to estimate recall and precision to be expected on real data. We then apply this workflow to a previously published genome-wide screen by RNA interference (RNAi) and live cell imaging, the Mitocheck dataset, in section~\ref{sec:mitocheck} of chapter~\ref{chap:reuse}.

\begin{figure*}[ht]
\centerline{\includegraphics[scale=0.3]{figures/Walter_295_fig_1.png}}
\caption{Overview of MotIW}
\label{workflow}
\end{figure*}  

\section{MotIW overview}
\label{sec:workflow}
In this section, we present MotIW, our workflow for the automatic and
quantitative analysis of single cell motility in video sets from
time-lapse microscopy-based screens. Figure~\ref{workflow} summarizes
its different steps. 
Briefly, for each video nuclei are segmented and features are
extracted as published previously \cite{Walter2010},\cite{cellcognition}. Cells are tracked using a
machine-learning based tracking procedure, described in
section~\ref{sec:celltracking}. The trajectories are then mapped to a
feature space described in section~\ref{sec:features}. Presented in
section~\ref{sec:stats}, an original statistical procedure then
enables the detection of experiments in which single cell motility is
significantly different than that in control movies. Finally,
section~\ref{sec:simulation} describes the simulation of trajectories
which allows us to validate the performance of the workflow.

%\subsection{Object segmentation}
\subsection{Segmentation and tracking}
\label{sec:motiw_seg}
\label{sec:celltracking}

%
The first step of the workflow is the establishment of single cell
trajectories: we want to follow individual cells over time and record
their spatial displacements. Many methods have been proposed in the
image analysis and computer vision communities for the automatic
tracking of individual objects. When it comes to the tracking of
biological objects, such as cells or particles (e.g.
single molecules or vesicles), there are two main approaches. On the one hand, it is possible to associate pre-segmented objects in consecutive frames, e.g. ~\cite{lou},
~\cite{citeulike:11229467}, ~\cite{Chenouard2014}. % the object-association approach associates pre-segmented objects in consecutive frames. 
On the other hand, the deformable model approach relies on identifying and modeling objects in the first frame, and linking them to objects in consecutive frames by updating the models, e.g. \cite{pmid12585703}. Methods also differ in the amount of prior knowledge which is hard-coded in the tracking model. For example,~\cite{pmid18656418} assumes that there are three types of
cell motion: Brownian motion, migration at a constant speed, and
migration at a constant acceleration. At the other
extreme,~\cite{pmid16043363} formulates tracking as an optimization
problem, where a cost function of particle matches, typically depending on distance and
intensity moments, is minimized.  
In particular, no constraint - except for maximal speed - on
the type of movement is imposed. 

We chose to keep segmentation and tracking steps independent: objects are identified before the establishment of object temporal correspondences by our tracking model. While the deformable model approach is in principle appealing as it jointly solves segmentation and tracking, it relies on a high time resolution and is therefore less generally applicable. Furthermore, as segmentation and tracking are not independent in this approach,
the resulting method is necessarily less modular.  

% Segmentation
\subsubsection{Segmentation}
Segmentation of nuclei is in principle a relatively simple problem, as
nuclei appear as bright objects on a dark background. The
main difficulty arises when two or more nuclei come in close proximity
to each other and are therefore segmented as one single object. Classically,
this problem is solved by splitting objects after the first
segmentation, e.g. \cite{cellcognition}. 
Briefly, a distance map of the binary segmentation is
calculated, where we assign to each pixel its distance to the closest
background pixel. If objects are touching, this typically generates
important concavities in the resulting binary shape, thereby producing
prominent maxima in the distance map. The final split is generated by
calculating the Watershed transformation on the inverted distance
map. To avoid false splits, the distance map is typically preprocessed
by either morphological or linear filtering. The problem of this
strategy is that non-convex shapes may be also split, and consequently
the detection of multi-nucleated morphologies will be more
difficult \cite{Walter2010}. Nevertheless, we chose to start from this segmentation
strategy, as implemented in CellCognition, as the main purpose of this
study is to track nuclei and to analyze spatial trajectories, which
will be eased by splitting the nuclei. 

%\subsection{Cell tracking}
%\label{sec:celltracking}
%\paragraph{mettre ou pas ?}
%There are two main parameters governing the design of a cell tracking
%algorithm. The first parameter is discrete and linked to the algorithm
%input. The deformable model approach is based on the images
%themselves. It relies on identifying and modeling objects on the first
%frame, and linking them to objects in consecutive frame by updating
%the models, e.g. \cite{pmid12585703}. On the other hand, the object
%association approach associates pre-segmented objects in consecutive
%frames, e.g. \cite{lou}.   

%The second parameter is continuous and describes how much prior knowledge is hard-coded in the tracking model. For example,~\cite{pmid18656418} assumes that there are three types of cell motion: Brownian motion, migration at a constant speed, and migration at a constant acceleration. At the other extreme,~\cite{pmid16043363} defines a cost function which particle matching should minimize, which is based on distance and intensity moments.
%We are willing to apply the current workflow to multiple screens. Since specific biological data may demand specific segmentation algorithms, the second approach was chosen. Indeed, it enables the workflow to remain as modular as possible.

\subsubsection{Cell tracking by supervised learning}
Cell tracking should be able to face several challenges which are
common in videos from high content screens. They
include high population density in each picture, high phenotypic
inter-cell variability, and possibly low time resolution between
successive images. Furthermore, the algorithm has to handle
apparitions, disparitions, divisions and fusions. Cells can indeed disappear, e.g. when they move outside the field of view or lose adhesion. They can also appear, for instance if they enter the field of view or more rarely if the expression of their fluorescent marker increases. Finally, they can fuse or seem to fuse, for example when a nucleus moves on top of another, or if two nuclei are still connected by chromosome bridges. %Fusions correspond to touching nuclei which were not correctly split by the segmentation algorithm, nuclei moving on top of each other, or segregation problems, where nuclei seem to be separated but are in reality still connected by chromosome bridges, which becomes apparent a few timepoints later when the nuclear envelope has reformed. 
%This latter event results from occlusion or segmentation errors; it can also be observed following the formation of chromosome bridges.
%Different data sets usually require different segmentation algorithms. For sake of modularity, we therefore prefer to keep
%segmentation and tracking steps independent. 

To be applicable in a screening context, we cannot \textit{a priori}
model cell motion, as such hypotheses are bound to break in the
presence of phenotypes. Indeed, the impact of chemical exposure on
cell motion is not known. As we also wished to avoid any manual parameter tuning, we extended a non-parametric structured learning approach from~\cite{lou}. %Briefly, tracking cells in two consecutive frames can be seen as a bi-partite graph matching problem:
%two sets of nodes have to be connected. Each node is
%characterized by a set of features, allowing for the definition of a
%cost function for linking two nodes. The solution can therefore by
%found by finding the node association with minimal cost under the
%constraint that all nodes have been appropriately considered. 
%The information which will be learned from
%a set of user-annotated cell associations is the relative importance
%of node features in determining the right matching. 

We first characterize each cell nucleus in each image by a set of 239 shape and texture features on the one hand (\cite{Walter2010}, ~\cite{cellcognition}), and geometric features on the other hand (its distance to the border, its position in the image, and the orientation of its main axis). The goal of cell tracking in this approach is to match cells in successive images, by assigning them the most likely instant temporal behaviour in the set $\mathbf{E}$ = \{\textit{move},~\textit{appear},~\textit{disappear},~\textit{split in 2 or 3},~\textit{merge at 2 or 3}\}. %The relative importance of match features for cell matching is the information which will be learned. 
All possible matches between cells in consecutive frames are exhaustively considered, subject to
distance thresholding. Match features are: 
\begin{itemize}
\item the absolute difference of shape and texture features if the event is \textit{move}, \textit{split}, \textit{merge}, the object features otherwise
\item the geometrical distance between object at time $t$, $Obj_{i, t}$ and object at time $t+1$, $Obj_{j, t+1}$, if the event is \textit{move}, \textit{split}, \textit{merge}, the minimal distance to the image border otherwise,
\item the angle between $Obj_{i, t}$ and the elements of $Obj_{j, t+1}$, if the event is a \textit{split} (angle $\alpha$ on fig.~\ref{fig:angles}),
\item the angle between the main axis of $Obj_{i, t}$ and $Obj_{j, t+1}$ weighted by their average eccentricity, if the event is a \textit{move} (angle $\beta$ on fig.~\ref{fig:angles}).
\end{itemize}

\begin{figure}[ht]
\centerline{\includegraphics[scale=0.27]{figures/angles_description.png}}
\caption{Illustration of angular match features.}
\label{fig:angles}
\end{figure}


%The optimal matching is defined as the one that maximizes a likelihood function, which depends on match features and on weights for each feature:
The optimal object matching $\widehat{z}(t)$ comes down to bi-partite graph matching: it is solved by maximizing a likelihood function $L$ which depends on the weights $w$ of match features and the match features $f^e_{i, j}$, subject to the constraint that all objects are matched in both frames (cf equation~\ref{formulation}).
%\begin{equation}
%\widehat{z}(t) = \argmax_{\substack{z(t)}}~ \mathbf{L}(z(t) ; w) 
%\label{formulation}
%\end{equation}
%where 
%\begin{align*}
%\mathbf{L}(z(t) ; w) & = \sum_{\substack{e\in \mathbf{E}\\Obj_{i, t}\\Obj_{j, t+1}}} <w^e, f^e_{i, j}> z^{e}_{i, j}(t)\\
% & s.t.\ \ \forall i \sum_{\substack{e\\Obj_{j, t+1}}} z^{e}_{i, j}(t) = 1\ \\ 
% & and\ \ \forall j \sum_{\substack{e\\Obj_{i, t}}} z^{e}_{i, j}(t) = 1
%\end{align*}

\begin{align}
\widehat{z}(t) & = \argmax_{\substack{z(t)}}~ \mathbf{L}(z(t) ; w) \\
 & s.t.\ \ \forall i \sum_{\substack{e\\Obj_{j, t+1}}} z^{e}_{i, j}(t) = 1\ \\ 
 & and\ \ \forall j \sum_{\substack{e\\Obj_{i, t}}} z^{e}_{i, j}(t) = 1
\label{formulation}
\end{align}
where 
\begin{equation*}
\mathbf{L}(z(t) ; w) = \sum_{\substack{e\in \mathbf{E}\\Obj_{i, t}\\Obj_{j, t+1}}} <w^e, f^e_{i, j}> z^{e}_{i, j}(t)\\
\end{equation*}

The weights $w$ are learned by a structured support vector machine using annotated trajectories, following the formulation of~\cite{lou} (drawing on~\cite{Tsochantaridis}). The likelihood maximization, an integer linear programming (ILP) problem, is solved by IBM Cplex.
% This results in a bi-partite graph matching problem; it is solved by an integer linear program. The weights are learned by a support vector machine using annotated trajectories, following the formulation of~\cite{lou}.

The extension compared to~\cite{lou} lies in the choice of match
features. Furthermore, we enabled the tracking model to learn from
partial annotations of different experiments\footnote{However, this is
  not learning from partial annotations in the sense
  of~\cite{loupartial}. Indeed, in our implementation of~\cite{lou},
  the user chooses a subset of cells which has to be annotated on all
  movie frames. In~\cite{loupartial}, the user can choose both a
  subset of movie frames and a subset of cells (s)he wishes to
  annotate on those frames.}. This permits the user to integrate
examples from both control and non-control experiments in the training
set, which is crucial to guarantee that the model can efficiently
track cells in all conditions. We also added three object division and
fusion to $\mathbf{E}$. This is important in a screening context,
where aberrant cell divisions may occur. We also implemented a more time-efficient computation of match hypotheses
using kd-trees.

\subsubsection{Validation of MotIW cell tracking model}
To validate MotIW's cell tracking procedure, we compare it to Cell Cognition's constrained nearest-neighbour (CNN) tracking algorithm, and to~\cite{jaqaman} as implemented in Cell Profiler \cite{Carpenter2006}. We have chosen these two approaches for benchmarking, as they are available in popular High Content Screening software. \cite{jaqaman} views tracking as a linear assignment problem (LAP). It starts by computing 1-to-1 matches in consecutive frames, which produces tracklets. It then connects these tracklets by solving an optimization problem which is global both in time and space (2D-space and time duration of the experiment). For performing this optimization, that is, for choosing when to perform tracklet merges, splits, appearances and disappearances, it uses user-defined costs. 

Our training set consists of approximately 32,000 matches, among which
0.5\% \textit{appear}, 0.5\% \textit{disappear}, 1\% \textit{merge}
and 2\% \textit{split}. Data was taken from the Mitocheck data
set, and in particular from both control experiments \textbf{and}
experiments as selected by~\cite{pmid20360735} for being significantly
different from controls regarding nuclear morphology. This ensures
that the algorithm also works in the presence of phenotypes. One may
nevertheless be willing to perform cell tracking before having
performed any statistical analysis. In this case, learning from
control experiments only slightly diminishes the cell tracker
performances, albeit non significantly. 

To establish the data set, we first performed tracking with
CellCognition's CNN tracking and manually corrected for mistakes made
by the algorithm. As in most cases, even such a simple tracker is able
to find the correct assignment, we found this procedure much less time
consuming than annotating all correspondences from scratch. 
As shown in Table~\ref{accuracy}, MotIW outperforms the other two
methods as measured by the average accuracy on the five movement
types. Note that these are not the overall accuracies of correct
assignments, which are much higher for all three methods. 
As can be seen on fig.~\ref{details}, all three methods show similarly good
performances on \textit{move} events and have therefore similar
overall (pooled) accuracies. The contribution of the learning approach
is most important for the other events, such as cell division, when
object matching is less trivial. 

In particular, it is interesting to see that this method even outperforms
~\cite{jaqaman}, which relies on an optimization scheme in space and
time, i.e. optimizes not only the assignments between two consecutive
frames, but on the entire video sequence. While this might seem
surprising at first sight, it is explicable by the fact that the
latter approach was developed for particle tracking. Particles have only few distinguishing features, such as
intensity and size. It is therefore feasible to manually define and tune a cost function for particle tracking  based on these features only. It is nevertheless hardly feasible for larger feature sets, which are necessary to track more complex objects in terms of texture and shape. %In this case, similarity between objects becomes a more important parameter for finding the correct correspondences.

In the future, it will be interesting to see whether this result can
still be improved by an optimization scheme in time. Another promising
strategy for future investigation will be to couple segmentation
with tracking without relying on a deformable model approach. In
particular, we could argue that split algorithms can provide us with
alternative hypotheses on segmentation. The best combination of
segmentation and tracking can then be found by global optimization in
time. 

Altogether, we conclude that the tracking procedure is sufficiently
accurate to generate single cell trajectories. 

%Nevertheless, an optimization in time might still be applicable and
%still improve the tracking result. 


%In the latter case, it cannot be based on user-defined costs without
%considering any characteristics of the object as
%in~\cite{jaqaman}. 
%Indeed, the latter approach was developed for
%particle tracking, where one is not able to quantitatively describe,
%e.g., the object's mean intensity or area. Therefore, although
%the~\cite{jaqaman} approach performs an optimization which is global
%in time and space, using object quantitative characteristics and
%learning their importances on the data proves to be even more
%efficient for the final tracking accuracy. 

\begin{table}[!ht]
\centering
\caption{Mean recall and precision on all types of matches $\mathbf{E}$ (10-fold cross-validation)\label{accuracy}}{
\begin{tabular}{l|c|c}
Algorithm & Mean recall (\%) & Mean precision(\%)\\
\hline
CNN & 72.7 & 62.8 \\
\cite{jaqaman} & 78.3 & 73.0 \\
MotIW  & 91.1 & 91.5 \\
\end{tabular}}

\end{table}
\begin{figure}[ht]
\centerline{\includegraphics[scale=0.4]{figures/Walter_295_fig_2.png}}
\caption{Details of tracking precision and recall according to event types}
\label{details}
\end{figure}



\subsection{Trajectory features}
\label{sec:features}
Once cell tracking has been performed, each experiment is summarized
as a set of cell trajectories in the two-dimensional space over
time. Instead of analyzing these point sequences directly, we
calculate for each trajectory a set of relevant features which will allow us to
represent each trajectory to a point in this feature space. 
For instance, cell speed is an important characteristic of cell motion, and it
is certainly the most studied one. However, it makes sense to also
describe other aspects of movement. For instance, if we consider the example
trajectory of fig.~\ref{fig:01}, one might also be interested in
quantifying the percentage of time which is spent in each englobing
blue ball, or whether diffusion would have been an appropriate model
for this particular cell. A multivariate study of cell trajectories is
therefore relevant to capture all the information which they
contain. For this, a set of 15 features was assembled, partly from previous
publications on quantitative motility analysis, partly newly
designed. 

Robust and precise features are needed to account for the partial
stochasticity of cell migratory behaviour. We use three types of
features, as detailed in table~\ref{features}. 
\begin{figure}[!tpb]%figure1
\centerline{\includegraphics[scale=0.22]{figures/Walter_295_fig_3.png}}
\caption{A cell trajectory with notations}
\label{fig:01}
\end{figure}
\begin{table}[!h]
\centering
\caption{Cell trajectory features and their formulas. Notations: $(m_t)_{t=1\ldots T}$, time sequence of cell 2D positions. T, track time duration. P, total track length \label{features}}
{\begin{tabular}{ll}\hline
\multicolumn{2}{c}{Particle motion features}\\
\hline
Diffusion coefficient & According to~\cite{pmid16043363} \\
Diffusion adequation & Correlation between MSD(t) and $t$ \\
Movement type & According to~\cite{pmid16043363} \\
Englobing ball number & See text\\
Track entropy & See text \\
\hline
&\\
\multicolumn{2}{c}{Other global features}\\
\hline
Convex hull area & - \\
Effective path length & $L=\|m_{T}-m_1\|_2$ \\
Effective speed &  $L/\sqrt{T} $ \\
Largest move & - \\
Straightness index & $\sqrt{T} L/P $ \\
Track curvature & See text \\
\hline
&\\
\multicolumn{2}{c}{Averaged local features}\\
\hline
%\parbox{2.5cm}{\raggedright Mean persistence} &$\dfrac{1}{T-1}\sum \cos(\alpha_{t+1}-\alpha_{t})$\\
\parbox{2.5cm}{\raggedright Mean squared displacement (MSD)} & $\dfrac{1}{T-1}\sum \|m_{t}-m_{t-1}\|_2^2 $\\
\parbox{2.5cm}{\raggedright Mean signed turning angle} & $\arctan(\dfrac{\Sigma \sin(\alpha_{t+1}-\alpha_{t}) }{\Sigma \cos(\alpha_{t+1}-\alpha_{t}) }) $\\
\hline
\end{tabular}}{}
\end{table}

\subsubsection{Particle motion features}
This group of features
encompasses the diffusion coefficient and the movement type, which
were in the first place used to study particle motion
(see~\cite{ferrari},~\cite{pmid16043363} for one of its applications to single particle motion in Biology).  

Let us consider a particle, and note $m_t$ its position at timepoint $t$. The moment of
order $p$ of this particle, $<d^p>$, can be computed according to the following formula: 
\[
<d^p>= < \|m_{t}-m_{t-1}\|_2^p >_t
\]
%\textcolor{red}{\bf First, $<d^p>$ is an expectation, not a   sum. Either you write 1/N or E(). Second, it is not clear on what   you average. On particles or on different time points for the same   particle. I know that this is one weakness of these features, but   still you need to tell what you do.} 
For large $t$, it is proportionate to $t^{\gamma_p}$ for
most dispersive processes~\cite{ferrari}. Assuming that $\gamma_p$ is
proportionate to $p$ (i.e. that the particle movement is strongly
self-similar), the constant $\gamma = \gamma_p /p$ (hereafter the particle's
\textit{Movement type}) quantifies how directed the particle motion is. If
$\gamma$ is equal to $1$, the movement is perfectly directed, whereas
if $\gamma$ is equal to $0.5$, it is perfectly diffusive. Between
$0.5$ and $1$, the movement is super-diffusive, whereas below $0.5$ it
is called sub-diffusive.  

Furthermore, assuming $ \gamma = 0.5 $, the constant linking $<d^2>$ (that is, the mean squared displacement) and $t$ can be computed: it is the \textit{Diffusion coefficient}. The \textit{Diffusion adequation} is the correlation coefficient between $<d^2>$ and $t$, hence measuring how well the diffusive model applies to the track at hand.
\paragraph*{}
Here, we present two newly designed features to characterize the
alternance between periods of diffusive motion and periods of directed
motion: the track entropy and the englobing ball number. 

It has been observed that cell motion in 2D alternates between
diffusive and directed motions (in the absence of any perturbation or
chemical gradient). %REF\alcomment{can't really find any references at   the right time scale...} \twcomment{We can omit the citation}. 
The feature track \textit{Entropy} was designed to measure how the time
sequence of 2D cell positions $m_t=(x_t, y_t)$ distributes in balls of radius $r$. This will be computed greedily by recursively searching the center of a new ball of radius $r$ among the set of remaining track points, that will contain the biggest number of them. This
feature is calculated according to the following procedure, for each
track of time duration $T$: 
\begin{enumerate}
\item $S = \{1, \ldots, T\}$ 
\item while $S\neq\{\}$:\\
\ \ \ \ i. do $t^\star\leftarrow \argmax_S card(B_r(t))$ \\ \ where
$B_r(t) = \{ i \quad | \quad \|m_i-m_t\|_2 \leq r \,\,\, \mbox{and}$ \\
\indent $\quad \quad \quad \quad min{(\|m_{i-1}-m_t\|_2,\|m_{i+1}-m_t\|_2)} \leq r \}$ \\
%\ where $B_r(t) = \{ i | dist((x_i,y_i), (x_t, y_t)) \leq r, $
%\[dist((x_{i-1},y_{i-1}), (x_t, y_t)) \leq r\ \cup\ dist((x_{i+1},y_{i+1}), (x_t, y_t)) \leq r \} \]
ii. do $S \leftarrow S \backslash B_r(t^\star) $

\item Compute the track \textit{Entropy} according to the following formula:
\end{enumerate}
\begin{equation}
Entropy_r= -\dfrac{1}{T} \sum_{\substack{B_r}} \dfrac{card(B_{r})}{T} \log(\dfrac{card(B_{r})}{T})
\end{equation}

The track \textit{Entropy} measures the entropy of the distribution of track positions in balls of radius r. To deal with cells whose trajectories are concentrated in space, but were not concentrated in time, the constraint is imposed that these balls shall contain only consecutive positions in time. The englobing \textit{Ball number} is the number of balls of radius $r$ that contain all track positions. It is normalized by the square root of $T$ to be independent of the track time-length $T$.

Different radii may be relevant for different data (depending of,
e.g., the experiment time-lapse, the pixel size or the cell type). We
chose to use two different radii $r_1$ and $r_2$ with $r_1<r_2$, to
incorporate information about cell trajectories on two different
time-scales. $r_1$ and $r_2$ were manually chosen, such that for the Mitocheck data set, the corresponding features are neither constant over a large number of trajectories nor too correlated. They respectively correspond to approximately $2.5\mu m$ and $12\mu
m$. In the following, the features \textit{Entropy $i$} and 
\textit{Ball number $i$} correspond to radius $r_i$.  

%They correspond
                                %to approximately $2.5$ microm and
                                %$12$ microm on the Mitocheck
                                %dataset. 

\subsubsection{Other global features}
We further defined the following global descriptors of cell
trajectories: the cell's \textit{Largest move} along the trajectory, its track
\textit{Convex hull area} and its average \textit{Track
  curvature}. The track \textit{Convex hull area} is the area of the
convex hull containing all track points, as coloured in green on
fig.~\ref{convexhull}. It is normalized by the square root of the
track time-length. Following~\cite{pmid18213366}, this feature enables
us to have an idea of the area which the cell has visited during its
trajectory, although it does not exactly indicate the area which its
cytoplasm has covered. Finally, for each trajectory and each time-point $t$, an orthogonal regression is performed on $\{(x_i, y_i) | i \in \{t,\ldots, t+\Delta_t\} \}$ using orthogonal distance regression ($\Delta_t=10$). The mean \textit{Track curvature} of the trajectory is the average of all regression sums of squares. 
\subsubsection{Averaged local features}
Finally, two features are averaged local features, which are the cell
means squared displacement
(\textit{MSD}) and its \textit{Mean signed turning angle} (see table~\ref{features} for the formulae). 
\begin{figure*}[!ht]%figure1
\centerline{\includegraphics[scale=0.25]{figures/convex_hull.png}}
\caption{Convex hull of the example track from figure~\ref{fig:01}}
\label{convexhull}
\end{figure*}
\begin{figure*}[!tpb]%figure1
\centerline{\includegraphics[scale=0.45]{figures/Walter_295_fig_4.png}}
\caption{Heatmap showing trajectory feature similarities on a subset of the Mitocheck dataset (1.1 million trajectories coming from detected motility hit experiments according to MotIW). The dengrograms were obtained using the \textit{Ward} method and the euclidean distance between feature correlations.}\label{correlations}
\end{figure*}
\subsubsection{Feature set evaluation}
Track time-length is an irrelevant random variable for studying single cell motility, which could bias some features. Therefore, we ensured that they are not significantly correlated with this parameter: the correlation between track time-lengths and features is maximal for the \textit{Effective space length}, where it is equal to approximately $30\%$ (on a subset of the Mitocheck dataset, data not shown).
%\alcomment{leave next paragraph here or put it in results?? leave the plot ?? leave pca info to corroborate ?}
%\twcomment{True: it is a result of the analysis to the mitocheck
%  data. Nevertheless I think it can also stay here. Yes: we leave the
%  plot and the PCA.}

Figure~\ref{correlations} shows the correlation matrix for the
extracted features. 
%A heatmap shows feature similarities on figure~\ref{correlations}. It
%was obtained using the \textit{Ward} method on the euclidean distances
%between the feature correlations on a subset of the Mitocheck
%dataset. %It indicates that some features are highly correlated.  
One group of highly correlated features are visible in the bottom-left
corner of the heatmap, which encompasses speed-related features. The
existence of two feature subgroups within this group can be explained
by the following observation: the first group of features, from
\textit{Ball number 1} to \textit{MSD}, is linked to cell
instantaneous displacements, whereas the second group, from
\textit{Effective speed} to \textit{Entropy 2}, is linked to its
displacements on the whole trajectory.  

The other correlations can as well be explained by feature definitions. As an example, the anti-correlation between \textit{Mean signed turning angle} and \textit{Movement type} can be interpreted as follows: a low signed turning angle is indicative of correlated motion, which is super-diffusive and translates into a high \textit{Movement type}.

Fig.~\ref{correlations} indicates that there are less degrees of
freedom than features, which was verified by a principal component
analysis (PCA). On the same trajectory subset, approximately 95\% of
the variance is explained by the first seven principal components. 

\subsection{Statistical procedure}
\label{sec:stats}
\subsubsection{Trajectory quality control}
 Prior to statistical analysis, a trajectory quality control is performed. First of all, trajectories resulting from object fusion are discarded. Indeed, trajectories resulting from a fusion are most of the times cell cluster trajectories, rather than cell trajectories. Trajectories which are shorter than 10 frames are also discarded, to ensure that features such as the diffusion coefficient are computed on a sufficiently large number of points. Finally, because we are interested by single cell motility rather than collective motility, all trajectories with more than 5 neighbours in a perimeter of 50 pixels are deleted. This trajectory quality control ensures that cell clusters are not considered, and increases the dataset robustness. This quality controls eliminates $11.6\pm 13.3 \%$ (median$\pm$interquartile range) of cell trajectories per experiment, as estimated on a random subset of 1,000 experiments from the Mitocheck dataset.
\paragraph*{}
HT screening data is organized in batches of experiments which have
been acquired simultaneously. Each batch includes a set of negative
controls, i.e. conditions where no effect is expected. Due to a
non-negligible batch effect, an experiment can only be compared with
controls of the same batch in most of the cases. 

Let us consider an experiment $i$. Following to trajectory feature extraction, it can be summarized as a set of $\Theta$ feature distributions ($\Theta = 15$). The comparison of these distributions with those of controls from the same batch $B_i$, using Kolmogorov-Smirnov 2-sample test, provides a list of p-values $(p_\theta)_{\theta=1\ldots\Theta} $. 

A final statistic $S_{i}$ combining the p-values of all features is  obtained by Fisher's formula :
\begin{equation}\label{FisherFormula}
S_{i}=-2 \sum_{\substack{\theta}}ln(p_\theta)
\end{equation}
As shown on fig.~\ref{correlations}, the features are not independent. Therefore, the distribution of this statistic under the null hypothesis does not follow a chi-squared law with $2\Theta$ degrees of freedom. To assess which values of this statistic should be considered as indicative of altered motility, a sample of the distribution of $S$ under the null hypothesis is then computed by comparing the control experiments  which were not used in the experiment-controls comparisons, with the other controls from the same batch.

In the absence of an explicit form for the null distribution, this sample allows to quantify the intra-batch variations of single cell motility features. The variations can be due to technical artefacts or biological variability. Then, the comparison of the distribution of $S$ statistics obtained from control-experiment comparisons, to the distribution obtained from control-control comparisons, permits the computation of empirical p-values. This enables the detection of hit experiments with regard to single cell motility.  

\subsubsection{False discovery rate control}
\label{sec:fdr}
False discoveries are controlled using the Benjamini-Hochberg procedure~\cite{Benjamini1}. Indeed, in the current case, approximately 150,000 statistical tests will be performed. Selecting experiments whose p-value is below 0.05 rigorously means that there will be 5\% false discoveries. Without any adjustment, 150,000 tests will produce 7,500 false discoveries, which is not acceptable. 

There are different ways to adjust p-values. If the statistical tests are independent, the Benjamini-Hochberg procedure~\cite{Benjamini1} is a procedure to adjust p-values and control the false discovery rate. If the tests are not independent, there are other procedures (see for example ~\cite{Benjamini2} or ~\cite{Roquain}). Here, we assume in a first approximation that the tests are independent. Rigorously speaking, they are not: experiments which are performed on the same day share some dependence, which may be the influence of the temperature, the pressure, the passage number of the cells or another experimental variable. We do the hypothesis that we can neglect this local dependence (384 experiments versus 150,000).

%The Benjamini-Hochberg procedure takes as an input a list of M p-values, ordered. Then for each p-value with rank k p(k) you are going to compute p(k, adj) = p(k)*M/k. According to [Benjamini and Hochberg, 1995] this will be <= to the false discovery rate. Neverthless, (p(k,adj)) is not necessarily monotonous with respect to k. As an example, if you have M=1,000 and the first 100 p-values are 0.0001, then the first adjusted p-value will be bigger than the 100th. Hence there is a correction which has to be made, for (p(k,adj)) to be monotonous with k. It's optimistic: here all adjusted p-values before the 100th will be set equal to the 100th. This explains that some p-values have exactly the same values: they were set equal to respect the monotonicity of the adjusted p-values with respect to the initial list of p-values.

\subsubsection{Formal statistical procedure}
This procedure is repeated $n$ times to ensure that the final p-value of an experiment $i$ does not depend on the choice of a specific subset of control experiments in its batch%IS THIS BOOTSTRAPPING? no because bootstrapping is sampling with replacement. Here what we do is sampling without replacement to be more robust ($n=5$)
. Here is its formalized description:
\begin{enumerate}
\item {\bf Compute a sample of statistic (\ref{FisherFormula})  under null hypothesis from control-control comparisons.}\\ For each batch $b$,
\item[] \ \ \
 For k in $\{1,\ldots,C_b(C_b-1)/2\}$, where $C_b$ is the number of controls of batch $b$ that passed the quality control
\item[] \ \ \ \ \ a. Randomly split the control experiments in two
  groups $A_{b,k}$ of cardinal $2$, and $B_{b,k}$ of cardinal $C_b-2$
\item[] \ \ \ \ \ b. For each control $j$ of $A_{b,k}$, compute the statistic $S^{0}_{b,k,j}$ (\ref{FisherFormula}) by comparing it to the pooled group of controls $B_{b,k}$

\item {\bf Compute statistics from experiment-control comparisons.} For computation time feasibility, only $n=5$ repetitions corresponding to $n$ splits of the controls set $(A_{b,k}, B_{b,k})$ are selected on each batch for experiment-control comparisons. 
\item[] \ \ \ a. For each repetition k in $\{1,\ldots,n\}$:
\item[]\ \ \ \ \ \ \ For each experiment $i$  belonging to a batch $b$, compute the statistic $S_{k,i}$(\ref{FisherFormula}) by comparing it to the pooled group of controls $B_{b,k}$
\item[] \ \ \  b. Combine distinct iterations. In order to be conservative, we chose the following approach:
\begin{equation}
S_{i} = {\max}_{\substack{k \in \{1\ldots n \}}} S_{k,i}
\end{equation}

\item  For each experiment $i$, compute the p-value $p_{i}$:
\[p_{i} =max \left( \dfrac{card( \{(b,k,j) | S^{0}_{b,k,j}\geq S_{i} \})}
{card( \{ (b,k,j) \})},
 \dfrac{1}
 {card( \{ (b,k,j) \})}\right) \]
\item For each experiment $i$, compute the adjusted p-value $p^\prime_{i}$ to control the false discovery rate (Benjamini-Hochberg procedure~\cite{Benjamini1})
\end{enumerate}
Each experiment is characterized by an adjusted p-value ; significantly different experiments are those whose adjusted p-values are below a certain threshold. An experimental condition is selected as being a hit if 50\% or more of its replicate experiments are significantly different. This is a way of ensuring reproducibility. It amounts to representing an experimental condition by the median of its replicate scores. Since the mean is sensitive to outliers, using the mean of its replicate scores instead of their median would lead to too many false positives.

%Therefore, in the absence of an explicit form for the null distribution, this procedure enables the computation of an empirical null distribution of p-values describing how single cell motility feature distributions vary from a control experiment to another. The $n$ iterations ensure that the final p-value $p_i$ of experiment $i$ does not depend on the choice of a specific subset of control experiments in its batch.
\section{Validation on a simulated screen}
\label{sec:simulation}
\subsection{Screen simulation}
In order to evaluate the performance of our workflow on data for which the groundtruth is
known, we designed a process to simulate a HT screening
experiment. 

In a first step, five types of single cell movements were designed, in
agreement with qualitative observations from the dataset: \textit{random}, \textit{fast-random}, \textit{curbed-directed}, \textit{flip-directed} and \textit{stop-and-go}~(see fig.~\ref{simulation}). 


Let $(d_t,\phi_t)$ be the polar coordinates of the difference vector $m_{t-1}-m_t$
of any two consecutive
points. For \textit{random} movement, $\phi_t$ is
chosen at random and the distance $d_t=\|m_t-m_{t-1}\|_2$ is
drawn from a normal distribution, whose parameters are estimated from
the data. The same holds for \textit{fast-random} with increased
distance $d_t$. For the \textit{curbed-directed} movement type, $d_t$
follows again a normal distribution as for \textit{random} movement,
but the angle is calculated as $\phi_t + \epsilon$ with $\phi_t =
\phi_{t-1} + \Delta \phi_t$, where $\Delta \phi_t$ and $\epsilon$
follow normal distributions, whose parameters are set manually to
visually match some observed trajectories. 

\textit{Flip-directed} and \textit{stop-and-go} are two composite types of movement, where the cells
alternate between different states. The dwelling times in the two
states are random integers with manually fixed ranges (which can be
different for the two states) and are drawn independently for each
trajectory. \textit{Flip-directed} movement corresponds to directed
movement ($\phi_t$ is drawn from a normal distribution) with a 180
degree flip for every state transition. Finally, \textit{stop-and-go}
movement alternates between slow random movement (where $\phi_t$ is
drawn from a uniform distribution)  and fast directed
movement (where $\phi_t$ is drawn from a normal distribution).  
\begin{figure}[ht]
\centering
%\centerline{\includegraphics[width=0.5\textwidth]{trajectories_6.pdf}}
\includegraphics[width=0.5\textwidth]{figures/Walter_295_fig_5.pdf}
\caption{Simulated trajectories: stop-and-go (green), flip-directed
  (red), random (orange), fast random (purple), curbed-directed (blue)}
\label{simulation}
\end{figure}

In a second step, we want to simulate movies (controls and experiments), i.e. sets of
trajectories. For this, we define five movie types with different
proportions of single cell movement types (cf supra). \textit{Normal}
movies account both for 
control movies and experiments in which cell motility is similar to
that of controls. They contain on average $80\%$ of \textit{random} trajectories,
and a mix of the four other trajectory types. This reflects our
observation that in real data, experiments and controls typically contain all possible types
of cell trajectories and that phenotypes are characterized in a shift
in percentage. All other movie types contain (on average) from 50 to 65\%
\textit{random} trajectories, the rest being completed according to
the movie type. For example, movie type \textit{fast} is composed of 30\% of
\textit{fast-random} trajectories, 60\% of \textit{random}
trajectories, and a mix of the three other trajectory types. 

The total number of trajectories in each movie was drawn at random
from real data in the following way: first, a batch is randomly chosen in the data
set. Then, we assign a permutation of the real trajectory numbers from
the experiments of the picked batch to the simulated positions. This enables to include potential batch effects in our simulated data. Furthermore, they match those from the dataset on which we developed this workflow, namely the Mitocheck dataset. The number of trajectories of each movement type in each movie is drawn from
the corresponding movie type multinomial distribution, where the
percentages were defined as described above. 

The third step was the simulation of approximately 50,000 experimental
conditions, which were distributed on 130 plates, and performed in
triplicate as in Mitocheck experimental
setup~\cite{pmid20360735}. For sake of simplicity, triplicates were
supposed to belong to the same movie type. On
each plate, between 5\% and 15\% of the experiments were selected to
be other than \textit{normal} movies. 
%
%
%\begin{table}[!t]
%\caption{This is table caption\label{Tab:01}}
%{\begin{tabular}{llll}\toprule
%head1 & head2 & head3 & head4\\\midrule
%row1 & row1 & row1 & row1\\
%row2 & row2 & row2 & row2\\
%row3 & row3 & row3 & row3\\
%row4 & row4 & row4 & row4\\\hline
%\end{tabular}}{This is a footnote}
%\end{table}

%\section{Experimental validation and application}
%\label{sec:results}
%radiuses $2.6 \mu m$ and $11.7 \mu m$\\
%n=5, combination: max

%We first evaluate MotIW on simulated data (see~\ref{sec:simu_results}). We then apply it to the whole genome-wide screen Mitocheck~\cite{pmid20360735}, which enables us to identify an ontology of 2D cell trajectories (see section~\ref{sec:mitocheck}).%In this latter screen, each gene was targeted by three different siRNAs, and each experimental condition was tried in triplicate.
\subsection{Application to a simulated screen}
%We simulated a whole HT screen, as detailed in section~\ref{sec:simulation}. 
Our workflow successfully recognized more than 98\% of the experiments, as detailed in table~\ref{simu_results}.
\begin{table}[!ht]
\centering
\caption{Results from the application of MotIW to simulated data\label{simu_results}}
{\begin{tabular}{l|c|c|}
 & Recall (\%)& Precision (\%)\\
\hline
\parbox{5cm}{Outlier experiment detection} & $99.2$ & $98.9$\\
\parbox{5cm}{Outlier condition detection} & $99.5$ & $100.0$\\
\hline
\hline
Trajectory clustering & $91.4\pm2.1$ &  $89.4\pm4.8$\\
\hline
\end{tabular}}{}
\end{table}

Our simulation pipeline was also used to estimate how useful the trajectory feature set is to capture the differences between different types of trajectory motion. %The same procedure as for real data was followed. 
500 samples of each trajectory type were simulated, and their features
extracted. A PCA was performed, after which we retained the eigth
first principal components, which explain approximately $95\%$ of the
data set variance. Finally, k-means was applied to the data set with
$k=5$. Many simulation parameters (e.g. each track length) are chosen
at random, and k-means' results depend on its initialization: the
procedure was therefore repeated 10 times. The results are presented
in table~\ref{simu_results}. Although distinguishing trajectory types
is subject to some errors, it shows that the whole pipeline is robust
enough to identify experiments in which cell motility is significantly
different. 

We therefore conclude that MotIW is capable of identifying trajectory
clusters in an unsupervised way. Hence, we are confident that this
methodological workflow will allow the identification of migratory behaviors in HT live cell imaging data sets. Moreover, we believe that the general strategy of identifying hit experiments prior to cluster analysis
might be useful for unsupervised approaches to HCS data. 

This strategy can indeed also be seen as a way of intelligently downsampling the data to a reasonable size. Compared to random sampling, it has the advantage of enriching the data set in extreme cases. In a supervised setting, this occurs as a natural result from the annotation process: the numbers of samples in each class of the training set rarely reflect their proportion in the whole dataset. The training set is therefore most of the times enriched in rare classes. This seems to be beneficial as well for unsupervised learning approaches to large biological data mining.




\clearpage
\lhead{Chapter 3 - \emph{High-content screening data as a resource}}

\chapter{High-content screening data as a resource}
\label{chap:reuse}
\section{Data re-use in Bioimage Informatics}
\paragraph*{}
\textit{Bioinformatics} solves biological questions using computer science methods: as put by Wikipedia, it "develops methods and software tools for understanding biological data". Hence it revolves around data.

Most of the time, biological data is produced by a laboratory or a consortium thereof in order to answer a specific question. As an example, the Mitocheck dataset~\cite{pmid20360735} was produced by the Mitocheck consortium to identify the human genes which are involved in cell division. 

However, there is never a single paradigm to look at a specific dataset. To go further, most datasets should be treated as a scientific resource to their parent laboratory and to the community, rather than a one-article commitment. Indeed, they can be re-used to either go further on the same question, either answer a different one. By \textbf{data re-use}, we target studies which will be based, mostly or as a starting point (e.g. getting a list of hit genes) on existing data. The re-use of biological data enables research to be sustainable by optimally exploiting existing experimental data. It also eases method comparisons by the existence of easy-to-access common datasets. Finally, as pointed out by~\cite{pmid24904347}, it is well in harmony with the current trend of general transparency in science. To continue with the Mitocheck example, two articles have been re-using it since the original paper ~\cite{pmid20360735} was published:~\cite{pmid25255318} searches for evidence of protein-protein interactions in siRNA experiment similarity, whereas~\cite{pmid24131777} dynamically models nuclear phenotypes.

\paragraph*{}
However big the advantages of data re-use, its requirements are not light. First of all, data and metadata should be of a minimal quality, for it to be understandable by other laboratories and re-analyzable. Laboratories should also share vocabularies, as well as formats, for data to be easily accessed and understood by all parties. Data re-use should be born in mind while acquiring it~\cite{pmid23047157}, since post-experimental data re-organization is harder and likely to never happen. Hence data re-use should never be a project afterthought. Finally, data should be persistent, which demands time, fundings, and incentive. Indeed, if web servers are used for data storage and access, their maintenance will need time and fundings. If individuals are contacted for information about data access, their responding will demand time and incentive~\cite{pmid18636105}.

\paragraph*{} The current state of data re-use in Bioinformatics depends much on the subfield, for maturity and culture reasons~\cite{pmid24904347}. Indeed, rather mature subfields have been sharing and re-using their experimental data for some time, whereas others only experiment isolated initiatives yet. If we consider the example of genomics,~\cite{pmid24109559} finds that this field sees quite a high level of third-party dataset re-use, since they expect 40 (resp. more than 150) PubMed papers re-using data by year 2 (resp. year 5) if 100 datasets were deposited at time 0\footnote{However, this meta-study suffers from some limitations, one of which is to not define data re-use.}. This is permitted by the existence of well-known servers where to deposit and search for functional genomics data, such as the Gene Expression Omnibus,  GEO\footnote{\href{http://www.ncbi.nlm.nih.gov/geo/}{http://www.ncbi.nlm.nih.gov/geo/}}. On the other hand, if we consider the example of Neuroscience,~\cite{pmid24904347} precisely suggests to learn from data sharing in genomics, and more specifically the Human Genome Project, in the context of the BRAIN Initiative. %QSAR models,~\cite{pmid24910716} proposes an approach for the digital organization and archiving of QSAR model information (although there are already two existing). 
%Does not define data reuse (does it mean they produced new results from published data, or that they compared their results/tried their methods on the existing datasets?). High level of third-party dataset reuse (if data deposited on GEO) : "for 100 datasets deposited in year 0, we estimate that 40 papers in PubMed reused a dataset by year 2, 100 by year 4, and more than 150 by year 5." (but data reuse is simply measured by presence of dataset number in text body).

\paragraph*{}
If we now consider this thesis subfield, namely \textit{Bioimage informatics}, initiatives remain isolated. A list of existing bench test datasets can be found in ~\cite{pmid18603566}, section 5.1, which are different of the existence of an ecosystem for data sharing. %Initiatives are quite numerous when it comes to medical images, e.g.~\cite{pmid24309199} for radiographical images, or [CITE] [CITE].
Most of the time, experiments are conducted in view of one or a few publications within the same team ; it is still rare that people re-mine third-party published data. This is probably not due to format issues. Indeed, there are open-source formats and plugins to convert from proprietary formats to open-source ones (eg OME-TIFF format\footnote{\href{http://www.openmicroscopy.org}{http://www.openmicroscopy.org}} or Bio-Formats\footnote{\href{http://www.loci.wisc.edu/ome/formats.html}{http://www.loci.wisc.edu/ome/formats.html}}, and translation plugins available for, e.g., ImageJ~\cite{imagej}). Concerns about patient consent and robust data anonymization are neither likely to be an issue as far as cell lines, tissues and small organisms are imaged, although it should be considered when sharing data about, e.g., primary tumors.

I see three main reasons for the little re-use which is currently experienced in Bioimage informatics. First of all, there is the need for shared controlled vocabularies, as also noted by~\cite{pmid18603566}. The network of excellence \textit{Systems microscopy} launched the initiative to build one\footnote{\href{http://www.ebi.ac.uk/ontology-lookup/browse.do?ontName=CMPO}{http://www.ebi.ac.uk/ontology-lookup/browse.do?ontName=CMPO}}, hence it will hopefully change in the future.

The second reason comes from practical considerations. There is a need for \textit{one} main famous physical server where Bioimage informatics data would be uploaded and further stored and accessed. Still, some datasets will not be downloadable (as an example, the Mitocheck dataset amounts to approximately 12 Tb). This prompts the need for further technical thinking (a common computing cloud?), which in its turn prompts the need for some dedicated collaborative funding. This is currently not easily done, as fundings are traditionally publication-based.% Nevertheless, more and more grant application calls insist that researchers commit to

But this will not be a problem, as long as people have a strong incentive to do it. And indeed the third most important reason is the lack of recognition from scientific publishers for papers which are doing re-mining only (as is apparently also the case in Neuroscience~\cite{pmid24904347}). If one sees little prospect of publishing in a good journal his/her data re-mining study, since it also demands finding, downloading and understanding somebody else's experiments, she/he will probably choose to produce new data.
%\begin{itemize}
%\item this is not 
%\item but more probably to the need for controlled vocabularies (\cite{pmid18603566} agrees, and an example of such an initiative: http://www.ebi.ac.uk/ontology-lookup/browse.do?ontName=CMPO in \textit{Systems Microscopy}, ie it's developing)
%\item the need also for a place where to put this data (exists various websites but none is really better recognized than the others. Also with what money for servers and maintenance?)
%\item AND most of all, to the lack of recognition from scientific publishers for papers which are doing data remining only.
%\end{itemize}  

\paragraph*{}Because we think that there is still unexploited information in the Mitocheck dataset, and more generally that data re-use should be encouraged in Bioimage Informatics, we chose to apply MotIW to this data. This led to the discovery of genes which are likely to be involved in cell motility, as well as an ontology of cell trajectories (section~\ref{sec:mitocheck}). Furthermore, combining tracking with nuclear phenotypic classification enables to detect genes which have an impact on cell cycle length, as described in section~\ref{sec:cellcycle}.

\section{MotIW's direct application to the Mitocheck dataset}
\label{sec:mitocheck}
 Analysis of the screening data through our workflow enables the labelling of a list of genes as likely to be involved in cell motility (section~\ref{sec:hitlist}). Furthermore, it reveals the existence of a cell trajectory ontology in the dataset. Without any prior assumption on cell motion, we are able to identify eight types
of cell trajectories (section~\ref{sec:ontology}).

% As we removed figure 4 ... 
Furthermore, studying how cell tracks distribute in the different motility patterns according to whether they are exposed agrees with the following paradigm: rather than creating new
motility behaviours, chemical exposure perturbs how cells distribute in the different possibilities, which also exist in the basal state.
\subsection{Hit list}
\label{sec:hitlist}
After evaluating MotIW on simulated data, we then apply it to the whole genome-wide screen Mitocheck~\cite{pmid20360735}. In the context of the Mitocheck dataset, the identification of an experiment in which cell motility is significantly different from negative controls leads to the identification of siRNAs which significantly and reproducibly alter cell motility. A gene was selected as possibly involved in motility mechanisms if it was targeted by at least one hit siRNA. Indeed, it is well known that strictly less than 100\% siRNAs which are targeted at a particular gene will effectively down-regulate it. The reasons which could explain this are still not completely understood. Therefore, requiring that more than one siRNA related to a given gene are selected for the gene to be selected would have led to too many false negatives.% Indeed, Mitocheck siRNAs supplier guaranteed that one out of 2 would actually work.

The application of MotIW to the Mitocheck dataset enabled the identification of the experiments which significantly deviate from controls ($5\%$; $7,153$ out of $144,909$). It amounts to $1,180$ genes (out of $17,816$), which are available as a supplementary to this thesis (see Supp. table 1).%http://cbio.ensmp.fr/~aschoenauer/qvalues_motility_Mitocheck.csv

\paragraph{Intersection with other published motility gene lists}
Some of these genes are well known to be involved in cellular motility, such as RhoA (Ras homolog family, member A) or CDK5 (cyclin-dependent kinase 5). However, the list is not overall significantly enriched in genes which are linked to cell motility according to the Gene Ontology database. This is also the case for all recent screens, regardless of their throughput.~\cite{pmid19160483} describes a study of 1,081 genes regarding cell motility in human breast cells (MCF-10A cell line). Using wound healing, they identify 66 high confidence genes of which only 24 were previously associated with cell motility. Similarly,~\cite{pmid21423205} is focused on the involvement of kinases (779 genes) in cell motility of human lung cancer cells (A549 cell line). Using single cell tracking, they identify 70 hit genes, of which only 13 were previously linked to cell motility. Finally,~\cite{pmid25774502} study 1,429 genes using phago-kinetic tracks in human lung cancer cells (H1299 cell line), finding 136 hits. Thanks to a personal communication of the authors, we could access their hit list and see that only 13 of their hit genes are functionally linked to cell motility (Gene Ontology biological process GO:0048870). 

Hence it seems that medium- to HT approaches to cell motility study tend to complement the older ones, rather than abide by them. In fact, this seems to be a more general trend of genetic screen hit lists. For example, Mitocheck mitotic hit list contains more than 50\% genes which were not functionally linked to cell division before~\cite{pmid20360735};~\cite{pmid15547975}'s cytokinesis-linked gene list contains only 20\% genes which were previously known to be involved in this cellular process. Modern systematic and automatic approaches to gene functional inference are indeed likely to detect genes whose involvement in the cellular process at hand was too subtle to be detected by lower-throughput and more ancient methods.

A question which arises next is to know the extent of the intersection between gene lists from published medium- to high-throughput studies about cell motility. They are detailed in table~\ref{details_biblio}.

\begin{table}[!ht]
\centering
\caption{Existing medium- to high-throughput studies of cell motility}
\label{details_biblio}
\begin{tabular}{|l|l|l|l|l|}
\hline
Study & Assay & Cell line & Gene list & Hit gene list \\
\hline
\cite{pmid19160483} & Wound healing & MCF-10A & 1,081 & 66\\%Simpson
\cite{pmid21423205} & Single cell tracking & A549 & 779 & 70\\%Lara
\cite{pmid23593504} & Matrigel invasion chambers & U87 & 1,954 & 25\\%Yang
\cite{pmid23751374} & Cell area growth & HeLa & 710 & 81\\%Zhang
Us & Single cell tracking & HeLa & 17,816 & 1,180\\%Us
\cite{pmid25774502} & Phago-kinetic tracks & H1299 & 1,429 & 136\\%Water
\hline
\end{tabular}
\end{table}
\begin{table}[!ht]
\centering
\caption{Hit list intersections}
\label{intersection}
\begin{tabular}{|l|l|l|l|l|l|l|}
\hline
& Simpson & Lara & Yang & Zhang & Us & van Roosmalen \\
\hline
\cite{pmid19160483}&  66& 4 &0&19&10&4\\%Simpson

\cite{pmid21423205} &4&70 &0&4&4&2 \\%Lara

\cite{pmid23593504} &0&0 &25&0&4&0\\%Yang

\cite{pmid23751374} & 19 &4&0&81&13&5\\%Zhang
Us & 10 & 4 & 4 & 13 &1,180&6 \\%Us
\cite{pmid25774502} & 4 & 2 & 0 & 5 & 6 & 136\\%van Roosmalen
\hline
\end{tabular}

\end{table}
%HC Simpson: 10 not in Mitocheck study, 10 in common out of 66. Both HC and MC: 19 not in Mitocheck study, 16 in common (out of 142). Have 19 MAR genes among the HC.
%We have 130 genes from Simpson study (138 siRNAs) of which 34 MAR genes (314 MAR genes). Hence that's approximately the same "concentration" of MAR genes although not designed for this (26\% for us, 29\% for them).
%
%Lara: 70 hit genes of which 13 not in Mitocheck. 4 genes in common with us, also 4 genes in common with Simpson HC. No MAR genes.

Following the information which is presented in table~\ref{intersection}, there is little intersection between the hit genes which were obtained by the existing medium- to high-throughput studies of cell motility. This can be explained by the fact that all studies use distinct assays and/or cell lines. On top of the experimental differences which could partly explain such little overlap, one could think about the following biological explanations. First of all, it is likely that (at least partly) different cellular processes are at play in collective and single cell migration. Second, it could be the case that each cellular model (such as normal breast epithelial cells, lung cancer epithelial cells, cervix cancer epithelial cells) is depending on specific and partly distinct machineries for cell motility. All these reasons would explain that all studies find such "stars" of cell motility as RhoA, but have little intersection otherwise. As for hit list enrichment in previously known genes, little overlap is generally observed between medium- to HT screen outputs~\cite{pmid20360735}.

\paragraph{Detailed functional analysis}
\textcolor{blue}{Wait for Piel's lists and then at least do two paragraphs here to discuss the results using Gene Ontology and his lists with this kind of statement}:  As an exemple, cells are observed to be motile when they divide as they detach from the plate surface, but it has nothing to do with directed motion. Hence it is normal that we find more than just cell migration. Think about~\cite{pmid18213366} "these data indicate that changes in cell migratory behavior caused by the different can-
didate migration genes are largely dominated by changes in cell matrix adhesion dynamics."

ZRANB1: found to produce stress fibers in \cite{pmid21834987} and inhibit wound closure in PC3 cells. If I look at intersections with other screens, I can also cite them as saying their genes were not found by Collins et al 2006.

\subsection{Cell trajectory ontology}
\label{sec:ontology}
A question which is related to motility gene discovery is to know whether there exists an ontology of cell trajectories. The approach to answer it would be to apply unsupervised clustering methods on the whole trajectory dataset and try to identify a number of motility patterns for which the clustering is of good quality. This is measured by cluster quality indices, which depend on the clustering method (see e.g.~\cite[Chapter~8]{Tan:2005:IDM:1095618}, or~\cite{Halkidi}). As an example, two common indices to evaluate the output of k-means are the intra-cluster cohesion $C(K)$ and the silhouette score $S(K)$. They both compare intra-cluster distances to inter-cluster distances, if $N$ is the number of data points centered in $c_{data}$:

\[
C(K) = \sum_{\substack{k}} \sum_{\substack{x \in c_k}} \dfrac{d(x, c_k)^2}{d(x, c_{data})^2}\]

\[
S(K) = \dfrac{1}{N} \sum_{\substack{x}} \dfrac{b_x-a_x}{\max (a_x, b_x)}\]
where \[ a_x = mean \{d(x,y) | y \in c_{k_x} \},\ \ b_x = \min_{\substack{k\neq k_x}} mean \{d(x,y) | y \in c_k \}
\]

A slope change in $C(k)$ and a maximum in $S(k)$ are expected at the appropriate number of clusters, if it exists. Other interesting measures of clustering quality encompasses for example the Bayesian Information Criterion or clustering stability~\cite{pmid11928511}.

This approach did not prove to be successful when applied to pooled trajectories from all experiments, for a wide range of clustering techniques (k-means, Gaussian mixtures models, spectral clustering, fuzzy c-means, kernel k-means with a radial basis function - data not shown). Concretely, no combination of clustering algorithm and cluster number could be found, whose quality was clearly over the quality obtained by the same clustering algorithm and other cluster numbers. It seemed as if no structure could be found in the data.

Nevertheless, the clustering approach succeeded when only trajectories from the \textbf{detected} experiments were pooled together. Indeed, this small subset contains only experiments which have been selected for being significantly different of controls in terms of single cell motility: it is enriched in rare trajectories. Our interpretation for this result is the following. Let us assume that we have a number of $k$ clusters in the whole trajectory dataset. Due to biological variability, each trajectory is at a certain random distance (in the feature space) to its latent trajectory cluster center. Given the dataset size (approximately 50 million trajectories), this produces a continuous dataset in the feature space, preventing the identification of any cluster. Furthermore, the different clusters are unbalanced. It is therefore not sufficient to uniformly subsample the whole dataset to detect the clustering structure (data not shown). Applying MotIW to find experiments which are enriched in rarer trajectories functioned as performing a stratified subsampling with respect to the underlying cluster structure of the dataset. It made it possible to identify the latter.

 After retaining the first seven principal components (explaining $95\%$ of the variance), k-means was applied to the resulting dataset of approximately 1.1 million trajectories. Fig.~\ref{clusterscore} shows the evolution of intra-cluster cohesion and silhouette score with respect to the number of clusters. It points to $k=8$ as being both the best and a good quality clustering on this dataset. Indeed, a break and a maximum are respectively expected in the cluster cohesion and the silhouette score curves at the correct cluster number, if it exists. It is compared to the evolution of those indices, if k-means is applied to a uniformly random dataset and a normal dataset (in $\mathbf{R}^8$). 

\begin{figure*}[!ht]
\centerline{\includegraphics[scale=0.3]{figures/Walter_295_fig_6.png}}
\caption{Evaluation of k-means clustering quality as a function of the number of clusters (average and standard deviation on 10 algorithm initializations). The same protocol was applied to a subset of the Mitocheck dataset, and two samples of the same dimensions, respectively drawn from the Uniform and the Normal distributions. }
\label{clusterscore}
\end{figure*}

A first insight on cell motility from this clustering is presented figure~\ref{distribution}. It presents the distribution of trajectory distributions in the eight identified clusters. One can observe that there are no cluster which is specific either to controls, either to hit experiments. This answers the question to know whether gene silencing creates new motility behaviours, or rather modify the probabilities that a cell belongs to a certain trajectory cluster. This figure shows that it is the latter which happens: siRNA exposure modifies how likely it is that a cell will belong to a certain cluster, but does not create new behaviours. This could be explained by the fact that a certain number of motility behaviours are possible for the cell to adopt at any time. Which one it chooses is function of its cellular and molecular state, as well as the stochasticity of gene expression. Hence some behaviours will be rarer in control videos, but exist nevertheless, which will become more frequent under a modification of the cell molecular state due to siRNA exposure.

\begin{figure*}[!ht]
\centering
\includegraphics[scale=0.25]{figures/traj_distribution_clusters_EXP_CTRL.png}
\caption{Comparison of cluster distributions between controls (Ctrl) and experiments (Exp) for the eight trajectory clusters which were identified in the Mitocheck dataset. The clusters are in the same order as in figure ~\ref{heatmap}.}
\label{distribution}
\end{figure*}
 
The cluster characteristics are detailed fig.~\ref{heatmap}. Each column in the heatmap corresponds to one cell trajectory, for which the rows show the standard scores of a subset of features. 

A result about single cell motility patterns is obtained from experiments which were selected on the basis of their trajectory feature distributions. This shows that meaningful single cell information can be retrieved by our statistical procedure, which works at the experiment level.

In the second place, it shows that there is more than speed for differentiating trajectory types. For example, clusters 2 and 3 present very similar MSDs and \textit{Effective space length}. However, trajectory curvatures are different: the features \textit{Mean curvature} and \textit{Straigthness index} are quite distinct between the two clusters. This can be observed in the Supplementary movie, where cells whose trajectory belongs to cluster 2 (green) are much straighter than those belonging to cluster 3 (red). In this video, cells whose trajectory passed the trajectory quality control have a dot, whose colour corresponds to its cluster as indicated in fig.~\ref{heatmap}.

\begin{figure*}[t!]
\centerline{\includegraphics[scale=0.3]{figures/Walter_295_fig_7.png}}
\caption{Characterization of our ontology of trajectories. Each column is a single cell trajectory ; trajectories are grouped by cluster label. 1,000 trajectories were randomly selected per trajectory cluster.}
\label{heatmap}
\end{figure*}
\subsection{Discussion}
discussion\clearpage
\section{Cell cycle length study}
\label{sec:cellcycle}
%\textit{Eventually look at intersection with~\cite{pmid16564017},~\cite{pmid17001007}}
\paragraph*{} The combination of the Mitocheck dataset and of our methodological workflow is also very well suited to study cell cycle genes. Indeed, one only needs to combine tracking and nucleus classification to recover a set of complete trajectories in each experiment, that is, trajectories which start with a mitosis and end with a mitosis. Using the methodological procedure described in~\ref{sec:stats}, it is then possible to detect the experiments in which cell cycle length distribution is significantly different than that in control experiments, resulting in a list of genes.
\begin{figure*}[ht!]
\centerline{\includegraphics[scale=0.6]{figures/cell_cycle_examples.png}}
\caption{Examples of object divisions from the Mitocheck dataset}
\label{cellcycle2}
\end{figure*}
\subsection{Complete cell cycle detection}
%\paragraph{Complete trajectory identification}
The first step to study cell cycle length distribution according to siRNA exposure is to filter out complete trajectories from the others. By complete, we mean trajectory which start with a mitosis rather than the end of a \textit{merge} event, an apparition, the split of an apoptotic debris or the beginning of the film, and which end with a mitosis rather than a \textit{merge}, a disappearance or the end of the film. The distinctions are easily made by a human eye, as can be seen on figure~\ref{cellcycle2}. To automatize this filtering, one can rely on the nucleus classification as described in section~\ref{Mito_classif} and~\cite{pmid20360735}: prior to a mitosis, the nucleus will likely be observed in $M_{-1}=\{$\textit{prometaphase}, \textit{metaphase}, \textit{metaphase alignement problem}$\}$. Similarly, following a mitosis, the nucleus will likely be observed in $M_{+1}=\{$\textit{anaphase}$\}$. 

However, given that the classifier is not 100\% correct, it is not fully certain. Furthermore, there could be cases of accelerated mitoses, in which it would not be possible to observe both the mother cell in $M_{-1}$ and the daughter cell in $M_{+1}$. Hence we developped a scoring approach with respect to each track of interest $\tau$ going from $T_0$ to $T_f$, as detailed in figure~\ref{cellcycle1}:
 \[score_{1, \tau} = \mathbf{1} (Mother_\tau \in M_{-1}) + \mathbf{1} (\tau_{T_0} \in M_{+1}) \]
 \[score_{2, \tau} = \mathbf{1} (\tau_{T_f} \in M_{-1}) + \sum_{\substack{\tau 's~ children}}\mathbf{1} (c \in M_{+1}) \]
 
Briefly, the scores compute the number of \textit{right} classifications in the starting and ending splits. To evaluate where to threshold those scores in order to filter out the unwanted tracks, 20 movies were randomly sampled from the Mitocheck dataset, all splits scored and all 2,100 tracks manually divided into complete and other trajectories. Selecting tracks longer than 1 frame with $score_1\geq 1$ and $score_2\geq 1$ enables to select more than 87\% complete trajectories.

\begin{figure*}[ht!]
\centerline{\includegraphics[scale=0.3]{figures/cell_cycle_study.png}}
\caption{Approach to cell cycle study}
\label{cellcycle1}
\end{figure*}

\subsection{Cell cycle length hit list}
The principle of MotIW's statistical procedure~(see~\ref{sec:stats} and~\cite{motiw}) was then applied to the distributions of complete trajectory length: 2 sample Kolmogorov-Smirnov tests were realized between cell cycle length distributions of each experiment and the controls of the same batch. It was compared to the empirical null distribution in a second time. The latter is the distribution of 2 sample Kolmogorov-Smirnov tests comparing controls to controls. Finally, the empirical p-values which were obtained were adjusted for multiple testing, following the Benjamini-Hochberg procedure~\cite{Benjamini1}.

Setting a threshold of $0.05$ enabled the identification of 66 genes whose cell cycle length distribution differs from that of controls from the same batch. The list is fully provided in appendix, see section~\ref{cellcyclelist}. Interestingly, the down-regulation of only three genes produces a \textit{longer} cell cycle length: APOA1, RPS20, SFMBT2 (coding respectively for the apolipoprotein A1, ribosomal protein S20 and Scm-like with four mbt domains 2). SFMBT2 silencing has already been found to decrease cell growth in human prostate cancer cells~\cite{pmid23385818}. 

Gene Ontology analysis reveals that there are three annotation clusters which are highly enriched in this list. The first one is related to protein kinase activity. It contains such genes as BMPR-IB which encodes the bone morphogenetic protein receptor, type IB, and whose reduced expression is correlated to poor prognosis in breast cancer~\cite{pmid19451094}, and tumor grade in prostate cancer~\cite{pmid10850425}. The second cluster is related to nucleotide binding. It contains genes such as the integrin-linked kinase whose silencing has interestingly been found to slow cell cycle in human gastric carcinoma cells~\cite{pmid23748822}, whereas we have found its silencing to speed cell cycle in HeLa cells. Finally, the third cluster contains genes which encode proteins which are intrinsic to plasma membrane, such as the melatonin receptor 1A whose absence has been found to be correlated with bad prognosis in triple-negative breast cancer~\cite{pmid23250547}.  

\subsection{Discussion}
Our approach for studying cell cycle length enabled us to obtain a list of 66 genes which may be involved in cell cycle regulation. Gene Ontology analysis revealed that a certain number of these genes has already been found to be linked to cell cycle duration regulation. This list contains 3 genes whose silencing lengthens cell cycle, and 63 whose silencing shortens it. It is possible that this more broadly reflects that there are more proteins which play a role of checkpoints rather than cell cycle enhancers, hence the fact that gene silencing experiments produce more experiments where cell proliferation is increased than decreased. This is supported as well by other studies such as~\cite{pmid16564017}, which found 87 (resp. 15) genes whose silencing significantly increases (resp. decreases) the mitotic index of HT29 cells.

\paragraph{Method bias}
Our method was applied to the whole Mitocheck dataset, enabling the obtention of complete trajectory length distributions for all experiments. Experiments with less than 10 complete trajectories were not considered for further analysis. This explains why decreased proliferation genes as provided by~\cite{pmid20360735} could not be found: the proliferation is so low that no mitosis is observed, hence no complete trajectories can be found. Trying to diminish this bias in experiment selection, we also included trajectories which finish with the end of the experiment rather than a mitosis (hereafter called \textit{incomplete} trajectories). 

However, this approach was not successful. For some siRNAs, it seems that incomplete trajectory length distribution indeed has the same shift as that of \textit{complete} ones (see the example of arylsulfatase F gene, ARSF, on fig.~\ref{ARSF}). However, for most of the siRNAs, incomplete trajectory length distributions seem to be more dependent on the batch than on the chemical exposure. Two examples are shown fig.~\ref{CAC} which concern the genes CACNA1D and DIMT1 (respectively coding for the calcium channel, voltage-dependent, L Type, Alpha 1D subunit and DIM1 dimethyladenosine transferase 1 homolog). The outcome of the statistical analysis consistently proved too noisy to enable the detection of any hit siRNA for incomplete trajectory length (up to the following threshold for adjusted p-values: $0.1$).

\begin{figure*}[ht!]
\centerline{\includegraphics[scale=0.3]{figures/ARSF_length_distribution_cens.png}}
\caption{Histograms showing cell cycle length for complete (top) and incomplete (bottom) trajectories, for two experiments of the Mitocheck dataset concerning ARSF which were detected as significantly different from controls for cell cycle length.}
\label{ARSF}
\end{figure*}
\begin{figure*}[ht!]
\centerline{\includegraphics[scale=0.3]{figures/CACNA1D_length_distribution_cens.png}
\includegraphics[scale=0.3]{figures/DIMT1_length_distribution_cens.png}
}
\caption{Histograms showing cell cycle length for complete (top) and incomplete (bottom) trajectories, for two experiments of the Mitocheck dataset concerning CACNA1D (left) and DIMT1 (right), which were detected as significantly different from controls for cell cycle length.}
\label{CAC}
\end{figure*}

\paragraph{Perspective: cell cycle phase detection}
The cell cycle can furthermore be split in four sequential phases: $G_1$, $S$, $G_2$, and $M$. $S$ is the DNA replication phase and $M$ stands for mitosis. To distinguish $S$ from $G_1$ and $G_2$, PCNA (proliferating cell nuclear antigen) is usually used. It forms sparkling nuclear dots during DNA replication, as it is recruited to replication foci~\cite{pmid10769021}.

However, the only marker available in the Mitocheck dataset is the core histone 2B (H2B). The question is then to know whether it is possible to use this marker for $S$ detection.~\cite{pmid16765323}, using cell size and H2B-GFP fluorescence, managed to FACS-purify RKO cells in $G_1$, $S$ and $G_2$. Furthermore,~\cite{pmid17401369} uses a "cell cycle heuristic" to separate between the different phases of the cell cycle using DNA size and intensity as provided by Hoechst 33342 as a DNA marker (Supplementary figure 3,b-c). They use it to infer links between small molecule exposure and cell cycle modification; however they do not prove the accuracy of their heuristic.

This led us to the hypothesis that it might be possible to use the information contained in the H2B-GFP fluorescence signal for \textit{in silico} sorting of HeLa cells. However, nuclear size and intensity as provided by H2B-GFP do not seem to make $S$ identification possible. Indeed, as shown on fig.~\ref{intensity_roisize}, they grow linearly during cell cycle without showing any clear slope break. This is confirmed on fig.~\ref{intensity_roisize2}. This latter plot has the same configuration as that of Supp. fig. 3 from~\cite{pmid17401369}. However, although this study uses a heuristic model that defines a clear separation between nuclei in different cell cycle phases, we do not find it in our data. This might be due to slight differences between the information from DNA intercalating agent fluorescence, such as Hoechst 33342, and the information from histone 2B fluorescence. Nevertheless, nuclear intensity and size are clearly linked to cell cycle phases.

Hence to answer this question, we had the idea to use a published dataset of HeLa cells which were stained for both H2B and PCNA~\cite{cellcognition}. Cell cycle phase annotations of this dataset are available on the Cell Cognition website\footnote{\href{http://www.cellcognition.org/downloads/data}{http://www.cellcognition.org/downloads/data}}. This training set was created using the information on the PCNA channel. It makes it possible to test the following hypothesis: can a classifier be trained, which uses H2B information for cell cycle phase identification? Preliminary work shows that all tested methods have an accuracy below 70\% (random forest, gradient boosting, support vector machine, logistic regression). However, using both this information and cell tracking, one can hope that the use of Markov chains for correcting classification errors could lead to good results.
\begin{figure*}[ht!]
\centerline{\includegraphics[scale=0.6]{figures/roisize-LT0013_01--00315.png}
}
\caption{Example of the time evolutions of nuclear size ("roisize", top left and bottom) and nuclear intensity ("total intensity", top right) for all complete trajectories of a control experiment from the Mitocheck dataset. As discussed in the text, no clear slope break is seen for most trajectories for any of the two indicators, hence preventing the delimitation of cell cycle phases using only this information.}
\label{intensity_roisize}
\end{figure*}
\begin{figure*}[ht!]
\centerline{\includegraphics[scale=0.3]{figures/fig_intensity_DNAsize_cyclephases2.png}
}
\caption{DNA intensity and size as provided by H2B-GFP information is not sufficient to differenciate between the different cell cycle phases. Data and labelling come from the PCNA dataset.}
\label{intensity_roisize2}
\end{figure*}
\clearpage
\section{Functional inference by in silico comparison of small-molecule and siRNA screens}
~\footnote{Manuscript in preparation}
Maybe briefly describe the drugs?\\
- some biblio ~\cite{pmid20418956}

type of approach as opposed to targeted bioassay which demand specific prior knowledge allows in principle for systematic and direct identification of drug target (as long as it is in the gene list)

\subsection{Materials and methods}
\subsubsection{Experimental work}
This data set was not produced in the context of this PhD. Rather, the experiments were conducted at the EMBL (Heidelberg, Germany) by **. For the sake of completeness, here is briefly how it was produced. [MISSING]
\subsubsection{Object segmentation}
\label{sec:ds_seg}
Since we are interested in nuclear morphologies in this context, the original segmentation of the Mitocheck project was used, as previously described~\cite{Walter2010}.

\subsubsection{Object classification and phenotypic scores}
\label{sec:ps}
The use of the original segmentation from the Mitocheck project made it possible to re-use its training set, albeit strengthened for classes which were previously slightly under-represented. This can be seen on fig.~\ref{ds_classifier} in comparison with figure 3 from~\cite{Walter2010}. A visual inspection of the dataset enabled us to verify that the drug screen experiments do not contain any nuclear morphological phenotype which were not observed in the Mitocheck experiments. This would have made it necessary to include nuclei from drug screen experiments into our training set.

Cell Cognition~\cite{cellcognition} was used for learning an RBF (Radial Base Function) kernel SVM classifier, whose precision and recall are also indicated on fig.~\ref{ds_classifier}. Its parameters were optimised by grid-search ($\gamma=2^{-7}$, $C=8$).

\begin{figure*}[ht!]
\centerline{\includegraphics[scale=0.6]{figures/ds_classifier.png}}
\caption{Precision and recall per class as provided by Cell Cognition. Compared with the original classifier as published in~\cite{Walter2010}, classes \textit{ADCCM} (Asymmetric Distribution of Condensed Chromosome Masses) and \textit{Out of focus} were added. More nuclei were furthermore included for training in most classes. \textit{Shape1} (resp. \textit{Shape3}, \textit{MetaphaseAlignement}) corresponds to binucleated (resp. polylobed, metaphase alignement problem) nuclei.}
\label{ds_classifier}
\end{figure*}

This provides a representation of each video as a set of time-series, which are the evolution of the percentage of nuclei in each phenotypic class over time. To evaluate how an experiment $i$ diverges from  control experiments from the same batch $C_i$ for its temporal evolution of class $k$, we used phenotypic scores as previously described~\cite{Walter2010}. Briefly, temporal evolutions of the percentages of nuclei in class $k$ for experiment $i$, $(\%_{k,i,t})_t$ and its controls $(\%_{k,C_i,t})_t$ are regularized using a locally weighted scatterplot smoothing as implemented in the Python package statsmodels. The fraction of data points which is used for smoothing was manually chosen to be $f=50\%$. The maximum deviation between the two regularized time series is then computed, where $ps$ stands for "phenotypic score":
\[
ps_{k,i} = \max_{0\ldots T} (\%^{reg}_{k,i,t} - \%^{reg}_{k,C_i,t})
\]

\subsubsection{Quality control}
One plate had to be eliminated due to an issue during image acquisition. An abrupt increase in fluorescence intensity around the 80\up{th} frame of all experiments from this plate causes a sudden increase in the number of detected object which prevent any time-consistent analysis.

For the other experiments, a threshold of $c$ cells at the beginning of the movie, and maximum $p$\% out-of-focus objects were used to remove unexploitable movies. $c=50$ and $p=40$ were selected. Out-of-focus objects and cells that were neither artefacts nor out-of-focus objects were identified following segmentation, feature extraction and classification as described supra.

Out of $1,232$ experiments on four plates, $904$ experiments from three plates passed the quality control, among which $98$ control experiments.

\subsubsection{Selection of Mitocheck experiments for target inference}
\label{sec:selection_mitocheck}
The Mitocheck project led to the identification of 1,249 mitotic hits in the primary screen. 1,042 were identified by manual thresholds on phenotypic scores from the following phenotypic classes: \textit{Prometaphase},\textit{Metaphase Alignement Problem}, \textit{Binucleated}, \textit{Polylobed}, \textit{Grape}. 207 genes were further identified by manual annotations. 1,128 of these genes were screened again in a validation screen.

Lists of hit genes according to the following measures were also published:
\begin{itemize}
\item phenotypic score of \textit{Large} nuclei,
\item cell death, as measured by phenotypic score of \textit{Apoptosis} nuclei,
\item nuclear dynamic changes, as measured by the sum of phenotypic scores for \textit{Hole}, \textit{Folded} and \textit{Small irregular},
\item and cell proliferation.
\end{itemize}

Finally, a list of hit experiments for \textit{Elongated} nuclei was computed, which can be found as a supplementary to this thesis (see Supp. table 2).

Given the evolution of the reference sequence of the genome, not all those genes were in fact targeted in the Mitocheck experiments. An updated mapping of the siRNAs which were used in the primary and validation Mitocheck screens to the present reference sequence of the genome was graciously provided by Jean-Karim Hériché (EMBL, Heidelberg, Germany). Once this and the quality control are taken into account, the final list of hits in at least one of the listed categories amounts to 2,614 genes (cf fig.~\ref{hit_mito}), which are covered by 4,847 siRNAs.

\begin{figure*}[ht!]
\centerline{\includegraphics[scale=0.4]{figures/gene_hit_mitocheck.png}
}
\caption{Number of hit genes per category. As hit detection is univariate, a gene can be in more than one category.}
\label{hit_mito}
\end{figure*}

Given the variations in siRNA coverage between genes which were, for example, included or not in the validation screen, each gene was chosen to be represented by the siRNA which showed the maximum effect. This was measured by the median of the phenotypic scores for \textit{Interphase} nuclei of this siRNA experiments.
%{ 'increased proliferation': 94, 'Cell Death': 746, 'Dynamic changes': 716, 'Large': 305, 'Elongated': 127, })

%publi mitotic: 'Grape': 91,'Binuclear': 402, 'Polylobed': 344, 'Mitotic delay/arrest': 345

\subsubsection{Selection of drug screen hit experiments}
\label{sec:selection_ds}
Phenotypic scores of the drug screen experiments were computed as described supra. Experiments whose \textit{Interphase} phenotypic score was lower than $Q_1^{ctrl} - 1.5\times IQR^{ctrl}$ were selected as hit experiments, where $Q_1^{ctrl}$ and $IQR^{ctrl}$ are respectively the first quartile and the inter-quartile range of control \textit{Interphase} scores. This is a robust one-sided way to select outliers, as the distribution of control \textit{Interphase} scores cannot be assumed to be Gaussian. 

It corresponds to the $197$ experiments which are under the bottom whisker on the \textit{Interphase} subplot of fig.~\ref{hit_ds}. Supplementary plots represent the distribution of phenotypic scores as a function of dose and drug, see "Phenotypic score plots" in "Supplementaries" folder.

Hit conditions are conditions for which stricly more than 50\% of their replicates are hit experiments.

\begin{figure*}[ht!]
\centerline{\includegraphics[scale=0.35]{figures/phenotypic_scores2.png}
}
\caption{Distributions of phenotypic scores from the drug screen experiments. Each boxplot corresponds to the distribution of control phenotypic scores, whereas each red dot is an experiment in which cells were exposed to a drug.}
\label{hit_ds}
\end{figure*}

\subsubsection{Other analyses}
%Proliferation hit detection was realized similarly to hit experiment detection: 
%
Condition hierarchical clustering was performed using the median of condition replicates. The Python package fastcluster~\cite{fastcluster} was used to this end.


A Web-based user interface was designed and implemented for result visualization and result sharing among collaborators. It is described more in details infra, see section~\ref{interface}.

\subsection{Phenotypic profile distances}
Screening experiments provide us with temporal sequences of information. They can be seen as sequences of intensity two-dimensional distributions~\cite{pmid22743764}, sequences of object feature distributions or sequences of phenotypic class distributions. In our case, we chose to summarize each experiment by a set of temporal evolutions of phenotypic classes, that is, we chose to represent our experiments by their phenotypic profile. The question remains to know how to compute divergences between these representations of the information.

%\subsubsection{Investigated distances}
We decided to test the following distances on the question to know whether we can apply phenotypic profiling for drug target inference from parallel drug and siRNA screens:
\begin{itemize}
\item a very simple approach, the Euclidean distance of phenotypic scores, 
\item a state-of-the-art approach, the phenotypic trajectory distance as defined in~\cite{Walter2010},
\item a divergence which enables the use of biological prior knowledge, the Sinkhorn divergence~\cite{sinkhorn}.
\end{itemize}

Let $i$ be a screening experiment, and $p$ the number of phenotypic classes ($p=15$ in our case). 
\subsubsection{Euclidean distance on phenotypic scores}
$i$ can be represented in $\mathbb{R}^p$ by $(ps_{k,i})_k$ the vector of its phenotypic scores. The distance between two experiments is then the Euclidean distance of their vectors of phenotypic scores, excluding \textit{Interphase} and \textit{Anaphase} scores. \textit{Interphase} score is excluded as its decrease is most of the time a summary of the increases of other scores. \textit{Anaphase} score is excluded because [WHATs the reason again?]. 

These vectors can be normalized with respect to the mean and standard deviation of phenotypic scores in the dataset. This will correspond to the \textbf{Normalized phenotypic score} distance in the following, whereas the non-normalized version will simply be called \textbf{Phenotypic score} distance.

These distances are robust to time delay in the onset of phenotypic changes. Indeed, as controls basically show a constant percentage of \textit{Interphase} nuclei, phenotypic scores will be identical for two experiments which show an increase in, e.g., \textit{Apoptosis} nuclei respectively at the beginning and at the end of the experiments. Hence the strength of these distances is that even if one experiment is identical with a delay to another, their distance will be small. It will however still be small if they're distinctly ordered, e.g. if one experiments shows the same phenotypic events than the other, albeit in the opposite order.

\subsubsection{Phenotypic trajectory distance}
On the other hand, it is possible to use the phenotypic trajectory distance as published in~\cite{Walter2010}. Briefly, let us re-use the notations of section~\ref{sec:ps}: $i$ is seen as $(\%_{k,i,t})_{k=1\ldots p,t=1\ldots T}$, that is, a sequence in $[0~;1]^p$. This sequence is then approximated by two p-dimensional vectors. The phenotypic trajectory distance between two experiments is then a distance between their vectors, as defined in formula 7 of~\cite{Walter2010}. This distance will be called \textbf{phenotypic trajectory} distance in the following.

This distance does not take explicitly time into account, but it respects the order of phenotypic changes. Hence its strength is that even if one experiment is identical with a delay to another, their distance will be small. It will not if they're distinctly ordered, as opposed to the phenotypic score distances.

\subsubsection{Sinkhorn divergence}
\paragraph{Motivations}
Finally, we wanted to test a distance which would enable us to use some prior biological knowledge of phenotypic class relationships. If we consider the two previous distances, they implicitly consider each phenotypic class to be independent of the others, and equally biologically far away from all. Indeed, the phenotypic score distances operate in $\mathbb{R}^p$ to sum the squared differences of phenotypic scores, hence treating the different phenotypes independently of eachother. Nevertheless, \textit{binucleated}, \textit{polylobed} and \textit{grape} nuclei are for example three different outcomes of cytokinesis defects. Hence the biological intuition is that a chemical causing a great increase in \textit{polylobed} nuclei has probably a closer mode of action to that of another one causing an increase in \textit{binucleated} nuclei than it is to that of a third one causing a strict increase in apoptosis.

Ideally, the idea is then that the distance of $a\%$ \textit{binucleated} nuclei to $b\%$ \textit{polylobed} nuclei is smaller than that to $b\%$ \textit{apoptosis} nuclei, or that it "costs" less to go from $a\%$ \textit{binucleated} nuclei to $b\%$ \textit{polylobed} nuclei than to $b\%$ \textit{apoptosis}. This is precisely the idea behind the Earth Mover's distance (or transportation distance, or Wasserstein distance). This distance was developed in the first place to compute the cost to move a certain number of piles of dirt into a certain numer of holes. To do so, one needs to take into account the distance between piles and holes.
\paragraph{Definitions}
Let us formalize this intuition and briefly introduce transportation distance\footnote{References and proofs can be found in~\cite{sinkhorn}.}. We note $\Sigma_d = \{ x \in \mathbb{R}^d_+ | x^T \mathbf{1}_d = 1\}$ the probability simplex. In our case, $d=13$: we can consider either the distributions of phenotypes in a given experiment over all time-points, or this distribution in a specific frame.

Given $r$ and $c$ in $\Sigma_d$, the transport polytope $U(r,c)$ is the set of matrices such that 
\[
U(r,c)=\{ P\in \mathbb{R}^{d\times d}_+ | P\mathbf{1}_d = r, P^T \mathbf{1}_d = c \}\] 
If $X$ and $Y$ are two discrete random variables with values in $ \{1,\ldots , d\} $ whose distributions are $r$ and $c$, the elements of $U(r,c)$ are in fact the possible joint probabilities of $\left( X,Y\right)$. Given a cost matrix $M$ in $\mathbb{R}^{d\times d}$, the optimal transportation distance between $r$ and $c$ is the solution of the following optimization program, where $<\cdot, \cdot> $ is the Frobenius matrix norm:
\begin{equation}
d_M(r,c) = \min_{\substack{P\in U(r,c)}} <M,P>
\label{exact_emd}
\end{equation}

Optimal solutions $P^*$ of~\ref{exact_emd} can be obtained. Furthermore, if $M$ is a metric matrix, this quantity is a distance~\cite{Villani}. This optimization program's complexity is in $O\left( d^3\log d\right) $ in theory and in practice, which makes it less applicable to high-dimensionality problems. 

In our case however, $d=13$, hence complexity is not a serious issue. An issue which is more relevant is that optimal solutions $P^*$ will lie on the vertices of $U(r,c)$. This is due to the linear quality of the optimization problem. It will produce almost deterministic joint probabilities~\cite{sinkhorn}. The idea is therefore to solve a regularized version of this program, placing ourselves in the following convex subset of $U(r,c)$, for $\alpha >0$:
\[
U_\alpha\left( r, c\right) = \{ P \in U\left( r,c \right) | \mathbf{KL}\left(P || rc^T\right)\leqslant \alpha\}
\]

The Sinkhorn divergence will be the following quantity, for $\alpha >0$:
\begin{equation}
d_{M, \alpha}(r,c) = \min_{\substack{P\in U_\alpha (r,c)}} <M,P>
\label{emd}
\end{equation}

This will produce less deterministic optimal solutions, which will converge to $P^*$ as $\alpha$ increases, while $d_{M, \alpha}(r,c)$ converges to $d_{M}(r,c)$. In practice, there exists an efficient method for solving the dual of this problem, Sinkhorn fixed-point algorithm~\cite{sinkhorn_original}. 

The solution for obtaining a faster computation of an approximate transportation distance is therefore to solve the dual problem of formula~\ref{emd}. Its solution will be used to compute the Sinkhorn divergence. For any $\alpha >0$, there exists $\lambda >0$ such that $d_{M, \alpha}(r,c) = d_{M, \lambda}(r,c)$, with
\[
d_{M, \lambda}(r,c) = <M,P^\lambda>,~~ P^\lambda = \argmin_{\substack{P\in U (r,c)}} <M,P> - \dfrac{1}{\lambda} h\left( P\right)
\]

The use of this divergence will enable us to take into account prior biological knowledge while computing distances between phenotypic distributions. This knowledge will be encoded in the cost matrix M. Its choice as well as $\lambda$'s is described infra. Practically, Sinkhorn fixed-point algorithm was implemented in Python to compute Sinkhorn divergences.

Finally, there are two ways to apply Sinkhorn divergence to the problem at hand with the parameters $M,~ \lambda$ as defined above. Let us consider two experiments $i$ and $j$ of duration $T$: 

\begin{itemize}
\item one can pool all nuclei from all frames by representing the experiments in $\Sigma_d$. This distance will be called \textbf{global Sinkhorn divergence} in the following (formula~\ref{global}),
\item one can choose to keep the temporal information by representing the experiments in $\left(\Sigma_d\right) ^T$. We will define two distances from this:
\subitem \textbf{max time Sinkhorn divergence}, which is the maximum of all timepoints Sinkhorn divergences (as detailed formula~\ref{max}),
\subitem \textbf{sum of time Sinkhorn divergence}, which is the sum of all timepoints Sinkhorn divergences (as detailed formula~\ref{int}).
\end{itemize}
\begin{equation}
D_{M,\lambda}(i,j)= d_{M,\lambda}((\%_{k,i})_{k},(\%_{k,j})_{k})
\label{global}
\end{equation}

\begin{equation}
D_{M,\lambda}^{max}(i,j)= \max_t d_{M,\lambda}((\%_{k,i,t})_{k},(\%_{k,j,t})_{k})
\label{max}
\end{equation}

\begin{equation}
D_{M,\lambda}^{sum}(i,j)= \sum_t d_{M,\lambda}((\%_{k,i,t})_{k},(\%_{k,j,t})_{k})
\label{int}
\end{equation}

\paragraph{Choice of phenotypic cost matrix}
The phenotypic cost matrix summarizes our biological knowledge about the phenotypes which were observed in the Mitocheck dataset (they include those which were observed in the drug screen). We made the choice to set inter-phenotypic costs according to the cellular process which is perturbed when they appear, or which they represent. The phenotypic cost matrix which we chose is illustrated fig.~\ref{cost}.

To resume our previous example, \textit{Polylobed}, \textit{Grape} and \textit{Binucleated} nuclei were considered to be closer to eachother than to any other phenotype, as they all result from cytokinesis defect. Nevertheless, \textit{Binucleated} nuclei were set slightly further apart from \textit{Polylobed} and \textit{Grape} nuclei than those two from eachother. This is because [EXPLANATION AGAIN ??].

Let us consider another example and take the group of \textit{Interphase}, \textit{Elongated} and \textit{Large} nuclei. They were set closer to eachother than to any other phenotype because they represent normal and abnormal versions of interphase nuclei.

\begin{figure*}[ht!]
\centerline{\includegraphics[scale=0.4]{figures/transportation_cost.png}
}
\caption{Cost matrix for phenotypic Sinkhorn divergence}
\label{cost}
\end{figure*}

\paragraph{Choice of $\lambda$} This choice will determine how close the Sinkhorn divergence is to the transportation distance. As expected, when $\lambda$ increases, the Sinkhorn divergence converges. This is illustrated fig.~\ref{lambda_choice1}. This figure also enables us to see that in the range of $\lambda$ which we investigated, there seems to be mainly two different behaviours: one which is shown at $\lambda=0.01$ and $\lambda=0.1$, and one which is shown at $\lambda=1$ and $\lambda=10$.

Our choice of $\lambda$ was driven by the ability to differenciate between Mitocheck hit experiments. As detailed in section~\ref{sec:selection_mitocheck}, these experiments are grouped according to the phenotype(s) of which they present a strikingly high percentage. We therefore visually compared the ability of $\lambda$'s two different value ranges to separate Mitocheck hit experiment from different phenotypic hit lists. 

This is shown on figures~\ref{lambda_choice_2} and~\ref{lambda_choice_3}: Mitocheck hit experiments are represented following the use of multi-dimensional scaling in 2 dimensions of their \textbf{global Sinkhorn divergences}. We clearly see that $\lambda=10$ seems to distinguish - to a certain extent - between distinct phenotypes, as opposed to $\lambda=0.1$. This is striking if we consider the example of \textbf{Binuclear} and \textbf{Cell death} hits (see also figures in Appendix, section~\ref{choice_param_Sinkhorn}). $\lambda=10$ was hence chosen.

\begin{figure*}[ht!]
\centerline{\includegraphics[scale=0.3]{figures/transportation_convergence_right_seg.png}
}
\caption{Convergence of Sinkhorn divergence as a function of lambda. Divergences were computed between drug screen experiments and Mitocheck hit experiments for different values of lambda, and the distribution of their relative variation to the divergences computed for $\lambda=30$ are showed here.}
\label{lambda_choice1}
\end{figure*}
\begin{figure*}[ht!]
\centerline{
\includegraphics[scale=0.25]{figures/MDS_lamb01.png}
}
\caption{Separation between Mitocheck hit categories (left) for $\lambda=0.1$. Global Sinkhorn divergences between Mitocheck hit experiments were computed for $\lambda=0.1$, and multi-dimensional scaling was used for representing them in two dimensions in the first two lines. Divergences between these experiments and the drug screen were included and their multi-dimension scaling is shown on the right plot.}
\label{lambda_choice_2}
\end{figure*}
\begin{figure*}[ht!]
\centerline{
\includegraphics[scale=0.25]{figures/MDS_lamb10.png}}
\caption{Separation between Mitocheck hit categories (left) for $\lambda=10$. Global Sinkhorn divergences between Mitocheck hit experiments were computed for $\lambda=10$, and multi-dimensional scaling was used for representing them in two dimensions in the first two lines. Divergences between these experiments and the drug screen were included and their multi-dimension scaling is shown on the right plot.}
\label{lambda_choice_3}
\end{figure*}

\subsubsection{Distance quality evaluation}
Six distances were selected to compare phenotypic profiles following drug/siRNA exposure. We then wanted to evaluate their ability to distinguish between different conditions without distinguishing between condition replicates. For this purpose, we computed for each distance $d$ and condition $C$ a separability score $S_d(C)$ as defined in formula~\ref{sepa} and a replicability score $R_d(C)$ as defined in formula~\ref{replica}. Separability compares the distance between replicates of the same condition to the distance to other conditions, whereas replicability measures the correlation between condition replicates to Mitocheck hit experiments.

Notations: for each experiment $i$ we note $C_i$ its condition, $d(i,M)$ the vector of distances between $i$ and all Mitocheck hit experiments and $corr$ the Pearson correlation.
\begin{equation}
S_d(C)=\dfrac{\sum_{\substack{i|C_i= C}} \sum_{\substack{k|C_k\neq C}} d(i,k)}{\sum_{\substack{i|C_i= C}} \sum_{\substack{j\neq i| C_j = C}} d(i,j)}
\label{sepa}
\end{equation}

\begin{equation}
R_d(C)=\dfrac{2}{(n-1)(n-2)}~ \sum_{\substack{i\\C_i= C}} \sum_{\substack{j\neq i\\ C_j = C}} corr(d(i,M), d(j,M))
\label{replica}
\end{equation}

The results are presented fig.~\ref{separability}. We can observe that all investigated distances score the same on average in terms of replicability and separability on drug screen hit conditions ; they are more different on all drug screen conditions. In the latter case, Sinkhorn divergences and the simple phenotypic score distance seem to better separate conditions than normalized phenotypic score and phenotypic trajectory distance, although not significantly. 

In both cases, there is a high standard deviation, as some conditions were visually observed to have lower reproducibility levels than the others. One can for example consider the reproducibility of the 10th dose of JNJ7706621 (see fig.~\ref{jnj} in appendix, section~\ref{sec:jnj}). This high standard deviation is therefore at least partly experimental, which means that none of the investigated distances is robust enough to cover it.

Based on these results, we chose to restrict ourselves to the phenotypic score distance, the phenotypic trajectory distance, the sum of time and global Sinkhorn divergences.


%For the separability score here it's distance to other conditions/distance to replicates if there is more than one replicate that passed QC. We look at hit conditions only because we feel that's where the real things are happening. It's easy to get a high reproducibility when nothing happens basically. Based on that we can restrict ourselves to phenotypic trajectory, euclidean distance on phenotypic score and either sum or max of time Sinkhorn distance.
\begin{figure*}[ht!]
\centerline{
\includegraphics[scale=0.3]{figures/separability_score_all_exp.png}
\includegraphics[scale=0.27]{figures/separability_score_hit_only.png}
}
\caption{Mean separability and replicability scores of investigated distances on all conditions (left) and hit conditions only (right - bars represent standard deviations).}
\label{separability}
\end{figure*}
%\begin{figure*}[ht!]
%\centerline{\includegraphics[scale=0.3]{figures/correlation_different_plates_hits_on_dist_to_Mitocheck_ttransport_max.png}
%}
%\caption{Replicate correlation of distances to Mitocheck hits}
%\label{replicate_correlation}
%\end{figure*}
%{'N_pheno_score': 0.59362688197421543,
% 'U_pheno_score': 0.48755546170997333,
% 'nature': 0.4740500150584811,
% 'transport': 0.3754272501191217,
% 'ttransport_INT': 0.41390462544644724,
% 'ttransport_MAX': 0.3963477934703899}

\subsection{Applications}
There are two main applications of phenotypic profiling to drug or small molecule screen experiments~\cite{pmid17401369}. On the one hand, one might be interested in studying the similarity between the different conditions cells were exposed to. On the other hand, if there exists a parallel siRNA screen, that is, a screen which was performed in the same experimental conditions, one might be interested in comparing phenotypic profiles resulting from gene silencing to phenotypic profiles resulting from drug exposure. These two applications could lead to the inference of possible targets for unknown chemicals, and might even give a hint as to the possible mode of action.

\subsubsection{Small molecule similarity evaluation}
\paragraph{Condition clustering}
The median distance to Mitocheck hit experiments was used to perform condition hierarchical clustering for each investigated distance. The output for the global Sinkhorn divergence is illustrated fig.~\ref{cond_clust_transport}, and outputs for the other distances can be seen in appendix (see section~\ref{appendix:heatmaps}).

From 89\% to 100\% of (drug screen) hit conditions are clustering together, depending on the distance at hand. Furthermore, the clusters they belong to are composed of hit conditions at purity levels ranging from 85\% to 91\%. The dendrogram shows that their clusters are at a certain distance from eachother (i.e. they are not flat). On the other hand, non-hit conditions are grouped in one large and flat cluster. This means that non-hit conditions as described by their distance to Mitocheck hit experiments are not distinguishable, whereas hit conditions are.

We have made the choice to summarize experiments by their nuclear phenotypic profiles, that is, the temporal sequence of nuclear phenotype distributions. Mitotic hits are conditions which alter cell mitosis. Hence they perturb the basal distribution of nuclei between normal nuclear phenotypes such as \textit{Interphase}, \textit{Metaphase}, etc. By construction, they are easily detected by our approach if they act within the experiment duration. 

It was not clear at the beginning that other type of hits could not be detected in our approach. From these clusterings, it nevertheless seems that chemicals which have an impact on different cellular processes, such as thalidomide whose known teratogenic effects might be linked to an inhibition of ubiquitin ligase~\cite{pmid20223979}, cannot be detected. It could also be that they do not have any effect on HeLa cells, or that they would need more exposure time for an effect to be detected.

%The precise mechanism of action for thalidomide is unknown, but possible mechanisms include anti-angiogenic and oxidative stress-inducing effects.[22]

The conclusion of these clusterings is that our information representation is suitable for detecting and infering knowledge regarding mitotic hit conditions, whereas it is not for non-hit conditions. We therefore restrict ourselves to hit conditions in the following.
%POINT 1 In theory we could learn from the screen even if the drug is not directly targeted at something directly involved in mitosis but here we see that no
\begin{figure*}[ht!]
\centerline{\includegraphics[scale=0.55]{figures/Clust_0transport_ward_ward.png}}

\caption{Drug screen condition - Mitocheck siRNA two-dimensional hierarchical clustering using global Sinkhorn divergence. Ward method was used in combination with the Euclidean distance.}
\label{cond_clust_transport}
\end{figure*}

POINT 2 so we focus on hits detected on phenotypic scores. Here is what we have in terms of hit reproducibility: means it's high. Good sign/ better say in terms of more than 50\% experiments that passed QC
Counter({3: 55, 2: 14, 1: 4})

Not working well for phenotypic score with other clustering methods as well (not shown, 'single' and 'centroid'). Working better for phenotypic trajectory distance: centroid
\subsubsection{Target pathway inference}
3.Link with Mitocheck for the hit experiments


\begin{figure*}[ht!]
\centerline{\includegraphics[scale=0.4]{figures/inference_evaluation_gene_rank.png}
}
\caption{Index of known drug targets in function of their closeness to drug screen experiments as measured by the different distances}
\label{gene_rank}
\end{figure*}
%Rang moyen des cibles connues en fonction des differentes distances:

%Taking conditions with 2 or 3 replicate hits
%(['Normalized\n phenotypic score',
%  'Phenotypic score',
%  'Global\n Sinkhorn div.',
%  'Phenotypic\n trajectory',
%  'Sum of time\n Sinkhorn div.',
%  'Max time \n Sinkhorn div.'],
%MEAN array([ 919.59701493,  697.86567164,  748.67164179,  728.32835821,
%         730.94029851,  790.94029851]),
%STD array([ 771.89969304,  697.96328209,  722.39546345,  722.65462918,
%         818.89530395,  833.70951097]))

%Conditions with 3 replicate hits
%(['Normalized\n phenotypic score',
%  'Phenotypic score',
%  'Global\n Sinkhorn div.',
%  'Phenotypic\n trajectory',
%  'Sum of time\n Sinkhorn div.',
%  'Max time \n Sinkhorn div.'],
% array([ 858.42105263,  650.49122807,  641.43859649,  645.56140351,
%         643.21052632,  732.49122807]),
% array([ 735.60186475,  666.84186618,  629.07743694,  655.08301382,
%         744.61728629,  800.71350031]))

%=> Pour la suite je me limite a 
%  'Phenotypic score',
%  'Global\n Sinkhorn div.',
%  'Phenotypic\n trajectory',
%  'Sum of time\n Sinkhorn div.',

\subsection{Discussion}
SOME ELEMENTS FOR THE DISCUSSION

Regarding the time transport distance, another idea could be to do dynamic time warping on frame correspondence rather than just 1-to-1. Indeed, the effect of the siRNA which we observe, is function of protein half-life, [prot] necessary for the cell to get its function rightly performed, as well as  if the prot has a direct or indirect function to play in cell division (or cell motility or whatever the process is that we consider).

Furthermore, the phenotypic onset of the siRNA can be rather different than that of the drug because the drug directly acts on its target rather than get the mRNAs of this target destructed.

Something else for the discussion: can have more than one target, and also not to forget are off-target effects

Also: penetrance of phenotypes always higher in the drug screen than in siRNA experiments

\cite{pmid18066055} work directly on feature distributions from the experiments (we work on nuclear phenotypes which are already an information summary of each image). Idea, we should check what we get by staying at the feature level on each image.

\clearpage
\lhead{Chapter 4 - \emph{Xenobiotic screen}}

\chapter{Xenobiotic screen}
\label{chap:xbsc}
\begin{table}[!ht]
\begin{tabular}{|l|}
\hline
~\\

\textbf{Résumé~}(see \textit{infra} for English text)\\
\parbox{15cm}{Environ 100,000 nouvelles molécules sont synthétisées chaque année. D'autre part, les réglementations européennes sont de plus en plus strictes en ce qui concerne l'expérimentation animale. L'utilisation de cribles biologiques à haut débit et haut contenu semble donc indiquée en toxicologie environnementale : ils constitueraient une procédure expérimentale ayant l'avantage d'être à la fois \textit{in vitro} et très informative. Toutefois, la plupart des tests actuellement utilisés en toxicologie environnementale sont à faible contenu, ou analysés manuellement.

Afin d'étudier la faisabilité d'une telle approche, nous avons réalisé un crible de 5 xénobiotiques connus comme la dioxine (TCDD), produisant un jeu de données de vidéomicroscopie à épifluoresence. Ces données ont d'une part été analysées à l'aide de MotIW~\cite{motiw} pour la motilité cellulaire individuelle (cf. section~\ref{sec:xb_motility}), d'autre part à l'aide de la procédure dévelopée par~\cite{Walter2010} pour le cycle et la division cellulaires (cf. section~\ref{sec:phenostudy}). Une interface Web a également été conçue pour le partage des résultats entre les laboratoires, présentée dans la section~\ref{sec:interface}.

Toutefois, ces expériences n'ont pas permis de conclure à l'utilité de l'approche pour l'étude de l'impact sub-toxique des xénobiotiques choisis. Plusieurs pistes sont discutées dans la section~\ref{sec:discussion}: il serait dans un premier lieu intéressant de réaliser une batterie d'expériences avant le crible afin de choisir un ensemble de doses resserré. Dans un second lieu, le choix de la lignée cellulaire pourrait être revu.}\\
~\\
\hline
\end{tabular}
\end{table}
\clearpage
Environmental health consists in studying the impact of Man's environment on his health. This can be done following either one of two major paradigms: \textit{Epidemiology} and \textit{Toxicology}. The former deals with human populations, looking for significant link between past exposure and present pathologies. Toxicology can itself be \textit{in vivo}, \textit{in vitro} or \textit{in silico}, depending whether one chooses to study the impact of a precise exposure on a population of living organisms, a population of cells, or based on chemical descriptors of the exposure.

Causal links can be quite delicate to establish in Epidemiology. There is a great number of variables which are involved (e.g. genetic or behavioral) ; the effects can have a weak penetration and are often delayed by a certain number of years with respect to the exposure. To mitigate these effects, Toxicology enables to select the chemical (or xenobiotic\footnote{A xenobiotic is, with regard to a species, any compound which was not produced by an individual of this species.}) of interest, as well as precisely control the experimental settings.

However, Toxicology is currently facing two major challenges:
\begin{itemize}
\item Around 100~000 new compounds are synthesized each year. There is therefore the need for high-throughput (HT) and safe toxicity tests.
\item European regulations are stricter and stricter with animal testing, hence the need for novel \textit{in vitro} toxicity tests.
\end{itemize}
\paragraph*{}	
This led us to think of importing the technique of drug screening from pharmacological Toxicology, replacing prospective drugs with xenobiotics. The basic idea of screening is to perform a given assay on hundreds of compounds in parallel.

Classical \textit{in vitro} toxicological screening procedures have been in use for some time, such as the Comet assay for DNA damage, which is known since 1984~\cite{pmid6477583}. This type of test is now routinely done in a HT setup. However, it provides very crude information compared to what high content (HC) assays would indicate regarding for example, cell cycle modulation following BPA exposure. Subtler tests are being developed in a HT setup, either endpoint~\cite{pmid24772387},~\cite{pmid24610750} or real-time assays~\cite{pmid21516415}, \cite{pmid24141454}. This was recently made possible  by technical and computational progresses.

Nevertheless, the majority of new \textit{in vitro} toxicological methods are not HC\footnote{3 papers dealing with new toxicological assays out of 10 present HC methods (PubMed searches: "environmental AND cellular AND (screening[Other Term] OR High throughput/high content assays[Other Term])", and "time-lapse AND toxicology" on the 3/19/2015).}, i.e. they do not allow detailed observation of phenotypes at the single cell level. Such methods are common practice in molecular biology and have been scaled up, so as to be presently applicable in a HC setup. More specifically, there are many biological processes which are best studied with time-lapse microscopy. Surprisingly, this type of experiment is still rarely used in Environmental Toxicology - regardless of the throughput. When it is, data is manually analyzed most of the time (e.g.~\cite{pmid17949680},~\cite{pmid15388243}, or~\cite{pmid24263567}). However, time-lapse experiments would be a significant improvement over endpoint assays, as they enable for example to assess event sequences following exposure rather than record cell death. 

%whose study not only requires access to single cell phenotypes, but 
%Time-lapse experiments are a specific type of HC experiment which deals with cell videos rather than fixed images, and is about to become a common procedure in Molecular Biology. This type of experiment is still rarely used in Environmental Toxicology - regardless of the throughput. When it is, data is manually analyzed most of the time (e.g.~\cite{pmid17949680},~\cite{pmid15388243}, or~\cite{pmid24263567}).

Environmental Toxicology time-lapse data was therefore newly generated, in order to assess whether HC time-lapse screening is applicable in this context : is this approach relevant for toxicity detection and characterization of environmentally relevant compounds? Time-lapse HCS usability in this context would lead to the potential development of a time-lapse HC-HT assay for Environmental Toxicology. %Indeed, until recently, Toxicology experiments were either HC and low-throughput, or low-content and HT.

Hence five well-known xenobiotics were selected and screened for their effects on nuclear motility and nuclear morphology. The whole pipeline is illustrated fig.~\ref{xbsc_workflow} (and the experimental settings are detailed fig.~\ref{exp_setting}). Briefly, cells were chemically exposed prior to image acquisition over time. Cells were segmented on each image of each experiment using the open-source software CellCognition~\cite{cellcognition}. Object features were extracted using the same software, for two purposes: nuclear tracking and nuclear morphology classification. 

Nuclear tracking was performed as described in the methodological article~\cite{motiw}. This step was followed by trajectory feature extraction, and statistical hit detection, as described in the same article.

CellCognition was used to establish a training set of annotated nuclear morphologies, to train a classifier and apply this classifier to each nucleus in the data set. This allowed us to describe each experiment by a set of class percentage time series, whose comparison with control time series allow us to detect significant differences.

As this was a proof of concept experiment, the goal was slightly different than that when applying MotIW on the Mitocheck dataset. In the latter case the goal was to select relevant genes for nuclear motility with high confidence, which explains why an ad hoc statistical procedure had to be developed to make sure that p-values were not over-estimated. In the case at hand, the goal was rather to select experimental conditions with mild-to-high confidence for performing confirmatory experiments. Our goal was therefore to obtain a ranking of all conditions with respect to the tested endpoint (motility or cell cycle) rather than computing absolute p-values.

\begin{figure}[!ht]
\centerline{\includegraphics[scale=0.33]{figures/workflow_xb_complet_fused_calques.png}}
\caption{Xenobiotic screen complete workflow}
\label{xbsc_workflow}
\end{figure}

\section{Materials and methods}
\label{protocoles}
\subsection{Experimental work}
\subsubsection{Chemicals}
TCDD was bought from LGC Standards (Molsheim, France), TGF~$\beta$1 from R\&D Systems\texttrademark \\ (Minneapolis, MN, USA), BPA, MeHg and PCB153 from Sigma Aldrich\up{\textregistered} (Saint-Louis, MO, USA). $\alpha$-Endosulfan was bought from Cambridge Isotope Laboratories, Inc. (Tewksbury, MA, USA) and DMSO from Merck Millipore (Billerica, MA, USA).

The following media were used:
\begin{itemize}
\item normal medium: DMEM (Gibco\up{\textregistered}, Life Technologies\texttrademark, Puteaux, France) with phenol red and supplemented with 10\% fetal calf serum (FCS), 200 units/ml penicillin, 500 $\mu$g/ml streptomycin, 3g/ml glutamin, 10$\mu$g/mL insulin and 0.1nmol/mL %Milieu avec Aa non essentiels 0.1nM final et insuline 10microgramme par ml.
of non-essential amino acids solution (all from Life Technologies\texttrademark), and 0.5 $\mu$g/ml fungizone (Squibb, Princeton, NJ, USA),
\item imaging medium: CO$_2$-independent medium (Gibco\up{\textregistered}) supplemented with 10\% FCS, 200 units/ml penicillin, 500 $\mu$g/ml streptomycin, 3g/ml glutamin, 10$\mu$g/mL insulin and 0.1nmol/mL %Milieu avec Aa non essentiels 0.1nM final et insuline 10microgramme par ml.
of non-essential amino acids solution, and 0.5 $\mu$g/ml fungizone.
\end{itemize}

\subsubsection{Cell culture}
The human mammary tumor cell line MCF-7 (ATCC\up{\textregistered} Catalog N\up{o}HTB-22\texttrademark)  was maintained in normal medium as defined above. %Forty-eight h before any experiment, the medium was removed and replaced by DMEM without phenol red (Life Technologies\texttrademark) with 3\% charcoal-treated calf serum (i.e. desteroidized calf serum), and the same concentrations of penicillin, streptomycin, and fungizone as described above\footnote{Medium called "experiment medium" in the following.}. 
\subsubsection{Cell transfection and clonal selection}
MCF-7 cells were transfected with two plasmids : one containing human histone H2B fused to the gene encoding a red fluorescent protein (mCherry), isolated from Discosoma species (Addgene, plasmid \#21045), one containing human membrane lipid Myr/Palm fused to the gene encoding the green fluorescent protein (GFP) of Aequorea victoria (Addgene, plasmid \#21037). The aim was to generate a stable line constitutively expressing H2B-mCherry and Myr/Palm-GFP.

On the day before transfection, MCF-7 cells were seeded into 10cm dish with normal medium (2 millions per dish). On the day of transfection, cell medium was replaced with 2mL of normal medium and a mix composed of 72$\mu$l Lipofectamin\up{\textregistered} 2000 reagent and 24$\mu$g of total plasmid DNA (either H2B-mCherry alone, Myr/Palm-GFP alone or both), completed to a volume of 3mL with Optimem (Life technologies\texttrademark). Cells were then incubated for 5 hours at 37\up{o}C, after what 5mL of normal medium was added. 
Antibiotic selection started 7 days after transfection, with 1mg/mL of neomycin and 1$\mu$g/mL of puromycin added to normal cell medium ; selection medium was replaced every other day.

Clonal selection was realized by infinite dilution: three weeks following the beginning of antibiotic selection, transfected cells were seeded in two 96-well plates with 64 wells at 0.3 cell/well, 64 wells at 1 cells/well and 64 wells at at 3 cells/well. Once clones had suffienctly grown, a few clones were selected based on their level of plasmid expression. They were tested for the expression of the following genes with and without exposure to 25nM TCDD for 48 hours%I don't think it's 48 hours but Celine did not answer
: Aryl hydrocarbon receptor  (AHR), Cytochrome P450 1A1 (CYP1A1) and E-cadherin genes. Most clones responded as expected (increased expression of AHR and CYP1A1 and decreased expression of E-cadherin - data not shown). One clone was discarded.

Following this step, modified cells were maintained in normal medium with 1mg/mL of neomycin and 1$\mu$g/mL of puromycin. Hereafter, "MCF-7 cells" refers to a selected clone of such modified cells.

%\paragraph{RNA extraction, reverse transcription and RT-qPCR} Total RNAs were extracted using the RNeasy mini kit (Qiagen, Les Ulis, France) and reverse transcription was performed using the cDNA high-capacity archive kit (Applied Biosystems) according to the manufacturer's instructions. The primers used for the real-time PCR are available on request. Quantitative real-time PCR was carried out in a 10-ml reaction volume containing 40 ng of cDNA, 300 nM of each primer and ABsolute QPCR SYBR Green (Abgene, Villebon sur Yvette, France) using an ABI Prism 7900 Sequence Detector system (Applied Biosystems). PCR cycles consisted of the following steps: Taq activation (15 min, $95^\circ$C), denaturation (15 s, $95^\circ$C) and
%annealing and extension (1 min, $60^\circ$C). The threshold cycle (Ct) was measured as the number of cycles for which the reporter fluorescent emission first exceeds the background. The relative amounts of mRNA were estimated using the DDCt method with the gene RPL13A (coding for 60S ribosomal protein L13a) for normalization.

\subsubsection{Production of 96 well-plate for imaging}
48h prior to imaging, MCF-7 cells were seeded in normal medium into one 96 well-plastic plate, at a density of 6,000 cells per well. The plate was placed back in the incubator at constant temperature ($37^\circ$C) and CO$_2$ pressure (5\%). 24 hours prior to imaging, dilutions (cf table~\ref{dilutions}) of selected compounds with constant solvent percentage were freshly prepared in imaging medium. Cell medium was changed with 198$\mu$L of imaging medium and 2$\mu$L of chemical dilution or solvent dilution per well. Six control wells for each solvent and three wells with cell medium only were put on each plate. Plate design was not random, but control wells were not grouped (see fig~\ref{plate_setup} for an example of a plate setup).

The plate was placed back in the incubator. Immediately before image acquisition, the plate was sealed using an adhesive optical film% (\textbf{ref. **}) wrong reference given by Celine
. This procedure is illustrated on figure~\ref{exp_setting}.

\begin{figure}
\centering
\includegraphics[scale=0.3]{figures/experimental_setting.png}
\caption{Illustration of the experimental settings}
\label{exp_setting}
\end{figure}
\begin{figure}
\centering
\includegraphics[scale=0.5]{figures/plate--271214.png}
\caption{Example of a plate setup. \textit{Cl1}: clone number, \textit{Indpt 10}: $CO_2$-independent cell medium with 10\% FCS.}
\label{plate_setup}
\end{figure}
\subsubsection{Time-lapse imaging}
Images were acquired every 15 minutes in each well for forty-eight hours, with an automated epifluorescence microscope (Axio Observer Z1; Zeiss, Oberkochen, Germany) with motorized objectives in z-axis (resolution 10nm) and using 10x objective (EC Plan-Neofluar;0.3 M27). Each well was imaged 800ms at lengthwave $\lambda = 555nm$ (mCherry) and 300 ms at $\lambda = 470nm$ (GFP).

The microscope is integrated into a microscope incubation chamber to provide constant temperature (+37\up{o}C), which was turned on at least one hour prior to the beginning of image acquisition. Zeiss software ZEN2011 was used for data recording. Focus was done manually, and was updated by definite focus.
\subsubsection{Phototoxicity assays}
One plate was prepared with 12 wells following the above-described procedure, except that no chemical was added prior to image acquisition. Images were acquired for 24h, with the four different imaging conditions :
\begin{itemize}
\item no imaging
\item 500ms at $\lambda = 555nm$, 300 ms at $\lambda = 470nm$
\item 1~000ms at $\lambda = 555nm$, 300 ms at $\lambda = 470nm$
\item 1~500ms at $\lambda = 555nm$, 300 ms at $\lambda = 470nm$
\end{itemize}
2$\mu$L of \textit{alamarBlue\texttrademark Cell Viability Assay Reagent} (Thermo Scientific, Rockford, IL, USA) were then added to each well. Following 2h of incubation at $37^\circ$C, absorbance was measured at $\lambda = 570nm$ and $600nm$ using an EnSpire\up{\textregistered} (PerkinElmer, Waltham, MS, USA). The index of cell viability was computed following the manufacturer's formula.

\subsubsection{Chemical dose choice}
Doses were chosen so that the lower end is close to human exposure levels, and that the higher end is not cytotoxic.

The first goal was attained through a literature review, whose results can be seen in appendix, section~\ref{literature_review}. The second goal was attained through cytoxicity tests. One plate was prepared with 76 wells as described above, although with 5,000 cells per well. 24h after seeding, cells were exposed either to solvents, either to doses 6 to 10 of selected compounds. 24h after exposure, 2$\mu$L of \textit{alamarBlue\texttrademark Cell Viability Assay Reagent} were added to each well. After 24h of incubation, fluorescence was measured in each well at $\lambda = 555nm$, and cell viability index was computed following the manufacturer's formula. In the end, 9 to 10 doses for each of the five chosen xenobiotics were selected, the total summing to approximately 50 different conditions to which cells were exposed.

\subsection{Bioinformatics methods}
\label{interface}
\subsubsection{Web-based user interface for result visualization}
\label{sec:interface} A web-based user interface was designed for raw  and quality control data visualization. It is accessible at the following address: \href{http://olympia.biomedicale.univ-paris5.fr/plates/}{http://olympia.biomedicale.univ-paris5.fr/plates/} (login: \textit{user}, password: \textit{xbscreen}). It was implemented using Django\footnote{\href{https://www.djangoproject.com/}{https://www.djangoproject.com/}} web framework, the programming language Python 2.7\footnote{\href{http://www.python.org}{http://www.python.org}}, and runs under Linux-Apache web server and mod\_wsgi module. SQLite\footnote{\href{https://sqlite.org/}{https://sqlite.org/}} was used for storing experimental metadata (plate setups, experimental conditions such as cell medium or the percentage of FCS). The database was built as visible on fig~\ref{db}.

Briefly, each well is linked to a unique plate, a unique condition (medium and percentage of FCS) and a unique treatment (xenobiotic and dose). As the data was (crudely) password-protected, a second database contains logins and passwords, as well as admin permissions.

After logging in, any user has access to the list of plates in anti-chronological order. A plate page displays the plate setup, as well as some overall features of all experiments, such as initial number of objects or the proliferation rate in each well. This enables to see a significant geographical bias, which could be due to chemical exposure or the experimental settings. In the latter case, the plate should not be used for analysis and the experimental protocol modified ; such visualization tools were useful for designing the protocol. 

By clicking on a particular well on any image of the plate page, one accesses per-well information: well experimental conditions, well raw movies, and time evolutions of mean intensity, percentage of out-of-focus objects, number of cells, number of objects, and percentage of objects in all classes (as detailed below).

\begin{figure}
\centering
\includegraphics[scale=0.4]{figures/schema_db2.png}
\caption{Diagram of the databases for experimental metadata storage}
\label{db}
\end{figure}

%The WCTDS has been implemented by using Django
%web framework (https://www.djangoproject.com/), R 3.0
%(http://www.r-project.org/), Python 2.7, and Linux shell pro-
%gramming, and it runs under Linux-Apache web server and
%mod wsgi module. MySQL is used to build a database for
%communicating between user input and the data storage.
%Python and Django are used to implement the web interface.
%The whole software runs under Linux shell. The computa-
%tional core, including mathematical models, data processes,
%and result generation, is implemented by R, which was
%detailed in our previous publication [9]. To speed up the
%computation, parallel computation was implemented to run
%the mathematical models.
%To reduce the complexity of its usage, WCTDS runs all
%functions and computations behind the screen under Linux
%shell and only requires a simple data frame as input. The sub-
%mitted data frame is directly passed by Python functions to R
%serial functions, including quality check, matrix preparation,
%parallel computation of math models, result summary, graph
%plots, and compressing all results in a zip file, which is sent
%back to web server coded by Python and Django for user
%download.

\subsubsection{Quality control}
Due to microscope hardware and/or software instability, images showed mean intensity variation on both channels. Hence, images $t$ whose mean intensity on the mCherry channel $I_t$ verified the following inequality were not considered for further analysis, where $\sigma_I $ is the standard deviation of $(I_t)_{t=1\ldots T}$:
\[ |I_t- \dfrac{\sum_{i=0\ldots 10} I_{t+i} }{10}| > 3\sigma_I  \]

Furthermore, a threshold of $c$ cells at the beginning of the movie, and maximum $p$\% out-of-focus objects were used to further remove unexploitable movies.   
We fixed the threshold after visual inspection to $c=23$ and $p=37$. Out-of-focus objects and cells that were neither artefacts nor out-of-focus objects were identified following segmentation, feature extraction and classification as described below.
%image quality control, experiment quality control

\subsubsection{Object segmentation}

Object segmentation was done using a newly-designed plugin implemented
in the open-source software CellCognition~\cite{cellcognition}. MCF-7
cells are smaller than HeLa cells and tended to form clusters in our
experiments. While the method described in ~\ref{sec:motiw_seg}
and previously published in several papers, e.g. in
\cite{cellcognition}, is in principle capable of splitting clustered
nuclei, we felt that the filtering of the distance transform, 
which ultimately influences the decision on whether to split or not, was suboptimal. 
Here, we used morphological dynamics to improve the splitting step of
this segmentation method.

The first steps of the method are therefore identical to what we have
presented in ~\ref{sec:motiw_seg}: images are prefiltered (here by a
median filter) and we assign to each pixel the difference of its value
to the average in a window centered in the pixel. This average value
can be efficiently calculated with integral images. By applying a
global threshold to the residue image, we obtain a first segmentation
result which gives accurate results for isolated nuclei, but which
tends to segment close nuclei together as a single object. 

The last step is to calculate the Watershed transformation on the
inverse distance map. This is a
standard technique to divide close convex objects after
segmentation~\cite[Chapter Geodesic segmentation]{lantuejoul}. This
method splits the binary segmentation
result in as many objects as there are local minima in the inverse
distance map. As small irregularities in the contours can lead to such
minima, it is often necessary to apply a filter on the distance map in
order to avoid oversegmentation. One option is to use a simple
Gaussian filter, as proposed in ~\cite{Wahlby2002} and
~\cite{cellcognition}. However, this does not permit an intuitive
control over which minima are really kept, and even worse: it does not guarantee that larger filters always suppress more minima. 
Here, we propose to use morphological dynamics for this
purpose~\cite{Soille:2003:MIA:773286}.  While this technique is widely
used in the morphology community, it is to our knowledge not used for
the splitting of cells, even though it perfectly applies to this
problem. 

Morphological dynamics assign to each local minimum the value at which
it fuses to a region coming from a minimum with lower value (the
dynamic of the lowest minimum is set to $\infty$). As the watershed
algorithm iterates through the values in an ordered way, this value is
identical to the minimum height that has to be passed to reach a lower
minimum. Let $p_{i,j}$ be a path that joins minima $i$ and $j$, the dynamic of minimum $i$ is the following:
\[ dyn(i) = \min_{\substack{p_{i,j} \\ f (j)<f (i)}} \max_{\substack{x\in p_{i,j}}} (f(x) - f(i))\]

Hence, to avoid the usual over-segmentation produced by the watershed
algorithm, we only use the subset of minima with dynamic larger than a
certain threshold. The number of objects decreases as this threshold
increases, and the control is intuitive: small concavities produce
local minima with relatively low dynamic (independently from their
spatial arrangement or the size of the objects). 

The computational cost of the morphological dynamics is the same as
the watershed algorithm. Depending on the size of the data sets, this
is not negligible, even though it compares favorably to 
smoothing filters, as they were used in this context. To further reduce the
computational complexity, we implemented the dynamic filter in such a way that it is directly
combined with the watershed algorithm. Indeed, the criterion can be checked each
time a pixel is about to be assigned to the watershed line. Consequently no additional flooding step is necessary. 

%Briefly, the inverse distance map of a binary image $X$ assigns to each of its points $x$ the inverse of the distance to the nearest background point: 
%\[ D(x) = \dfrac{1}{\min_{\substack{y\not\in X}} d(x, y)}\]
%d can be any arbitrary metric, for which popular choices are $\parallel\cdot\parallel_\infty$ (the maximum distance), $\parallel\cdot\parallel_1$ (the $L_1$ distance) or $\parallel\cdot\parallel_2$ (the Euclidean distance). 
%The Watershed transformation partitions the image plane into disjoint regions, which are separated by a one-pixel-wide line (the watershed line). This is done by starting from seeds, the local minima of the image (i.e. regions with constant value, surrounded by pixels with strictly greater value) . The watershed algorithm iterates over the possible values the image mask f can take. Starting with the seeds, it extends each region with the pixels of the current value t. If during this extension a pixel
%could be associated to two regions (i.e. if in the neighborhood of the pixel with value t,
%there are at least two pixels with different region appartenance), the pixel is part of the
%watershed line.
%
%in principle other regions
%could be chosen, too (marker controlled watershed algorithm)

Finally, an object filter is applied to eliminate the objects that are (or whose mean intensity is) too small.

%i. pre-filtering of noise. High frequency signal (in space)
%ii. background subtraction. Low frequency signal (in space)
%iii. Watershed distance to merge objects
%iv. 
\subsubsection{Object feature extraction}
Object feature extraction was done using the open-source software CellCognition~\cite{cellcognition}, as previously described~\cite{Walter2010}. Briefly, for each object on each image, approximately 240 features are extracted, characterizing their shape and texture. These features enable to classify nuclei in user-defined nuclear phenotypic classes and to track them over time.

\begin{table}[!ht]
\caption{Nuclear morphology classes with examples. The \textit{artefact} and \textit{cluster} examples are shown with segmentation contours.}
\label{classexamples}
\centerline{
\begin{tabular}{|ll|}
\hline
\multicolumn{2}{c}{Normal classes}\\
\hline
\textit{Interphase}&\includegraphics[scale=1]{figures/cut_inter1.png}\ \includegraphics[scale=1]{figures/cut_inter2.png}\\
\textit{Pro-metaphase}&\includegraphics[scale=1]{figures/cut_promet1.png}\ \includegraphics[scale=1]{figures/cut_promet2.png}\\
\textit{Metaphase}&\includegraphics[scale=1]{figures/cut_met1.png}\ \includegraphics[scale=1]{figures/cut_met2.png}\\
\textit{Anaphase}&\includegraphics[scale=1]{figures/cut_anap1.png}\ \includegraphics[scale=1]{figures/cut_anap2.png}\\
\textit{Apoptosis}&\includegraphics[scale=1]{figures/cut_apop1.png}\ \includegraphics[scale=1]{figures/cut_apop2.png}\\
\hline
\multicolumn{2}{c}{Aberrant morphology classes}\\
\hline
\textit{Frozen}&\includegraphics[scale=1]{figures/cut_frozen1.png}\ \includegraphics[scale=1]{figures/cut_frozen2.png}\\
\textit{Cluster}&\includegraphics[scale=1]{figures/cut_cluster1.png}\ \includegraphics[scale=1]{figures/cut_cluster2.png}\\
\textit{Folded}&\includegraphics[scale=1]{figures/cut_folded1.png}\ \includegraphics[scale=1]{figures/cut_folded2.png}\\
\textit{Micronucleated}&\includegraphics[scale=1]{figures/cut_micronu1.png}\ \includegraphics[scale=1]{figures/cut_micronu2.png}\\
\textit{Polylobed}&\includegraphics[scale=1]{figures/cut_poly1.png}\ \includegraphics[scale=1]{figures/cut_poly2.png}\\
\hline
\multicolumn{2}{c}{Technical problem classes}\\
\hline
\textit{Out-of-focus}&\includegraphics[scale=1]{figures/cut_focus1.png}\\\
\textit{Artefacts}&\includegraphics[scale=1]{figures/cut_artefact1.png}\\
\hline
\end{tabular}
}
\end{table}

\subsubsection{Object classification}
Object classification was also performed using CellCognition. The
class definitions are illustrated in table~\ref{classexamples}. The
set of morphological classes is supposed to cover the morphological
variability of the screen. The set therefore contains wildtype
morphological classes, such as the morphologies corresponding to the
different mitotic phases and aberrant morphologies indicating the
presence of a phenotype. As our screen consisted in 50 conditions, we
almost exhaustively inspected the dataset and are therefore confident
that no aberrant nuclear phenotype was missed. 

%By default, phenotypic classes are the mitotic classes which the time-lapse permits to observe with a sufficient precision : \textit{interphase}, \textit{pro-metaphase}, \textit{metaphase} and \textit{anaphase} in our case, to which \textit{apoptosis} as well as \textit{out-of-focus} objects and \textit{artefacts} are added. The latter classes are indeed present to various extents in all videos. To these shall be added any aberrant phenotype that can be observed in a visual inspection of the dataset. 

In detail, \textit{Clusters} are clustered nuclei which the
segmentation algorithm failed to split. \textit{Folded} nuclei
represent elongated or round nuclei with two shades of
grey. \textit{Frozen} nuclei are nuclei whose DNA shows a
heterogeneous condensed pattern, persistent over time. Frozen nuclei
either remain in the same class or lead to 
\textit{apoptosis}. Biologically, they might correspond to dying
nuclei and resemble nuclei experiencing phototoxicity (personal
communication, Beate Neumann, EMBL Heidelberg,
Germany). As these nuclei were observed following exposure to certain conditions only, we believe that this class does not translate a simple technological artifact. It would rather be the consequence of cell sensitization to phototoxicity which certain exposures could have produced. 

%They are most probably
%dying nuclei and resemble nuclei experiencing phototoxicity (as kindly
%pointed out by Beate Neumann, EMBL, Heidelberg, Germany). 
%We chose the
%names "\textit{cluster}", "\textit{folded}" and "\textit{frozen}"
%following our visual observations. 

A training set was annotated, containing 2,576 nuclei. Support Vector Machines (SVMs) were used for classification, as they work well for nucleus phenotypic classification (\cite{kovalev} for a comparison of classification algorithms in this context, and for application e.g.~\cite{cellcognition},~\cite{Walter2010}). An RBF (Radial Base Function) kernel SVM was trained, whose parameters were obtained by grid search, using Cell Cognition's interface ($\gamma=2^{-7}$, $C=8$).% The overall classifier accuracy is 65.5\% (10-fold cross validation). 

This step provides us with a representation of each video as a set of time-series, which are the evolution of the percentage of nuclei in each phenotypic class over time. The distance of an experiment $i$ to its reference set $j$ for class $obj$ is the following:
\[
d^{obj}_{i,j} = \int_0^{T} (\% obj_{i,t} - \% obj_{j,t} )~dt
\]
%Maxima are less robust to outlier points which could arise in any of
%the two time series than integrals. Therefore, we chose to replace the
%maximal difference between the two time series which is used in
%phenotypic scores~\cite{Walter2010} by the integral of all
%differences, as the xenobiotic screen data set is relatively noisy. 

Due to the visual similarities between \textit{Micronucleated} and \textit{Polylobed} nuclei in our dataset, these classes were pooled together for computing distances. The corresponding distance is named \textit{Micronucleated} in the following.

\subsubsection{Object tracking and trajectory feature extraction} Those steps were performed as described in the methodological article~\cite{motiw}.


\begin{table}[!ht]
\caption{Chemical dilutions}
\vspace{0.4cm}
\label{dilutions}
\begin{tabular}{|l|l|l|l|}
\hline
Chemical & Solvent & \parbox{3cm}{Final solvent percentage (vol)} & Doses (nM) \\
\hline
BPA & DMSO & $1.0~10^{-1} $ &0.1, 1, 10, 50, 100, \\
&&&1~000, 5~000, 10~000, 50~000\\
\hline
Endo & DMSO & $2.0~10^{-1} $ &1, 10, 50, 100, 500,\\
&&& 1~000, 5~000, 10~000, 50~000, 100~000\\
\hline
MeHg & DMSO & $1.0~10^{-3} $ & 0.01, 0.1, 1, 5, 10,\\
&&& 50, 100, 500, 1~000\\
\hline
PCB153 & DMSO & $3.6~10^{-1} $ &0.1, 1, 10, 50, 100,\\
&&& 1~000, 5~000, 10~000, 50~000,100~000\\
\hline
TCDD & Nonane &$3.2~10^{-2} $ & 0.001, 0.01,	0.025, 0.1, 0.25,\\
&&& 1, 10, 25, 50\\
\hline
\end{tabular}
\end{table}
\section{Results}
\subsection{Preliminary choices}
After testing a few clones for the expected behaviour in response to TCDD exposure, one clone was selected and five plates were imaged following the described procedure, producing 415 videos. In the following, each plate is named after the day image acquisition was launched. The quality control eliminated 22\% of the experiments, leaving 324 experiments for analysis with three biological replicates per condition minimum.

Due to the observed high variability in control trajectory statistics on the different plates (cf fig.~\ref{control_heatmap}), the reference was chosen to be the whole plate as opposed to control wells only. This is a valid choice as long as (1) there is no bias in the distribution of chemical conditions (and possibly results) on distinct plates, and (2) most wells do not show any effect~\cite{pmid19644458}. 
\begin{figure}
\centering
\includegraphics[scale=0.4]{figures/heatmap_control_trajfeatures_medstandardized.png}
\caption{Trajectory feature heatmaps corresponding to control wells. Each line corresponds to an experiment, whose plate and name are indicated. Each column corresponds to a trajectory feature. A robust normalization (with median and inter-quartile range) was applied, using all plate, which still permits to see that control responses vary from well to well and plate to plate.}
\label{control_heatmap}
\end{figure}

\subsection{Motility study}
\label{sec:xb_motility}
MotIW was applied to this dataset, which as described in~\cite{motiw} enabled to go from a set of nuclear trajectories to a single statistic for each experiment. However, this statistic was not evenly distributed with respect to the plates. This was measured by a Mann-Whitney U test comparing the list of statistics for one plate with the list of statistics for all remaining plates. Indeed, plates 271214 and 271114 respectively output p-values of 0.06 and 0.02 at this test.

In the Mitocheck study, all experiments from all plates were ranked together ; a p-value threshold was set ; conditions which were more than 50\% of the time under that threshold were considered as significantly modifying nuclear motility. 

In the case at hand, statistics from all plates cannot be ranked together because of an important batch effet. Hence, the approach which was chosen is that of the \textit{RankProduct}~\cite{pmid15327980}. Briefly, this approach consists in formalizing the following idea: we are interested in conditions which have consistently high statistics on the different plates. Our goal is therefore to compute their rank in the statistic list of each plate, and its variations depending on the plate. The \textit{Rank Product} statistic of a condition $c$ is the following, where $pl.~i$ is the $i^{th}$ plate and $card(i)$ the number of experiments performed on the same plate\footnote{If a condition was replicated on the same plate, we used\\
 $rg(c,~i)=median(\{ rg(c_j,i)|j$ technical replicate of $c$ on $i\} )$.}: 

\[RP(c) = \prod_{\substack{pl.~i}} \dfrac{rg(c,~i)}{card(i)} \]

Intuitively, the \textit{Rank Product} of a condition which significantly alters nuclear motility is going to be small. Empirical p-values for the \textit{Rank Product} statistics are computed by permutation, that is, each plate trajectory statistics are permuted $N$ times and the \textit{Rank Product} statistics for each condition computed each time. This produces the empirical null distribution of the \textit{Rank Product} statistics, that is, the distribution of statistics under the hypothesis $H_0$ that no condition alters single nucleus motility more than the average response of all conditions (which was observed to be biologically non-interesting). Empirical p-values are then the proportion of permuted \textit{Rank Product} statistics which are bigger.

Selecting conditions whose empirical p-values are smaller than $0.05$ ($N=10,000$ permutations) produces the following result:
\begin{table}[!ht]
\centering
\begin{tabular}{|l|r|c|l|}
\hline
Condition & Dose & P-value & Example\\
\hline
Endo,10 &$100\mu M$ &0.0001&Supp. movie 1\\
BPA,9 &$50\mu M$ &0.0001 & Supp. movie 2\\
PCB,10 & $100\mu M$ &0.0028&\\
Endo,9 & $50\mu M$ &0.016&\\
MeHg,9 & $1\mu M$&	0.042&\\
PCB,9 & $50\mu M$ &0.048&\\
\hline
\end{tabular}
\end{table} 				%SUPP MOVIE 1 : 271214 w13
%BPA_9 		0.0001					201214 w61
%PCB_10 	0.0028
%Endo_9 		0.0162
%MeHg_9 	0.0416
%PCB_9 		0.0475

Rather than peculiar movement types, according to our approach, the
consistent results are conditions which are so strong as to freeze any
nuclear motion (cf. supplementary movies 1 and 2). We therefore
conclude that motility is significantly altered, albeit not primarily:
measured motility alterations result from potent effects on cell
viability. In contrast, motility alterations in Mitocheck are not coupled to cell
death or cell division phenotypes. Consequently, this confirms that
the identified genes are candidates for cell motility regulators: they do not
trigger motility alterations as a secondary effect of other
alterations, as it seems to be the case here. 

\subsection{Phenotypic study}
\label{sec:phenostudy}
\subsubsection{Phenotypic class selection}
Phenotypic classes were chosen after an almost exhaustive visual
inspection of the dataset. The first result is that except
\textit{frozen} nuclei which were observed in medium and high dose
experiments only, no other striking phenotypes were observed following
xenobiotic exposure only. Indeed \textit{micronucleated} and
\textit{polylobed} nuclei are present in all wells at a non-negligible
base level, and do not significantly appear following xenobiotic
exposure. 

\subsubsection{Results}
As opposed to the motility case, no specific plate distance distribution for any of the classes is significantly different to that of all other plate distance distribution. Hence distinct plate distances can be directly compared. 

When no effect is expected, most cells are in interphase. Therefore, strong effects will be visible in a decrease of \textit{interphase} percentage. A preliminary step consists in looking at conditions for which \textit{interphase} distance is especially low. The \textit{interphase} distances are represented on fig.~\ref{interphase}. One can observe that only high doses (plain red dots) are consistently under the rest of the scatter plot. This depletion of interphases is explained by an increase in \textit{frozen} and \textit{apoptotic} nuclei (cf. the \textit{frozen} distance scatter plot on fig.~\ref{interphase}).

The observation of the distance distributions of other classes does not provide other results as to possible consequences of xenobiotic exposure on cell division, and especially low dose exposure: there is no low dose condition which show a consistent effect over distinct plates for any phenotypic class. Graphs are shown in annex, section~\ref{phenotypic_annex}.

These observations are supported by the computation of \textit{Rank product} statistics for \textit{Interphase}, \textit{Frozen} and \textit{Micronucleated} nuclei (cf. table~\ref{dist_pheno}). The significant increase in \textit{Micronucleated} nuclei following exposure to PCB at the first dose could be visually confirmed in 1 out of 4 experiments only. Furthermore, the level of micronucleation in the latter experiment is comparable to the level that can be observed in other experiments with \textit{nothing} or \textit{Nonane}. Indeed, there is a significant base-level of aberrant cell divisions, as is stressed by the small p-value for wells containing nothing (\textit{Rien}) and visual inspection of the dataset.
\begin{table}[!ht]
\centering
\caption{\textit{Rank product} p-values for different phenotypic distances (<0.05)}
\label{dist_pheno}
\bigskip
\begin{tabular}{|l|r|c|l|}
\hline
Condition & Dose & P-value & Example\\
\hline
\multicolumn{4}{|c|}{Decreased \textit{Interphase} distances}\\
\hline
BPA,9 &$50\mu M$ &0.0001 & \\
MeHg,9 & $1\mu M$&	0.0048&\\
PCB,10 & $100\mu M$ &0.0059&\\
\hline
\multicolumn{4}{|c|}{Increased \textit{Frozen} distances}\\
\hline
BPA,9 &$50\mu M$ &0.0001 & Supp. movie 2\\
PCB,10 & $100\mu M$ &0.0003&Supp. movie 3\\%1212 w 37
MeHg,9 & $1\mu M$&	0.0054&\\
Endo,10 &$100\mu M$ &0.0097&\\
PCB,9 & $50\mu M$ &0.048&\\
\hline
\multicolumn{4}{|c|}{Increased \textit{Micronucleated} distances}\\
\hline
PCB,1 & $0.1 nM$ & 0.015 & Supp. movie 4 \\ %1212 w 48, others nothing
\textit{Rien} & 0 & 0.072 & Supp. movie 5\\ %2011 w 82, 2712 w87
\hline
\end{tabular}

\end{table} 		
\begin{figure}
\caption{Interphase (up) and frozen(down) distances. Colors are black for control wells, yellow to red for xenobiotics ranked by increased dose, and magenta for TGF-$\beta$1. \textit{Legend}: B: BPA, D: DMSO, E: Endo, M: MeHg, N: Nonane, P: PCB, R: nothing, T(red): TCDD, T(magenta): TGF-$\beta$1}
\centering
\label{interphase}
\centerline{
\includegraphics[scale=0.4]{figures/interphase.png}}
\centerline{
\includegraphics[scale=0.4]{figures/frozen3.png}
}
\end{figure}

\section{Discussion}
\label{sec:discussion}
A workflow is established, which is accompanied by a Web interface, to
enable the analysis of time-lapse xenobiotic screening
experiments. However, on the panel of xenobiotics and doses which were
chosen for our set of experiments, it was not possible to detect
subtler effects than a toxic effect at the highest doses.
This is
probably due to one or a combination of the following reasons: we observed a
relatively high level of experimental noise, even in negative
controls, which might be due to (1) the biological variability of the used
cell line, which showed many aberrant divisions, to (2) experimental noise due to microscope intensity
variations and (3) low intensity of the fluorescence signal and (4) the sensitivity of our
trajectory measurements. Even though our results are not conclusive on
this project, we feel that it would be premature to conclude that the screening approach
is not suited for Environmental Toxicology. 

As a general remark, it should not be forgotten that it is very likely that a xenobiotic screen will never be as visually extraordinary as a siRNA or drug screen. Most xenobiotics are not - at sub-toxic doses - targeted at one single vital cellular process, as opposed to some siRNAs (especially those which were enlighted by the Mitocheck project, whose goal was to study cell division). Hence, they are less specific, and their effects can be slower. Nevertheless, previous results had let one hope that, e.g., significant cell division defects could be obtained following TCDD exposure (\cite{pmid20089886},~\cite{pmid18640100},~\cite{pmid11479202}).

\paragraph*{Dose choices}~\\
Ten doses were chosen for each xenobiotic, spanning from four to six orders of magnitude. Dose ranges were selected in order to be non-toxic, as well as to include real human exposure dose (following a literature survey, see appendix~\ref{literature_review}). This may not have been the surest way to detect an effect. Indeed, depending on their mechanisms of action, xenobiotics are going to be active in a smaller range, which we could easily have missed in the current setting.

In the future, dose choices could be improved in two ways. First of all, although human exposure doses are extremely relevant, they may be too small for an exploratory screen, where effects should be observed for the screen to be an efficient proof of concept. Rather, if previous similar studies exist, the doses which they find to be effective on cell division or cell motility should be used. However, given the novelty of the current approach in Environmental Toxicology, such previous similar studies are extremely rare. Previous studies using different cell lines or different measures for evaluating the same process should be considered with caution\footnote{As an example, although there exists several studies about measuring the impact of TCDD exposure on MCF-7 cell motility (e.g.~\cite{pmid16619036},~\cite{pmid22266097}), it is not clear at all that the same parameters are measured by MotIW (single cell motility versus cell population migration).}.

Hence a second and more robust way to improve dose choices would be to perform pre-screening (which also enables quality control setting and instrument and reagent validation - personal communications, Dr Wolfgang Huber and Dr Beate Neumann, EMBL, Heidelberg, Germany). The simplest way is to look for the lowest slightly toxic dose starting from $10$ to $50 \mu M$, and further dilute it of a factor 2 or 3 (rather than 10). 

This is precisely the approach which was chosen by~\cite{pmid24691702}. Willing to define a framework for developmental toxicity test battery, they had to select compounds for performing a proof of concept. They do note that human plasma concentrations are relevant when it comes to environmental contaminants such as PCB153. However given that they are around 1nM for the latter example, they also search for evidence of developmental toxicity in the literature, and do use higher concentrations while pre-screening and screening it. More elaborate approaches to pre-screening are also possible. As an example,~\cite{pmid24997295} used both cell viability assays and measures of cell area, mitochondrial activity and percentage of cells below 2N to define a "zone of interest" for further investigation.

In summary, dose choices should be chosen according to pratical reasons (significant effect observed with local experimental parameters) instead of theoretical reasons (human exposure). Another parameter which should be set in this way is the time-lapse between cell exposure and screen start.

\paragraph*{Cell line choice}~\\
A second parameter of importance is the cell line. Although MCF-7 cells are often chosen as a model for breast cancer, they exhibit a significant basal heterogeneity with regard to oestrogen receptor status and cell area~\cite{pmid11153613}. Screening is already subject to a certain amount of variance in its results, due to cell sensitivity to experimental parameters such as cell local confluence level~\cite{pmid19710653} or passage number, and gene expression stochasticity (\cite{pmid12183631},~\cite{pmid18957198}). For a proof of concept, it may therefore be relevant to choose a cell line which is inherently as homogeneous as possible.

In the current case, MCF-7 cells were genetically modified for incorporating H2B-mCherry and myrPalm-GFP. It appears that the chosen clone exhibits a high basal level of micronuclei. In the future, genetically modified clones should be checked both for a normal response to TCDD exposure and normal cell division.


%Ideas to discuss: \\
%
%1. five xb, 10 doses but maybe pre-select the doses and exposure time first and then study deeper on a smaller concentration range? \cite{pmid24997295} 

%2. motility :  interesting to the extent that the conditions were not detected as toxic following cytotoxicity assays. However visually it looks like phototoxicity. Possible synergy between toxicity and phototox
%
%3. high level of micronucleated nuclei at base-leve => possible pbl with the cell line. Probably made the dataset noisy on a phenotypic level, hence nothing pops out.
\clearpage
%----------------------------------------------------------------------------------------
%	THESIS CONTENT - APPENDICES
%----------------------------------------------------------------------------------------
\appendix
\lhead{\emph{Appendices}}

\chapter{Appendices}
\lhead{Appendices}
%\section{Motility gene list}
%\label{motilitylist}
%\begin{center}
%
%\begin{tiny}
%\begin{longtable}{|l|l||l|l|}
%
%\caption{Cell cycle gene list. In bold are the three genes for which we found an extension of cell cycle length.}
%\\
%
%\hline
%Hugo Gene Name & Ensembl gene id & Hugo Gene Name & Ensembl gene id\\
%\hline
%\endfirsthead
%
%\hline \multicolumn{4}{|r|}{{Continued on next page}} \\ \hline
%\endfoot
%
%\hline \hline
%\endlastfoot
%
%A3GALT2  &  ENSG00000184389  &  LSM4  &  ENSG00000130520 \\
%A4GNT  &  ENSG00000118017  &  LST1  &  ENSG00000204482 \\
%ABCB5  &  ENSG00000004846  &  LYAR  &  ENSG00000145220 \\
%ABI3BP  &  ENSG00000154175  &  LYG2  &  ENSG00000185674 \\
%ABTB1  &  ENSG00000114626  &  LYRM2  &  ENSG00000083099 \\
%AC016885.1  &  ENSG00000268080  &  MAB21L1  &  ENSG00000180660 \\
%AC093677.1  &  ENSG00000269559  &  MAFA  &  ENSG00000182759 \\
%AC233263.1  &  ENSG00000228325  &  MAGEA10  &  ENSG00000124260 \\
%ACADVL  &  ENSG00000072778  &  MAGEB18  &  ENSG00000176774 \\
%ACAP2  &  ENSG00000114331  &  MAGED1  &  ENSG00000179222 \\
%ACE  &  ENSG00000159640  &  MAGEE1  &  ENSG00000198934 \\
%ACKR4  &  ENSG00000129048  &  MAMDC2  &  ENSG00000165072 \\
%ACOT12  &  ENSG00000172497  &  MAN1A1  &  ENSG00000111885 \\
%ACOXL  &  ENSG00000153093  &  MAP2K1  &  ENSG00000169032 \\
%ACSL1  &  ENSG00000151726  &  MAP2K3  &  ENSG00000034152 \\
%ACSM1  &  ENSG00000166743  &  MAP2K7  &  ENSG00000076984 \\
%ACSS3  &  ENSG00000111058  &  MAP3K11  &  ENSG00000173327 \\
%ACTB  &  ENSG00000075624  &  MAP3K13  &  ENSG00000073803 \\
%ACVR1C  &  ENSG00000123612  &  MAP4K2  &  ENSG00000168067 \\
%ADAM32  &  ENSG00000197140  &  MAP9  &  ENSG00000164114 \\
%ADAMTS4  &  ENSG00000158859  &  MAPK9  &  ENSG00000050748 \\
%ADAMTS6  &  ENSG00000049192  &  MARK1  &  ENSG00000116141 \\
%ADAMTSL1  &  ENSG00000178031  &  MASP2  &  ENSG00000009724 \\
%ADC  &  ENSG00000142920  &  MAX  &  ENSG00000125952 \\
%ADIPOR1  &  ENSG00000159346  &  MC2R  &  ENSG00000185231 \\
%ADORA3  &  ENSG00000121933  &  MC4R  &  ENSG00000166603 \\
%ADRA1A  &  ENSG00000120907  &  MCCC1  &  ENSG00000078070 \\
%AGAP1  &  ENSG00000157985  &  MCMBP  &  ENSG00000197771 \\
%AGAP4  &  ENSG00000188234  &  MDFI  &  ENSG00000112559 \\
%AGT  &  ENSG00000135744  &  MED29  &  ENSG00000063322 \\
%AGTR1  &  ENSG00000144891  &  MED30  &  ENSG00000164758 \\
%AHDC1  &  ENSG00000126705  &  MED9  &  ENSG00000141026 \\
%AJUBA  &  ENSG00000129474  &  METTL7A  &  ENSG00000185432 \\
%AK3  &  ENSG00000147853  &  MFSD3  &  ENSG00000167700 \\
%AKAP3  &  ENSG00000111254  &  MGST1  &  ENSG00000008394 \\
%AKAP9  &  ENSG00000127914  &  MIA2  &  ENSG00000150526 \\
%AKR1C2  &  ENSG00000151632  &  MIEF2  &  ENSG00000177427 \\
%AL109659.1  &  ENSG00000269709  &  MINPP1  &  ENSG00000107789 \\
%AL590822.1  &  ENSG00000203301  &  MITF  &  ENSG00000187098 \\
%ALDOC  &  ENSG00000109107  &  MKL1  &  ENSG00000196588 \\
%ALOX12  &  ENSG00000108839  &  MLYCD  &  ENSG00000103150 \\
%ALS2CR11  &  ENSG00000155754  &  MMP23B  &  ENSG00000189409 \\
%ALX3  &  ENSG00000156150  &  MMP24  &  ENSG00000125966 \\
%AMD1  &  ENSG00000123505  &  MMP3  &  ENSG00000149968 \\
%AMOT  &  ENSG00000126016  &  MMP7  &  ENSG00000137673 \\
%AMPH  &  ENSG00000078053  &  MPI  &  ENSG00000178802 \\
%ANKDD1A  &  ENSG00000166839  &  MPO  &  ENSG00000005381 \\
%ANKDD1B  &  ENSG00000189045  &  MPP4  &  ENSG00000082126 \\
%ANKFN1  &  ENSG00000153930  &  MPPED2  &  ENSG00000066382 \\
%ANKRD13C  &  ENSG00000118454  &  MPZL1  &  ENSG00000197965 \\
%ANKRD17  &  ENSG00000132466  &  MRAS  &  ENSG00000158186 \\
%ANKRD33  &  ENSG00000167612  &  MRGPRD  &  ENSG00000172938 \\
%ANKRD36  &  ENSG00000135976  &  MRGPRX4  &  ENSG00000179817 \\
%ANKRD40  &  ENSG00000154945  &  MRPL32  &  ENSG00000106591 \\
%ANXA5  &  ENSG00000164111  &  MSANTD4  &  ENSG00000170903 \\
%AP002884.2  &  ENSG00000268472  &  MSR1  &  ENSG00000038945 \\
%AP3D1  &  ENSG00000065000  &  MSRA  &  ENSG00000175806 \\
%APC2  &  ENSG00000115266  &  MT-CO1  &  ENSG00000198804 \\
%APMAP  &  ENSG00000101474  &  MT-ND4L  &  ENSG00000212907 \\
%APOB  &  ENSG00000084674  &  MT1H  &  ENSG00000205358 \\
%APOC3  &  ENSG00000110245  &  MT4  &  ENSG00000102891 \\
%AQP11  &  ENSG00000178301  &  MTBP  &  ENSG00000172167 \\
%AQP4  &  ENSG00000171885  &  MTERF  &  ENSG00000127989 \\
%ARHGAP1  &  ENSG00000175220  &  MTG2  &  ENSG00000101181 \\
%ARHGAP21  &  ENSG00000107863  &  MTOR  &  ENSG00000198793 \\
%ARHGAP22  &  ENSG00000128805  &  MUC2  &  ENSG00000198788 \\
%ARHGAP35  &  ENSG00000160007  &  MUC5B  &  ENSG00000117983 \\
%ARHGEF2  &  ENSG00000116584  &  MX1  &  ENSG00000157601 \\
%ARHGEF4  &  ENSG00000136002  &  MXD4  &  ENSG00000123933 \\
%ARL11  &  ENSG00000152213  &  MYH7B  &  ENSG00000078814 \\
%ARMCX3  &  ENSG00000102401  &  MYH9  &  ENSG00000100345 \\
%ARPC2  &  ENSG00000163466  &  MYL5  &  ENSG00000215375 \\
%ARSD  &  ENSG00000006756  &  MYLK3  &  ENSG00000140795 \\
%ART3  &  ENSG00000156219  &  MYO1C  &  ENSG00000197879 \\
%ASAP2  &  ENSG00000151693  &  MYO1E  &  ENSG00000157483 \\
%ASB17  &  ENSG00000154007  &  MYO1H  &  ENSG00000174527 \\
%ASRGL1  &  ENSG00000162174  &  N4BP1\_HUMAN  &  ENSG00000102921 \\
%ATAD3B  &  ENSG00000160072  &  N4BP2L1  &  ENSG00000139597 \\
%ATCAY  &  ENSG00000167654  &  NACAD  &  ENSG00000136274 \\
%ATOH8  &  ENSG00000168874  &  NAPG  &  ENSG00000134265 \\
%ATP1A3  &  ENSG00000105409  &  NAPSA  &  ENSG00000131400 \\
%ATP1B1  &  ENSG00000143153  &  NAT8L  &  ENSG00000185818 \\
%ATP2A3  &  ENSG00000074370  &  NCAM2  &  ENSG00000154654 \\
%ATP4B  &  ENSG00000186009  &  NDST3  &  ENSG00000164100 \\
%ATP5A1  &  ENSG00000152234  &  NDUFB10  &  ENSG00000140990 \\
%ATP5G1  &  ENSG00000159199  &  NDUFS7  &  ENSG00000115286 \\
%ATP5G2  &  ENSG00000135390  &  NEDD9  &  ENSG00000111859 \\
%ATP6V0A4  &  ENSG00000105929  &  NEK10  &  ENSG00000163491 \\
%ATP6V0E1  &  ENSG00000113732  &  NEK11  &  ENSG00000114670 \\
%ATP6V1G1  &  ENSG00000136888  &  NEK2  &  ENSG00000117650 \\
%ATP7A  &  ENSG00000165240  &  NEMF  &  ENSG00000165525 \\
%ATXN10  &  ENSG00000130638  &  NFE2L3  &  ENSG00000050344 \\
%ATXN3  &  ENSG00000066427  &  NFKBIA  &  ENSG00000100906 \\
%B3GNT4  &  ENSG00000176383  &  NFX1  &  ENSG00000086102 \\
%BAHCC1  &  ENSG00000171282  &  NFYB  &  ENSG00000120837 \\
%BAI1  &  ENSG00000181790  &  NHSL1  &  ENSG00000135540 \\
%BAI3  &  ENSG00000135298  &  NKX3-1  &  ENSG00000167034 \\
%BAK1  &  ENSG00000030110  &  NKX6-2  &  ENSG00000148826 \\
%BAP1  &  ENSG00000163930  &  NLRP8  &  ENSG00000179709 \\
%BARD1  &  ENSG00000138376  &  NMUR1  &  ENSG00000171596 \\
%BBX  &  ENSG00000114439  &  NOS3  &  ENSG00000164867 \\
%BCAT2  &  ENSG00000105552  &  NOSIP  &  ENSG00000142546 \\
%BCL2L10  &  ENSG00000137875  &  NOXA1  &  ENSG00000188747 \\
%BCL2L14  &  ENSG00000121380  &  NPVF  &  ENSG00000105954 \\
%BDH2  &  ENSG00000164039  &  NPY1R  &  ENSG00000164128 \\
%BEGAIN  &  ENSG00000183092  &  NP\_598407.1  &  ENSG00000091436 \\
%BHLHA15  &  ENSG00000180535  &  NP\_940879.1  &  ENSG00000189377 \\
%BHLHE22  &  ENSG00000180828  &  NP\_997392.1  &  ENSG00000196979 \\
%BLVRB  &  ENSG00000090013  &  NR2C2  &  ENSG00000177463 \\
%BMPR1B  &  ENSG00000138696  &  NR2E1  &  ENSG00000112333 \\
%BMPR2  &  ENSG00000204217  &  NR2F2  &  ENSG00000185551 \\
%BOLA3  &  ENSG00000163170  &  NR3C2  &  ENSG00000151623 \\
%BPIFB3  &  ENSG00000186190  &  NR4A2  &  ENSG00000153234 \\
%BRD4  &  ENSG00000141867  &  NRL  &  ENSG00000129535 \\
%BTBD11  &  ENSG00000151136  &  NSF  &  ENSG00000073969 \\
%BTF3L4  &  ENSG00000134717  &  NTMT1  &  ENSG00000148335 \\
%BZW1  &  ENSG00000082153  &  NTRK3  &  ENSG00000140538 \\
%C10orf71  &  ENSG00000177354  &  NTSR2  &  ENSG00000169006 \\
%C10orf99  &  ENSG00000188373  &  NUAK1  &  ENSG00000074590 \\
%C12orf39  &  ENSG00000134548  &  NUFIP1  &  ENSG00000083635 \\
%C12orf55  &  ENSG00000188596  &  NUP160  &  ENSG00000030066 \\
%C14orf159  &  ENSG00000133943  &  NUP98  &  ENSG00000110713 \\
%C14orf177  &  ENSG00000176605  &  NWD1  &  ENSG00000188039 \\
%C15orf40  &  ENSG00000169609  &  NXPE4  &  ENSG00000137634 \\
%C15orf43  &  ENSG00000167014  &  NYX  &  ENSG00000188937 \\
%C15orf52  &  ENSG00000188549  &  ODAM  &  ENSG00000109205 \\
%C15orf62  &  ENSG00000188277  &  ODF3  &  ENSG00000177947 \\
%C17orf85  &  ENSG00000074356  &  OGG1  &  ENSG00000114026 \\
%C19orf35  &  ENSG00000188305  &  OPLAH  &  ENSG00000178814 \\
%C1QTNF7  &  ENSG00000163145  &  OPRD1  &  ENSG00000116329 \\
%C1orf167  &  ENSG00000215910  &  OR10H2  &  ENSG00000171942 \\
%C2orf27A  &  ENSG00000197927  &  OR10K1  &  ENSG00000173285 \\
%C2orf49  &  ENSG00000135974  &  OR10S1  &  ENSG00000196248 \\
%C3AR1  &  ENSG00000171860  &  OR10Z1  &  ENSG00000198967 \\
%C3orf27  &  ENSG00000198685  &  OR13C3  &  ENSG00000204246 \\
%C3orf36  &  ENSG00000221972  &  OR13D1  &  ENSG00000179055 \\
%C4orf36  &  ENSG00000163633  &  OR13G1  &  ENSG00000197437 \\
%C5orf55  &  ENSG00000221990  &  OR13J1  &  ENSG00000168828 \\
%C7orf43  &  ENSG00000146826  &  OR1F1  &  ENSG00000168124 \\
%C8G  &  ENSG00000176919  &  OR2C1  &  ENSG00000168158 \\
%C8orf44  &  ENSG00000213865  &  OR2L8  &  ENSG00000196936 \\
%C8orf74  &  ENSG00000171060  &  OR2T1  &  ENSG00000175143 \\
%C9orf16  &  ENSG00000171159  &  OR4A5  &  ENSG00000221840 \\
%C9orf169  &  ENSG00000197191  &  OR4S2  &  ENSG00000174982 \\
%C9orf85  &  ENSG00000155621  &  OR51F1  &  ENSG00000188069 \\
%CA1  &  ENSG00000133742  &  OR5B3  &  ENSG00000172769 \\
%CABP4  &  ENSG00000175544  &  OR5D14  &  ENSG00000186113 \\
%CACNA1D  &  ENSG00000157388  &  OR7C2  &  ENSG00000127529 \\
%CACNA1H  &  ENSG00000196557  &  OR9A4  &  ENSG00000258083 \\
%CACNG3  &  ENSG00000006116  &  OSM  &  ENSG00000099985 \\
%CACNG8  &  ENSG00000142408  &  OSMR  &  ENSG00000145623 \\
%CAD  &  ENSG00000084774  &  OSR2  &  ENSG00000164920 \\
%CALCB  &  ENSG00000175868  &  P2RY2  &  ENSG00000175591 \\
%CALHM2  &  ENSG00000138172  &  P2RY8  &  ENSG00000182162 \\
%CALR  &  ENSG00000179218  &  PABPC4  &  ENSG00000090621 \\
%CAPN10  &  ENSG00000142330  &  PACSIN3  &  ENSG00000165912 \\
%CAPN2  &  ENSG00000162909  &  PAK3  &  ENSG00000077264 \\
%CAPN5  &  ENSG00000149260  &  PAK4  &  ENSG00000130669 \\
%CAPNS2  &  ENSG00000256812  &  PAMCI  &  ENSG00000198774 \\
%CARD18  &  ENSG00000255501  &  PAPOLA  &  ENSG00000090060 \\
%CARKD  &  ENSG00000213995  &  PAQR3  &  ENSG00000163291 \\
%CASP7  &  ENSG00000165806  &  PAQR9  &  ENSG00000188582 \\
%CASR  &  ENSG00000036828  &  PARP16  &  ENSG00000138617 \\
%CATX-2  &  ENSG00000267958  &  PARP9  &  ENSG00000138496 \\
%CBLL1  &  ENSG00000105879  &  PAX1  &  ENSG00000125813 \\
%CBR4  &  ENSG00000145439  &  PBX2  &  ENSG00000204304 \\
%CCBE1  &  ENSG00000183287  &  PBX3  &  ENSG00000167081 \\
%CCDC11  &  ENSG00000172361  &  PCDH10  &  ENSG00000138650 \\
%CCDC120  &  ENSG00000147144  &  PCDHB16  &  ENSG00000196963 \\
%CCDC134  &  ENSG00000100147  &  PCDHB2  &  ENSG00000112852 \\
%CCDC155  &  ENSG00000161609  &  PCK2  &  ENSG00000100889 \\
%CCDC36  &  ENSG00000173421  &  PCSK6  &  ENSG00000140479 \\
%CCDC57  &  ENSG00000176155  &  PCSK9  &  ENSG00000169174 \\
%CCDC94  &  ENSG00000105248  &  PDE6D  &  ENSG00000156973 \\
%CCER1  &  ENSG00000197651  &  PDE7A  &  ENSG00000205268 \\
%CCKAR  &  ENSG00000163394  &  PDE8A  &  ENSG00000073417 \\
%CCNA2  &  ENSG00000145386  &  PDHA1  &  ENSG00000131828 \\
%CCND3  &  ENSG00000112576  &  PDIA4  &  ENSG00000155660 \\
%CCNO  &  ENSG00000152669  &  PDP1  &  ENSG00000164951 \\
%CCNT2  &  ENSG00000082258  &  PDPK1  &  ENSG00000140992 \\
%CCR6  &  ENSG00000112486  &  PDSS2  &  ENSG00000164494 \\
%CCSER1  &  ENSG00000184305  &  PDZD4  &  ENSG00000067840 \\
%CCSER2  &  ENSG00000107771  &  PEX13  &  ENSG00000162928 \\
%CCT8  &  ENSG00000156261  &  PFKM  &  ENSG00000152556 \\
%CD164  &  ENSG00000135535  &  PGC  &  ENSG00000096088 \\
%CD22  &  ENSG00000012124  &  PGLYRP4  &  ENSG00000163218 \\
%CD276  &  ENSG00000103855  &  PHLDB2  &  ENSG00000144824 \\
%CD300LB  &  ENSG00000178789  &  PHLPP1  &  ENSG00000081913 \\
%CD83  &  ENSG00000112149  &  PHPT1  &  ENSG00000054148 \\
%CD8A  &  ENSG00000153563  &  PHYHIP  &  ENSG00000168490 \\
%CDA  &  ENSG00000158825  &  PI4K2B  &  ENSG00000038210 \\
%CDC23  &  ENSG00000094880  &  PIGC  &  ENSG00000135845 \\
%CDC42  &  ENSG00000070831  &  PIGK  &  ENSG00000142892 \\
%CDC42EP1  &  ENSG00000128283  &  PIGL  &  ENSG00000108474 \\
%CDH12  &  ENSG00000154162  &  PIK3C2A  &  ENSG00000011405 \\
%CDHR3  &  ENSG00000128536  &  PIK3R1  &  ENSG00000145675 \\
%CDK19  &  ENSG00000155111  &  PIK3R2  &  ENSG00000105647 \\
%CDK5  &  ENSG00000164885  &  PIM2  &  ENSG00000102096 \\
%CDK5R1  &  ENSG00000176749  &  PINK1  &  ENSG00000158828 \\
%CDK8  &  ENSG00000132964  &  PIP5K1A  &  ENSG00000143398 \\
%CDK9  &  ENSG00000136807  &  PIP5KL1  &  ENSG00000167103 \\
%CDKL3  &  ENSG00000006837  &  PITX3  &  ENSG00000107859 \\
%CDKN2C  &  ENSG00000123080  &  PKIA  &  ENSG00000171033 \\
%CDS2  &  ENSG00000101290  &  PKN2  &  ENSG00000065243 \\
%CELA2B  &  ENSG00000215704  &  PLA2G15  &  ENSG00000103066 \\
%CELF4  &  ENSG00000101489  &  PLA2G2A  &  ENSG00000188257 \\
%CEP76  &  ENSG00000101624  &  PLA2G2C  &  ENSG00000187980 \\
%CEPT1  &  ENSG00000134255  &  PLA2G3  &  ENSG00000100078 \\
%CETN1  &  ENSG00000177143  &  PLAUR  &  ENSG00000011422 \\
%CFDP1  &  ENSG00000153774  &  PLD4  &  ENSG00000166428 \\
%CFLAR  &  ENSG00000003402  &  PLEK  &  ENSG00000115956 \\
%CGNL1  &  ENSG00000128849  &  PLEKHA4  &  ENSG00000105559 \\
%CHD7  &  ENSG00000171316  &  PLEKHH3  &  ENSG00000068137 \\
%CHDH  &  ENSG00000016391  &  PLEKHM2  &  ENSG00000116786 \\
%CHMP1A  &  ENSG00000131165  &  PLOD2  &  ENSG00000152952 \\
%CHMP2B  &  ENSG00000083937  &  PLXND1  &  ENSG00000004399 \\
%CHMP4C  &  ENSG00000164695  &  PNPLA8  &  ENSG00000135241 \\
%CHRDL1  &  ENSG00000101938  &  POLA2  &  ENSG00000014138 \\
%CHRM1  &  ENSG00000168539  &  POLD3  &  ENSG00000077514 \\
%CHRM2  &  ENSG00000181072  &  POLD4  &  ENSG00000175482 \\
%CHRM3  &  ENSG00000133019  &  POLE  &  ENSG00000177084 \\
%CHRNA4  &  ENSG00000101204  &  POLE2  &  ENSG00000100479 \\
%CHRNB4  &  ENSG00000117971  &  POLE4  &  ENSG00000115350 \\
%CHRNE  &  ENSG00000108556  &  POLL  &  ENSG00000166169 \\
%CHRNG  &  ENSG00000196811  &  POLR1B  &  ENSG00000125630 \\
%CHST6  &  ENSG00000183196  &  POLR2B  &  ENSG00000047315 \\
%CLEC12A  &  ENSG00000172322  &  POLRMT  &  ENSG00000099821 \\
%CLGN  &  ENSG00000153132  &  PON2  &  ENSG00000105854 \\
%CLK3  &  ENSG00000179335  &  POR  &  ENSG00000127948 \\
%CLRN1  &  ENSG00000163646  &  PPFIBP2  &  ENSG00000166387 \\
%CLRN2  &  ENSG00000113448  &  PPM1A  &  ENSG00000100614 \\
%CLVS2  &  ENSG00000146352  &  PPP1R3C  &  ENSG00000119938 \\
%CMC1  &  ENSG00000187118  &  PPP4R2  &  ENSG00000163605 \\
%CMTM2  &  ENSG00000140932  &  PRDM10  &  ENSG00000170325 \\
%CNKSR3  &  ENSG00000153721  &  PRKACG  &  ENSG00000165059 \\
%CNOT7  &  ENSG00000198791  &  PRKD1  &  ENSG00000184304 \\
%CNP  &  ENSG00000173786  &  PRLHR  &  ENSG00000119973 \\
%CNR2  &  ENSG00000188822  &  PRMT3  &  ENSG00000185238 \\
%COG2  &  ENSG00000135775  &  PROKR1  &  ENSG00000169618 \\
%COMMD2  &  ENSG00000114744  &  PRPF4  &  ENSG00000136875 \\
%COMTD1  &  ENSG00000165644  &  PRPS1  &  ENSG00000147224 \\
%COPB2  &  ENSG00000184432  &  PRPS2  &  ENSG00000101911 \\
%COQ3  &  ENSG00000132423  &  PRRC2C  &  ENSG00000117523 \\
%COX10  &  ENSG00000006695  &  PRTFDC1  &  ENSG00000099256 \\
%COX14  &  ENSG00000178449  &  PSMB1  &  ENSG00000008018 \\
%COX4I2  &  ENSG00000131055  &  PSMB2  &  ENSG00000126067 \\
%COX7C  &  ENSG00000127184  &  PTGDR2  &  ENSG00000183134 \\
%CPB1  &  ENSG00000153002  &  PTGIR  &  ENSG00000160013 \\
%CREB1  &  ENSG00000118260  &  PTH1R  &  ENSG00000160801 \\
%CRISP2  &  ENSG00000124490  &  PTPDC1  &  ENSG00000158079 \\
%CRNKL1  &  ENSG00000101343  &  PTPLA  &  ENSG00000165996 \\
%CRYM  &  ENSG00000103316  &  PTPLAD1  &  ENSG00000074696 \\
%CSAD  &  ENSG00000139631  &  PTPN4  &  ENSG00000088179 \\
%CSGALNACT1  &  ENSG00000147408  &  PTPN5  &  ENSG00000110786 \\
%CSNK1D  &  ENSG00000141551  &  PTPN7  &  ENSG00000143851 \\
%CSNK1E  &  ENSG00000213923  &  PTPRA  &  ENSG00000132670 \\
%CST2  &  ENSG00000170369  &  PXDNL  &  ENSG00000147485 \\
%CST5  &  ENSG00000170367  &  PYCR1  &  ENSG00000183010 \\
%CTAG2  &  ENSG00000126890  &  PYGM  &  ENSG00000068976 \\
%CTH  &  ENSG00000116761  &  Q8WUC5\_HUMAN  &  ENSG00000168569 \\
%CTSE  &  ENSG00000196188  &  Q9H521\_HUMAN  &  ENSG00000121388 \\
%CTU2  &  ENSG00000174177  &  QRSL1  &  ENSG00000130348 \\
%CUL4A  &  ENSG00000139842  &  QTRTD1  &  ENSG00000151576 \\
%CUL7  &  ENSG00000044090  &  RAB20  &  ENSG00000139832 \\
%CUX2  &  ENSG00000111249  &  RAB5C  &  ENSG00000108774 \\
%CWF19L2  &  ENSG00000152404  &  RALGPS1  &  ENSG00000136828 \\
%CXCR3  &  ENSG00000186810  &  RANBP3  &  ENSG00000031823 \\
%CXCR6  &  ENSG00000172215  &  RANGAP1  &  ENSG00000100401 \\
%CXorf36  &  ENSG00000147113  &  RAP1B  &  ENSG00000127314 \\
%CXorf56  &  ENSG00000018610  &  RASAL1  &  ENSG00000111344 \\
%CXorf64  &  ENSG00000183631  &  RASSF1  &  ENSG00000068028 \\
%CYB5R2  &  ENSG00000166394  &  RBKS  &  ENSG00000171174 \\
%CYHR1  &  ENSG00000187954  &  RBM15B  &  ENSG00000179837 \\
%CYP2F1  &  ENSG00000197446  &  RBM33  &  ENSG00000184863 \\
%CYSLTR2  &  ENSG00000152207  &  RBM34  &  ENSG00000188739 \\
%DAGLB  &  ENSG00000164535  &  RBM47  &  ENSG00000163694 \\
%DAP  &  ENSG00000112977  &  RBMS1  &  ENSG00000153250 \\
%DAPK3  &  ENSG00000167657  &  RC3H1  &  ENSG00000135870 \\
%DAZL  &  ENSG00000092345  &  RDH13  &  ENSG00000160439 \\
%DBN1  &  ENSG00000113758  &  RELA  &  ENSG00000173039 \\
%DCDC1  &  ENSG00000170959  &  RFWD3  &  ENSG00000168411 \\
%DCLRE1C  &  ENSG00000152457  &  RGL1  &  ENSG00000143344 \\
%DCST1  &  ENSG00000163357  &  RGMA  &  ENSG00000182175 \\
%DCT  &  ENSG00000080166  &  RGS1  &  ENSG00000090104 \\
%DDC  &  ENSG00000132437  &  RGS10  &  ENSG00000148908 \\
%DDOST  &  ENSG00000244038  &  RHOA  &  ENSG00000067560 \\
%DDR1  &  ENSG00000204580  &  RIMS1  &  ENSG00000079841 \\
%DEFA5  &  ENSG00000164816  &  RIPK2  &  ENSG00000104312 \\
%DEFB129  &  ENSG00000125903  &  RNASE2  &  ENSG00000169385 \\
%DEGS1  &  ENSG00000143753  &  RNASEH2C  &  ENSG00000172922 \\
%DEPDC4  &  ENSG00000166153  &  RNF113A  &  ENSG00000125352 \\
%DESI2  &  ENSG00000121644  &  RNF19A  &  ENSG00000034677 \\
%DEXI  &  ENSG00000182108  &  ROBO4  &  ENSG00000154133 \\
%DFFA  &  ENSG00000160049  &  ROCK2  &  ENSG00000134318 \\
%DGCR2  &  ENSG00000070413  &  ROR2  &  ENSG00000169071 \\
%DGKG  &  ENSG00000058866  &  RP11-321C1.1  &  ENSG00000269964 \\
%DHCR24  &  ENSG00000116133  &  RPL13A  &  ENSG00000142541 \\
%DHFRL1  &  ENSG00000178700  &  RPL36AL  &  ENSG00000165502 \\
%DHRS2  &  ENSG00000100867  &  RPL41  &  ENSG00000229117 \\
%DHX34  &  ENSG00000134815  &  RPL7L1  &  ENSG00000146223 \\
%DIAPH1  &  ENSG00000131504  &  RPL9  &  ENSG00000163682 \\
%DLC1  &  ENSG00000164741  &  RPS18  &  ENSG00000231500 \\
%DLG4  &  ENSG00000132535  &  RPS28  &  ENSG00000233927 \\
%DLK1  &  ENSG00000185559  &  RPS9  &  ENSG00000170889 \\
%DMD  &  ENSG00000198947  &  RPUSD4  &  ENSG00000165526 \\
%DNAH5  &  ENSG00000039139  &  RRNAD1  &  ENSG00000143303 \\
%DNAI2  &  ENSG00000171595  &  RRP9  &  ENSG00000114767 \\
%DNAJC11  &  ENSG00000007923  &  RSPH6A  &  ENSG00000104941 \\
%DNMT3B  &  ENSG00000088305  &  RSU1  &  ENSG00000148484 \\
%DOCK11  &  ENSG00000147251  &  RTP5  &  ENSG00000188011 \\
%DOCK7  &  ENSG00000116641  &  S100A7A  &  ENSG00000184330 \\
%DOK3  &  ENSG00000146094  &  S100PBP  &  ENSG00000116497 \\
%DPEP1  &  ENSG00000015413  &  S100Z  &  ENSG00000171643 \\
%DSTYK  &  ENSG00000133059  &  SACS  &  ENSG00000151835 \\
%DTYMK  &  ENSG00000168393  &  SARDH  &  ENSG00000123453 \\
%DUSP13  &  ENSG00000079393  &  SBK3  &  ENSG00000231274 \\
%DYNC2H1  &  ENSG00000187240  &  SCARA3  &  ENSG00000168077 \\
%DYNLRB2  &  ENSG00000168589  &  SCARF2  &  ENSG00000244486 \\
%DYRK1B  &  ENSG00000105204  &  SCG3  &  ENSG00000104112 \\
%DYRK3  &  ENSG00000143479  &  SCGB2A2  &  ENSG00000110484 \\
%EBPL  &  ENSG00000123179  &  SCNN1G  &  ENSG00000166828 \\
%EEF1E1  &  ENSG00000124802  &  SCRN3  &  ENSG00000144306 \\
%EFCAB4B  &  ENSG00000130038  &  SCYL1  &  ENSG00000142186 \\
%EFCAB6  &  ENSG00000186976  &  SDHA  &  ENSG00000073578 \\
%EGLN1  &  ENSG00000135766  &  SDHC  &  ENSG00000143252 \\
%EID2B  &  ENSG00000176401  &  SDR39U1  &  ENSG00000100445 \\
%EIF2AK2  &  ENSG00000055332  &  SEC23A  &  ENSG00000100934 \\
%EIF2S3  &  ENSG00000130741  &  SECTM1  &  ENSG00000141574 \\
%EIF4A2  &  ENSG00000156976  &  SELK\_HUMAN  &  ENSG00000113811 \\
%ELL  &  ENSG00000105656  &  SEMA3D  &  ENSG00000153993 \\
%EMILIN2  &  ENSG00000132205  &  SEMA3G  &  ENSG00000010319 \\
%EML1  &  ENSG00000066629  &  SEMA4A  &  ENSG00000196189 \\
%ENAH  &  ENSG00000154380  &  SENP1  &  ENSG00000079387 \\
%ENDOG  &  ENSG00000167136  &  SENP2  &  ENSG00000163904 \\
%ENDOV  &  ENSG00000173818  &  SEPN1  &  ENSG00000162430 \\
%EPGN  &  ENSG00000182585  &  SEPT12  &  ENSG00000140623 \\
%EPHA2  &  ENSG00000142627  &  SERPINI1  &  ENSG00000163536 \\
%EPHA8  &  ENSG00000070886  &  SETD9  &  ENSG00000155542 \\
%EPN3  &  ENSG00000049283  &  SF3B4  &  ENSG00000143368 \\
%ERBB2  &  ENSG00000141736  &  SFTPA2  &  ENSG00000185303 \\
%ERBB3  &  ENSG00000065361  &  SGK1  &  ENSG00000118515 \\
%ERCC8  &  ENSG00000049167  &  SH2D4B  &  ENSG00000178217 \\
%ESD  &  ENSG00000139684  &  SHISA5  &  ENSG00000164054 \\
%ESX1  &  ENSG00000123576  &  SHMT2  &  ENSG00000182199 \\
%ESYT1  &  ENSG00000139641  &  SKIV2L  &  ENSG00000204351 \\
%ETFB  &  ENSG00000105379  &  SLC12A6  &  ENSG00000140199 \\
%EXOC1  &  ENSG00000090989  &  SLC16A3  &  ENSG00000141526 \\
%FAM110C  &  ENSG00000184731  &  SLC16A5  &  ENSG00000170190 \\
%FAM174A  &  ENSG00000174132  &  SLC20A1  &  ENSG00000144136 \\
%FAM188A  &  ENSG00000148481  &  SLC22A24  &  ENSG00000197658 \\
%FAM192A  &  ENSG00000172775  &  SLC23A1  &  ENSG00000170482 \\
%FAM207A  &  ENSG00000160256  &  SLC25A34  &  ENSG00000162461 \\
%FAM213A  &  ENSG00000122378  &  SLC25A35  &  ENSG00000125434 \\
%FAM3B  &  ENSG00000183844  &  SLC25A37  &  ENSG00000147454 \\
%FAM47A  &  ENSG00000185448  &  SLC25A4  &  ENSG00000151729 \\
%FAM49B  &  ENSG00000153310  &  SLC26A5  &  ENSG00000170615 \\
%FAM84B  &  ENSG00000168672  &  SLC26A8  &  ENSG00000112053 \\
%FAS  &  ENSG00000026103  &  SLC29A1  &  ENSG00000112759 \\
%FASTKD3  &  ENSG00000124279  &  SLC30A1  &  ENSG00000170385 \\
%FBXL14  &  ENSG00000171823  &  SLC33A1  &  ENSG00000169359 \\
%FBXL18  &  ENSG00000155034  &  SLC35E2  &  ENSG00000215790 \\
%FBXO36  &  ENSG00000153832  &  SLC36A1  &  ENSG00000123643 \\
%FCRL3  &  ENSG00000160856  &  SLC38A5  &  ENSG00000017483 \\
%FCRLB  &  ENSG00000162746  &  SLC38A9  &  ENSG00000177058 \\
%FDPS  &  ENSG00000160752  &  SLC5A10  &  ENSG00000154025 \\
%FES  &  ENSG00000182511  &  SLC8A2  &  ENSG00000118160 \\
%FETUB  &  ENSG00000090512  &  SLC9A1  &  ENSG00000090020 \\
%FGF10  &  ENSG00000070193  &  SLC9A3  &  ENSG00000066230 \\
%FGF2  &  ENSG00000138685  &  SMARCA1  &  ENSG00000102038 \\
%FIGLA  &  ENSG00000183733  &  SMDT1  &  ENSG00000183172 \\
%FKBP10  &  ENSG00000141756  &  SMG1  &  ENSG00000157106 \\
%FKBP11  &  ENSG00000134285  &  SMIM14  &  ENSG00000163683 \\
%FKBP14  &  ENSG00000106080  &  SMNDC1  &  ENSG00000119953 \\
%FKBP2  &  ENSG00000173486  &  SMOC1  &  ENSG00000198732 \\
%FLCN  &  ENSG00000154803  &  SNRK  &  ENSG00000163788 \\
%FLT3LG  &  ENSG00000090554  &  SNTA1  &  ENSG00000101400 \\
%FMO4  &  ENSG00000076258  &  SNTB2  &  ENSG00000168807 \\
%FNDC1  &  ENSG00000164694  &  SNUPN  &  ENSG00000169371 \\
%FOLR2  &  ENSG00000165457  &  SNX17  &  ENSG00000115234 \\
%FOXRED1  &  ENSG00000110074  &  SORCS3  &  ENSG00000156395 \\
%FRMD1  &  ENSG00000153303  &  SOS2  &  ENSG00000100485 \\
%FRMD3  &  ENSG00000172159  &  SOWAHA  &  ENSG00000198944 \\
%FTL  &  ENSG00000087086  &  SOX11  &  ENSG00000176887 \\
%FUT4  &  ENSG00000196371  &  SP140L  &  ENSG00000185404 \\
%FUT9  &  ENSG00000172461  &  SPANXN2  &  ENSG00000203924 \\
%FZD10  &  ENSG00000111432  &  SPATC1  &  ENSG00000186583 \\
%G3BP1  &  ENSG00000145907  &  SPCS1  &  ENSG00000114902 \\
%G6PC  &  ENSG00000131482  &  SPHK2  &  ENSG00000063176 \\
%GAB1  &  ENSG00000109458  &  SPRED1  &  ENSG00000166068 \\
%GABARAPL3  &  ENSG00000238244  &  SPSB2  &  ENSG00000111671 \\
%GABBR1  &  ENSG00000204681  &  SPTAN1  &  ENSG00000197694 \\
%GABRG1  &  ENSG00000163285  &  SRGAP1  &  ENSG00000196935 \\
%GABRP  &  ENSG00000094755  &  SRP9  &  ENSG00000143742 \\
%GADD45GIP1  &  ENSG00000179271  &  SRPK2  &  ENSG00000135250 \\
%GAL3ST1  &  ENSG00000128242  &  SRSF3  &  ENSG00000112081 \\
%GAL3ST4  &  ENSG00000197093  &  SSTR2  &  ENSG00000180616 \\
%GALNS  &  ENSG00000141012  &  STAC2  &  ENSG00000141750 \\
%GAPVD1  &  ENSG00000165219  &  STAG3  &  ENSG00000066923 \\
%GATA2  &  ENSG00000179348  &  STARD9  &  ENSG00000159433 \\
%GATM  &  ENSG00000171766  &  STEAP1  &  ENSG00000164647 \\
%GBA2  &  ENSG00000070610  &  STH  &  ENSG00000256762 \\
%GCC1  &  ENSG00000179562  &  STK17A  &  ENSG00000164543 \\
%GCC2  &  ENSG00000135968  &  STK17B  &  ENSG00000081320 \\
%GCNT4  &  ENSG00000176928  &  STK31  &  ENSG00000196335 \\
%GEMIN2  &  ENSG00000092208  &  STOM  &  ENSG00000148175 \\
%GFER  &  ENSG00000127554  &  STOX2  &  ENSG00000173320 \\
%GFRA1  &  ENSG00000151892  &  STRA6  &  ENSG00000137868 \\
%GHDC  &  ENSG00000167925  &  STRADA  &  ENSG00000266173 \\
%GIPR  &  ENSG00000010310  &  STX5  &  ENSG00000162236 \\
%GJA4  &  ENSG00000187513  &  SUCLG2  &  ENSG00000172340 \\
%GJB6  &  ENSG00000121742  &  SULT1E1  &  ENSG00000109193 \\
%GJC2  &  ENSG00000198835  &  SYCE1  &  ENSG00000171772 \\
%GLB1L3  &  ENSG00000166105  &  SYN3  &  ENSG00000185666 \\
%GLIPR1  &  ENSG00000139278  &  TAAR8  &  ENSG00000146385 \\
%GLOD4  &  ENSG00000167699  &  TAB2  &  ENSG00000055208 \\
%GLP1R  &  ENSG00000112164  &  TAB3  &  ENSG00000157625 \\
%GLRA1  &  ENSG00000145888  &  TACC1  &  ENSG00000147526 \\
%GLS  &  ENSG00000115419  &  TACR2  &  ENSG00000075073 \\
%GLTPD1  &  ENSG00000224051  &  TAF1  &  ENSG00000147133 \\
%GLTSCR1L  &  ENSG00000112624  &  TAF15  &  ENSG00000172660 \\
%GMDS  &  ENSG00000112699  &  TAF2  &  ENSG00000064313 \\
%GNA12  &  ENSG00000146535  &  TAOK3  &  ENSG00000135090 \\
%GNAI3  &  ENSG00000065135  &  TAS2R10  &  ENSG00000121318 \\
%GNE  &  ENSG00000159921  &  TAS2R9  &  ENSG00000121381 \\
%GNG4  &  ENSG00000168243  &  TBCB  &  ENSG00000105254 \\
%GNG7  &  ENSG00000176533  &  TBL3  &  ENSG00000183751 \\
%GPBP1L1  &  ENSG00000159592  &  TCEB1  &  ENSG00000154582 \\
%GPC6  &  ENSG00000183098  &  TCERG1  &  ENSG00000113649 \\
%GPM6A  &  ENSG00000150625  &  TCIRG1  &  ENSG00000110719 \\
%GPR128  &  ENSG00000144820  &  TCP1  &  ENSG00000120438 \\
%GPR146  &  ENSG00000164849  &  TDP1  &  ENSG00000042088 \\
%GPR15  &  ENSG00000154165  &  TDRD10  &  ENSG00000163239 \\
%GPR160  &  ENSG00000173890  &  TERF2IP  &  ENSG00000166848 \\
%GPR161  &  ENSG00000143147  &  TESK2  &  ENSG00000070759 \\
%GPR171  &  ENSG00000174946  &  TESPA1  &  ENSG00000135426 \\
%GPR174  &  ENSG00000147138  &  TEX26  &  ENSG00000175664 \\
%GPR179  &  ENSG00000188888  &  TEX264  &  ENSG00000164081 \\
%GPR18  &  ENSG00000125245  &  TEX38  &  ENSG00000186118 \\
%GPR180  &  ENSG00000152749  &  TFAP2C  &  ENSG00000087510 \\
%GPR26  &  ENSG00000154478  &  TFAP2D  &  ENSG00000008197 \\
%GPR27  &  ENSG00000170837  &  TGIF1  &  ENSG00000177426 \\
%GPR87  &  ENSG00000138271  &  THEG  &  ENSG00000105549 \\
%GPRC5B  &  ENSG00000167191  &  TIGD6  &  ENSG00000164296 \\
%GPSM1  &  ENSG00000160360  &  TIMM17A  &  ENSG00000134375 \\
%GPX4  &  ENSG00000167468  &  TINAG  &  ENSG00000137251 \\
%GRAPL  &  ENSG00000189152  &  TIRAP  &  ENSG00000150455 \\
%GRB14  &  ENSG00000115290  &  TK2  &  ENSG00000166548 \\
%GRHPR  &  ENSG00000137106  &  TLDC1  &  ENSG00000140950 \\
%GRIK1  &  ENSG00000171189  &  TLE3  &  ENSG00000140332 \\
%GRIN2A  &  ENSG00000183454  &  TLL2  &  ENSG00000095587 \\
%GRK5  &  ENSG00000198873  &  TLR2  &  ENSG00000137462 \\
%GRM1  &  ENSG00000152822  &  TLR8  &  ENSG00000101916 \\
%GRM4  &  ENSG00000124493  &  TMCC2  &  ENSG00000133069 \\
%GRPR  &  ENSG00000126010  &  TMEM106B  &  ENSG00000106460 \\
%GSS  &  ENSG00000100983  &  TMEM107  &  ENSG00000179029 \\
%GSTM5  &  ENSG00000134201  &  TMEM132B  &  ENSG00000139364 \\
%GTSF1L  &  ENSG00000124196  &  TMEM16F  &  ENSG00000177119 \\
%GUCA1B  &  ENSG00000112599  &  TMEM214  &  ENSG00000119777 \\
%GUCA1C  &  ENSG00000138472  &  TMEM230  &  ENSG00000089063 \\
%GUCY2C  &  ENSG00000070019  &  TMEM80  &  ENSG00000177042 \\
%GUCY2D  &  ENSG00000132518  &  TMEM82  &  ENSG00000162460 \\
%HAL  &  ENSG00000084110  &  TMEM89  &  ENSG00000183396 \\
%HARS  &  ENSG00000170445  &  TMEM8C  &  ENSG00000187616 \\
%HCN1  &  ENSG00000164588  &  TMPRSS11A  &  ENSG00000187054 \\
%HCN4  &  ENSG00000138622  &  TMPRSS13  &  ENSG00000137747 \\
%HDAC11  &  ENSG00000163517  &  TNFRSF10B  &  ENSG00000120889 \\
%HDAC6  &  ENSG00000094631  &  TNFSF9  &  ENSG00000125657 \\
%HDAC9  &  ENSG00000048052  &  TNNC2  &  ENSG00000101470 \\
%HECTD1  &  ENSG00000092148  &  TNS3  &  ENSG00000136205 \\
%HES2  &  ENSG00000069812  &  TOMM20L  &  ENSG00000196860 \\
%HGD  &  ENSG00000113924  &  TOPORS  &  ENSG00000197579 \\
%HIST1H2BH  &  ENSG00000197459  &  TOR4A  &  ENSG00000198113 \\
%HIST1H2BK  &  ENSG00000197903  &  TOX2  &  ENSG00000124191 \\
%HK2  &  ENSG00000159399  &  TPM1  &  ENSG00000140416 \\
%HLA-DOB  &  ENSG00000241106  &  TPM2  &  ENSG00000198467 \\
%HLA-DRB1  &  ENSG00000196126  &  TPM3  &  ENSG00000143549 \\
%HLCS  &  ENSG00000159267  &  TPST1  &  ENSG00000169902 \\
%HM13  &  ENSG00000101294  &  TRAF3  &  ENSG00000131323 \\
%HMBS  &  ENSG00000256269  &  TRIB3  &  ENSG00000101255 \\
%HMGN2  &  ENSG00000198830  &  TRIM2  &  ENSG00000109654 \\
%HMX2  &  ENSG00000188816  &  TRIM29  &  ENSG00000137699 \\
%HNRNPD  &  ENSG00000138668  &  TRIM47  &  ENSG00000132481 \\
%HOXA4  &  ENSG00000197576  &  TRO  &  ENSG00000067445 \\
%HOXC10  &  ENSG00000180818  &  TRPM7  &  ENSG00000092439 \\
%HPD  &  ENSG00000158104  &  TRUB1  &  ENSG00000165832 \\
%HPGDS  &  ENSG00000163106  &  TSC2  &  ENSG00000103197 \\
%HRH3  &  ENSG00000101180  &  TSHZ3  &  ENSG00000121297 \\
%HRK  &  ENSG00000135116  &  TSKS  &  ENSG00000126467 \\
%HRNR  &  ENSG00000197915  &  TSNAX  &  ENSG00000116918 \\
%HSD11B1L  &  ENSG00000167733  &  TSPAN3  &  ENSG00000140391 \\
%HSD17B4  &  ENSG00000133835  &  TSTA3  &  ENSG00000104522 \\
%HSD17B8  &  ENSG00000204228  &  TTC30B  &  ENSG00000196659 \\
%HTR1B  &  ENSG00000135312  &  TTF1  &  ENSG00000125482 \\
%HTR1E  &  ENSG00000168830  &  TTN  &  ENSG00000155657 \\
%HTR3A  &  ENSG00000166736  &  TUBA8  &  ENSG00000183785 \\
%HTR4  &  ENSG00000164270  &  TUBB4A  &  ENSG00000104833 \\
%HTR7  &  ENSG00000148680  &  TUFT1  &  ENSG00000143367 \\
%HUG1  &  ENSG00000269259  &  TULP2  &  ENSG00000104804 \\
%HUNK  &  ENSG00000142149  &  TUSC1  &  ENSG00000198680 \\
%HYAL4  &  ENSG00000106302  &  TWF1  &  ENSG00000151239 \\
%HYPK  &  ENSG00000242028  &  TYR  &  ENSG00000077498 \\
%IARS2  &  ENSG00000067704  &  UAP1  &  ENSG00000117143 \\
%IDH3A  &  ENSG00000166411  &  UBA2  &  ENSG00000126261 \\
%IDH3G  &  ENSG00000067829  &  UBBP4  &  ENSG00000263563 \\
%IDI1  &  ENSG00000067064  &  UBE2G1  &  ENSG00000132388 \\
%IDI2  &  ENSG00000148377  &  UBFD1  &  ENSG00000103353 \\
%IER3  &  ENSG00000137331  &  UBL3  &  ENSG00000122042 \\
%IFNA4  &  ENSG00000236637  &  UBQLN4  &  ENSG00000160803 \\
%IFNA5  &  ENSG00000147873  &  UBR2  &  ENSG00000024048 \\
%IFRD1  &  ENSG00000006652  &  UGT2B15  &  ENSG00000196620 \\
%IGF1R  &  ENSG00000140443  &  UGT3A1  &  ENSG00000145626 \\
%IGFBP1  &  ENSG00000146678  &  UMPS  &  ENSG00000114491 \\
%IGSF11  &  ENSG00000144847  &  UNG  &  ENSG00000076248 \\
%IL1R2  &  ENSG00000115590  &  UPK3A  &  ENSG00000100373 \\
%IL22RA2  &  ENSG00000164485  &  UROD  &  ENSG00000126088 \\
%IL24  &  ENSG00000162892  &  UROS  &  ENSG00000188690 \\
%IL27RA  &  ENSG00000104998  &  USP1  &  ENSG00000162607 \\
%IL3RA  &  ENSG00000185291  &  USP15  &  ENSG00000135655 \\
%IL4  &  ENSG00000113520  &  USP29  &  ENSG00000131864 \\
%IL6R  &  ENSG00000160712  &  USP3  &  ENSG00000140455 \\
%ILVBL  &  ENSG00000105135  &  USP34  &  ENSG00000115464 \\
%IMP3  &  ENSG00000177971  &  USP38  &  ENSG00000170185 \\
%INTS7  &  ENSG00000143493  &  USP4  &  ENSG00000114316 \\
%IQCH  &  ENSG00000103599  &  USP6NL  &  ENSG00000148429 \\
%IQUB  &  ENSG00000164675  &  USP7  &  ENSG00000187555 \\
%IRAK1  &  ENSG00000184216  &  UVSSA  &  ENSG00000163945 \\
%IRAK2  &  ENSG00000134070  &  VARS  &  ENSG00000204394 \\
%ITGAE  &  ENSG00000083457  &  VAV2  &  ENSG00000160293 \\
%ITGB5  &  ENSG00000082781  &  VCL  &  ENSG00000035403 \\
%ITIH6  &  ENSG00000102313  &  VIPR1  &  ENSG00000114812 \\
%IZUMO4  &  ENSG00000099840  &  VN1R4  &  ENSG00000228567 \\
%JUN  &  ENSG00000177606  &  VPS28  &  ENSG00000160948 \\
%JUNB  &  ENSG00000171223  &  VPS51  &  ENSG00000149823 \\
%KCNA3  &  ENSG00000177272  &  VPS53  &  ENSG00000141252 \\
%KCNA6  &  ENSG00000151079  &  VPS72  &  ENSG00000163159 \\
%KCND2  &  ENSG00000184408  &  VPS9D1  &  ENSG00000075399 \\
%KCND3  &  ENSG00000171385  &  VSIG1  &  ENSG00000101842 \\
%KCNH1  &  ENSG00000143473  &  VSIG4  &  ENSG00000155659 \\
%KCNH4  &  ENSG00000089558  &  VSTM4  &  ENSG00000165633 \\
%KCNH5  &  ENSG00000140015  &  VTI1A  &  ENSG00000151532 \\
%KCNMA1  &  ENSG00000156113  &  WASF1  &  ENSG00000112290 \\
%KCNN3  &  ENSG00000143603  &  WBSCR28  &  ENSG00000175877 \\
%KDM2B  &  ENSG00000089094  &  WDR37  &  ENSG00000047056 \\
%KHDC1  &  ENSG00000135314  &  WDR61  &  ENSG00000140395 \\
%KHDRBS2  &  ENSG00000112232  &  WDR63  &  ENSG00000162643 \\
%KIAA1147  &  ENSG00000257093  &  WDR77  &  ENSG00000116455 \\
%KIF18B  &  ENSG00000186185  &  WFDC12  &  ENSG00000168703 \\
%KIF21B  &  ENSG00000116852  &  WNT11  &  ENSG00000085741 \\
%KIF4B  &  ENSG00000226650  &  WNT4  &  ENSG00000162552 \\
%KIF5A  &  ENSG00000155980  &  WSCD2  &  ENSG00000075035 \\
%KIFAP3  &  ENSG00000075945  &  XAGE5  &  ENSG00000171405 \\
%KLHL14  &  ENSG00000197705  &  XYLT2  &  ENSG00000015532 \\
%KLRG1  &  ENSG00000139187  &  YIPF1  &  ENSG00000058799 \\
%KMT2A  &  ENSG00000118058  &  YWHAZ  &  ENSG00000164924 \\
%KNSTRN  &  ENSG00000128944  &  ZADH1  &  ENSG00000140043 \\
%KPNA1  &  ENSG00000114030  &  ZBP1  &  ENSG00000124256 \\
%KRR1  &  ENSG00000111615  &  ZBTB44  &  ENSG00000196323 \\
%KRT14  &  ENSG00000186847  &  ZC3H12C  &  ENSG00000149289 \\
%KRT31  &  ENSG00000094796  &  ZC3H6  &  ENSG00000188177 \\
%KRT39  &  ENSG00000196859  &  ZCCHC10  &  ENSG00000155329 \\
%KRT6C  &  ENSG00000170465  &  ZCCHC7  &  ENSG00000147905 \\
%KRT71  &  ENSG00000139648  &  ZDHHC19  &  ENSG00000163958 \\
%KRTAP17-1  &  ENSG00000186860  &  ZFHX2  &  ENSG00000136367 \\
%KRTAP4-5  &  ENSG00000198271  &  ZG16  &  ENSG00000174992 \\
%KRTAP9-3  &  ENSG00000204873  &  ZMAT3  &  ENSG00000172667 \\
%LARGE  &  ENSG00000133424  &  ZMYND8  &  ENSG00000101040 \\
%LARS2  &  ENSG00000011376  &  ZNF169  &  ENSG00000175787 \\
%LATS2  &  ENSG00000150457  &  ZNF195  &  ENSG00000005801 \\
%LCAT  &  ENSG00000213398  &  ZNF202  &  ENSG00000166261 \\
%LCE1D  &  ENSG00000172155  &  ZNF251  &  ENSG00000198169 \\
%LCE1E  &  ENSG00000186226  &  ZNF256  &  ENSG00000152454 \\
%LCE3B  &  ENSG00000187238  &  ZNF283  &  ENSG00000167637 \\
%LDHA  &  ENSG00000134333  &  ZNF284  &  ENSG00000186026 \\
%LEPREL1  &  ENSG00000090530  &  ZNF3  &  ENSG00000166526 \\
%LGI3  &  ENSG00000168481  &  ZNF311  &  ENSG00000197935 \\
%LGR5  &  ENSG00000139292  &  ZNF384  &  ENSG00000126746 \\
%LILRA5  &  ENSG00000187116  &  ZNF385D  &  ENSG00000151789 \\
%LIMA1  &  ENSG00000050405  &  ZNF407  &  ENSG00000215421 \\
%LIMK1  &  ENSG00000106683  &  ZNF430  &  ENSG00000118620 \\
%LIN7A  &  ENSG00000111052  &  ZNF503  &  ENSG00000165655 \\
%LONRF1  &  ENSG00000154359  &  ZNF517  &  ENSG00000197363 \\
%LOX  &  ENSG00000113083  &  ZNF572  &  ENSG00000180938 \\
%LPAR5  &  ENSG00000184574  &  ZNF589  &  ENSG00000164048 \\
%LPHN2  &  ENSG00000117114  &  ZNF639  &  ENSG00000121864 \\
%LRCH3  &  ENSG00000186001  &  ZNF646  &  ENSG00000167395 \\
%LRP3  &  ENSG00000130881  &  ZNF726  &  ENSG00000213967 \\
%LRPPRC  &  ENSG00000138095  &  ZNF823  &  ENSG00000197933 \\
%LRRC59  &  ENSG00000108829  &  ZNHIT6  &  ENSG00000117174 \\
%LRRIQ3  &  ENSG00000162620  &  ZRANB1  &  ENSG00000019995 \\
%LRRTM3  &  ENSG00000198739  &  ZSCAN23  &  ENSG00000187987 \\
%
%\end{longtable}
%\end{tiny}
%\end{center}
%\clearpage
\section{Cell cycle gene list}
\label{cellcyclelist}
\begin{table}[!ht]
\centering
\caption{Cell cycle gene list. In bold are the three genes for which we found an extension of cell cycle length.}
\begin{tabular}{|l|l||l|l|}
\hline
Hugo Gene Name & Ensembl gene id & Hugo Gene Name & Ensembl gene id\\
\hline
ADAMTS3  &  ENSG00000156140  &  MARK1  &  ENSG00000116141  \\
\textbf{APOA1}  &  ENSG00000118137  &  MECR  &  ENSG00000116353  \\
ARSF  &  ENSG00000062096  &  MGAT4A  &  ENSG00000071073  \\
ATR  &  ENSG00000175054  &  MMP24  &  ENSG00000125966  \\
B4GALT3  &  ENSG00000158850  &  MST1R  &  ENSG00000164078  \\
BMPR1B  &  ENSG00000138696  &  MT-CO1  &  ENSG00000198804  \\
BMPR2  &  ENSG00000204217  &  MTNR1A  &  ENSG00000168412  \\
CACNA1D  &  ENSG00000157388  &  NEK10  &  ENSG00000163491  \\
CDK15  &  ENSG00000138395  &  NOX1  &  ENSG00000007952  \\
CDKN3  &  ENSG00000100526  &  NPC1  &  ENSG00000141458  \\
CHRNA5  &  ENSG00000169684  &  NR1D1  &  ENSG00000126368  \\
CIB3  &  ENSG00000141977  &  OR1F1  &  ENSG00000168124  \\
CKB  &  ENSG00000166165  &  OSBP2  &  ENSG00000184792  \\
CP  &  ENSG00000047457  &  OSMR  &  ENSG00000145623  \\
DCK  &  ENSG00000156136  &  PAPD7  &  ENSG00000112941  \\
DHPS  &  ENSG00000095059  &  PRPS2  &  ENSG00000101911  \\
DIMT1  &  ENSG00000086189  &  PXDNL  &  ENSG00000147485  \\
EIF2AK1  &  ENSG00000086232  &  PYGB  &  ENSG00000100994  \\
F9  &  ENSG00000101981  &  RAB6B  &  ENSG00000154917  \\
GDA  &  ENSG00000119125  &  RGL1  &  ENSG00000143344  \\
GPR12  &  ENSG00000132975  &  RIPK2  &  ENSG00000104312  \\
GPRC5C  &  ENSG00000170412  &  RNASEL  &  ENSG00000135828  \\
HAS3  &  ENSG00000103044  &  \textbf{RPS20 } &  ENSG00000008988  \\
ILK  &  ENSG00000166333  &  RPS6KA3  &  ENSG00000177189  \\
ILVBL  &  ENSG00000105135  &  SERPINF1  &  ENSG00000132386  \\
IP6K3  &  ENSG00000161896  &  \textbf{SFMBT2}  &  ENSG00000198879  \\
KCNH6  &  ENSG00000173826  &  SGK1  &  ENSG00000118515  \\
KCNMA1  &  ENSG00000156113  &  SPSB2  &  ENSG00000111671  \\
KCNN1  &  ENSG00000105642  &  TAB1  &  ENSG00000100324  \\
KIF20B  &  ENSG00000138182  &  TGM5  &  ENSG00000104055  \\
KIFC2  &  ENSG00000167702  &  TRMT2A  &  ENSG00000099899  \\
MAP4K4  &  ENSG00000071054  &  TTL  &  ENSG00000114999  \\

\hline
\end{tabular}
\end{table}
\clearpage
\section{Functional inference by in silico comparison of small-molecule and siRNA screens}
\subsection{Choice of $\lambda$ parameter}
\label{choice_param_Sinkhorn}
\begin{figure*}[ht!]
\centerline{
\includegraphics[scale=0.4]{figures/MDS_lamb01_perPheno.png}
}
\caption{Separation between Mitocheck hit categories for $\lambda=0.1$. Global Sinkhorn divergences between Mitocheck hit experiments were computed for $\lambda=0.1$, and multi-dimensional scaling was used for representing them in two dimensions in the first two lines. Divergences between theses experiments and the drug screen were included and their multi-dimension scaling is showed on the last line (grey: controls).}
\label{lambda_choice2}
\end{figure*}
\begin{figure*}[ht!]
\centerline{
\includegraphics[scale=0.4]{figures/MDS_lamb10_perPheno.png}}
\caption{Separation between Mitocheck hit categories for $\lambda=10$. Global Sinkhorn divergences between Mitocheck hit experiments were computed for $\lambda=10$, and multi-dimensional scaling was used for representing them in two dimensions in the first two lines. Divergences between theses experiments and the drug screen were included and their multi-dimension scaling is showed on the last line (grey: controls).}
\label{lambda_choice3}
\end{figure*}
\clearpage
\subsection{Phenotypic scores of JNJ7706621}
\label{sec:jnj}
\begin{figure*}[ht!]
\centerline{
\includegraphics[scale=0.35]{figures/phenoscore_nice_JNJ7706621.png}}
\caption{Phenotypic scores of JNJ7706621 experiments, as a function of plate (left, middle, right) and dose (abscissa). The redder a square, the further away from control phenotypic scores.}
\label{jnj}
\end{figure*}



\clearpage
\section{Two-dimensional hierarchical clustering of drug screen condition distance to Mitocheck siRNAs for different distances}
\label{appendix:heatmaps}

\begin{figure*}[ht!]
\centerline{\includegraphics[scale=0.55]{figures/Clust_0ttranspor_ward_ward.png}
}
\caption{Drug screen condition - Mitocheck siRNA two-dimensional hierarchical clustering using sum of time Sinkhorn divergence.}
\label{cond_clust_ttransport}
\end{figure*}

\begin{figure*}[ht!]
\centerline{\includegraphics[scale=0.55]{figures/Clust_0nature_ward_ward.png}
}
\caption{Drug screen condition - Mitocheck siRNA two-dimensional hierarchical clustering using phenotypic trajectory distance.}
\label{cond_clust_nature}
\end{figure*}

\begin{figure*}[ht!]
\centerline{\includegraphics[scale=0.55]{figures/Clust_0U_pheno_s_ward_ward.png}
}
\caption{Drug screen condition - Mitocheck siRNA two-dimensional hierarchical clustering using Euclidean distance of phenotypic scores.}
\label{cond_clust_ps}
\end{figure*}



\clearpage
\section{Literature review}
\label{literature_review}
\begin{center}


\begin{longtable}{|L{3cm}|%L{4cm}|
L{14cm}|}
\caption{Human xenobiotic levels}\\

\hline
Name & %Human exposure & 
Human levels \\
\hline
\hline
\endfirsthead

\hline
Name & %Human exposure & 
Human levels \\
\hline
\hline
\endhead

\hline \multicolumn{2}{|r|}{{Continued on next page}} \\ \hline
\endfoot

\hline \hline
\endlastfoot


BPA & %Oral exposure: from <1$\mu$g/kg/day to 5$\mu$g/kg/day  \cite{pmid19074586} &
Serum levels: order 0.9 to 87.6 nM(unconjugated, 0.2 to 20 ng/ml)~\cite{pmid19074586} \\
\hline
Dioxin &  Blood level: between $0.28.10^{-3}$~pM and $0.27$~pM
\begin{itemize}
\item Seveso, Italy: 12.4 pg/g lipid and 5.5 pg/g lipid (medians, plasma sampled between 1992 and 1994 vs accident in 1976, resp high contamination area and low contamination area, ~\cite{pmid22880488})
\item Japan: 1 pg/g lipid, 2 pg/g lipid (resp median, p75)~\cite{pmid21138777}. Same order of magnitude in Canada~\cite{pmid22750796}
\item Germany: 0.020 pg/g lipid, 3.92 pg/g lipid (resp median, max)~\cite{pmid19665752}
\item Industrialized area, Germany: 1.3 pg/g lipid, 4.9 pg/g lipid (resp median, max)~\cite{pmid17217986}
\end{itemize} 
Adipose tissue: 2.05 pg/g lipid, 2.45 pg/g lipid (resp mean, P75 Spain~\cite{pmid18682306})\\
&
 Breast milk: 0.882 pg/g lipid, 3.58 pg/g lipid (resp median, max, China~\cite{pmid21531025}), 1.5 pg/g lipid, 5.3 pg/g lipid (resp median, max, industrialized area, Germany~\cite{pmid17217986}). Same order of magnitude in France~\cite{pmid23500409} \\
\hline
Endosulfan &%Oral exposure (mix?): 14.2 ng/kg/day (California, \cite{pmid23140444}) &
 Serum levels (1, 2):
\begin{itemize}
\item Contaminated Brasilian area: approx. 0.5, 0.6 nM (median), 1.1-1.2, 1.5-1.8 nM (p75) (resp. approx. 0.22, 0.25 ng/ml and 0.42-0.51, 0.62-0.75 ng/ml, \cite{pmid23972672})
\item Baseline serum levels in farm workers: 1.30 $\mu$M (mean, 530 ng/ml, \cite{pmid19280480})
\item Young male Spaniards: 3.61,3.43 nM (median, 1.47, 1.00 ng/ml~\cite{pmid16889768})
\end{itemize}
 \\
\hline
MeHg & %Oral exposure: 14.5 ng/kg/day (California, \cite{pmid23140444})  &
 Blood plasma: 1.30 nM, 7.23nM (resp mean, max, 0.28$ \mu$g/L, 1.56$ \mu$g/L, Hong-Kong residents~\cite{pmid23680090})\\
& Whole blood:
\begin{itemize}
\item 78.8 nM, 519.4 nM (resp median, max, 17.0 $ \mu$g/L, 112 $ \mu$g/L, Canadian inuits~\cite{pmid22959488})
\item 89.0 nM (P75 19.2 $ \mu$g/L, contaminated environment, China~\cite{pmid18675410})
\item m\^eme ordre de grandeur (eg Sardaigne) ou un au-dessus (eg Br\'esil) dans diff\'erents endroits contamin\'es~\cite{pmid18675410}
\end{itemize}
    \\
\hline
PCB153 &  Serum levels: approx between 1nM and 500 nM
\begin{itemize}
\item Napoli: 42.3 ng/g lipid, 195.3 ng/g lipid (median, max~\cite{pmid24112656})
\item Slovakia: 232/578 ng/g lipid, 5,193/25,089 ng/g lipid (background area/contaminated area, median, max~\cite{pmid24112656})
\item >65 years old, Canada: 73.6 ng/g lipid, 208 ng/g lipid (median, P95~\cite{pmid22001220})
\item Inuits, Canada:177 ng/g lipid, 6,020 ng/g lipid (geom mean, max~\cite{pmid20435334})
\end{itemize} Whole blood: 0.89 nM, 6.65 nM (0.32 $\mu$g/l median, 2.4 $\mu$g/l max, industrialized region, Germany~\cite{pmid17217986})\\ &
 Breast milk: 20 ng/g lipid, 49 ng/g lipig (median, max, Philippines~\cite{pmid23178840}), 3.4 ng/g lipid, 8.0 ng/g lipid (mean, max, Turkey~\cite{pmid22280929}), range 20-183 ng/g lipid (Northern Russia~\cite{pmid18063018})%lipid amounts Philippines median 3.4%, range 1.5-5.6% ; Turkey mean 3.6%, range 1.0-9.6% ; Northern Russia range 0.9-10.4%
\\
\label{table_xb2}
\end{longtable}
\end{center}
\clearpage
\section{Phenotypic study}
\label{phenotypic_annex}
\begin{figure}
\caption{Apoptosis distances. Colors (up) are black for control wells, yellow to red for xenobiotics ranked by increased dose, and magenta for TGF-$\beta$1. B: BPA, D:DMSO, E: Endo, M: MeHg, N: Nonane, P:PCB, R: nothing, T(red):TCDD, T(magenta): TGF-$\beta$1. Colors (down) are for plates.}
\label{apoptosis}
\includegraphics[scale=0.4, angle=90]{figures/apoptosis.png}
\end{figure}



\begin{figure}
\caption{Frozen distances. Colors (up) are black for control wells, yellow to red for xenobiotics ranked by increased dose, and magenta for TGF-$\beta$1. B: BPA, D:DMSO, E: Endo, M: MeHg, N: Nonane, P:PCB, R: nothing, T(red):TCDD, T(magenta): TGF-$\beta$1. Colors (down) are for plates.}
\label{frozen2}
\includegraphics[scale=0.4, angle=90]{figures/frozen2.png}
\end{figure}
\begin{figure}
\caption{Interphase distances. Colors (up) are black for control wells, yellow to red for xenobiotics ranked by increased dose, and magenta for TGF-$\beta$1. B: BPA, D:DMSO, E: Endo, M: MeHg, N: Nonane, P:PCB, R: nothing, T(red):TCDD, T(magenta): TGF-$\beta$1. Colors (down) are for plates.}
\label{interphase2}
\includegraphics[scale=0.4, angle=90]{figures/interphase2.png}
\end{figure}
\begin{figure}
\caption{Metaphase distances. Colors (up) are black for control wells, yellow to red for xenobiotics ranked by increased dose, and magenta for TGF-$\beta$1. B: BPA, D:DMSO, E: Endo, M: MeHg, N: Nonane, P:PCB, R: nothing, T(red):TCDD, T(magenta): TGF-$\beta$1. Colors (down) are for plates.}
\label{metaphase}
\includegraphics[scale=0.4, angle=90]{figures/metaphase.png}
\end{figure}
\begin{figure}
\caption{Micronucleated distances. Colors (up) are black for control wells, yellow to red for xenobiotics ranked by increased dose, and magenta for TGF-$\beta$1. B: BPA, D:DMSO, E: Endo, M: MeHg, N: Nonane, P:PCB, R: nothing, T(red):TCDD, T(magenta): TGF-$\beta$1. Colors (down) are for plates.}
\label{micronucleated}
\includegraphics[scale=0.4, angle=90]{figures/micronucleated.png}
\end{figure}

\begin{figure}
\caption{Prometaphase distances. Colors (up) are black for control wells, yellow to red for xenobiotics ranked by increased dose, and magenta for TGF-$\beta$1. B: BPA, D:DMSO, E: Endo, M: MeHg, N: Nonane, P:PCB, R: nothing, T(red):TCDD, T(magenta): TGF-$\beta$1. Colors (down) are for plates.}
\label{prometaphase}
\includegraphics[scale=0.4, angle=90]{figures/prometaphase.png}
\end{figure}
\clearpage

%----------------------------------------------------------------------------------------
%	BIBLIOGRAPHY
%----------------------------------------------------------------------------------------
\lhead{\emph{Bibliography}} % Change the page header to say "Bibliography"

\bibliographystyle{apalike}
\bibliography{these,refs}{}
\end{document}