\lhead{\emph{Introduction}} % Set the left side page header to "List of Figures"
\chapter{Introduction [TODO]}
What is the topic and why is it important? State the problem(s) as simply as you can. How does it fit into the broader world of your discipline? 
Where did the problem come from? What is already known about this problem? What other methods have been tried to solve it? 


mentionner la toxicologie environnementale. Ce qu'on fait c'est de la mise en place d'outils. Etre un peu global sur le time-lapse en criblage. Il y a toxicology(envtal pharmaco)/genetic. Explain what a screen is, give examples

What are the challenges and questions that come up in such a setting => pour que ca nous arrange

Not necessarily very long. Just prepare the reader to know what screens are, their scope. Also give a reasoning fil rouge for the rest, for it to be clear for the reader
\section{Data sets}
\subsection{Mitocheck data set}

We used a previously published genome-wide data set of time-resolved
records of cellular phenotype responses to gene silencing, which
were generated for virtually all protein-coding genes~\cite{pmid20360735}. It is publicly available at
\href{http://www.mitocheck.org}{mitocheck.org}. 

For this, arrays of transfection cocktails containing small interfering RNA (siRNA) were spotted directly into live cell-imaging chambers in a 384 format. HeLa cells (ATCC\up{\textregistered} CCL-2\texttrademark) stably expressing the core histone 2B tagged with GFP were seeded on top of the arrays, and imaged 18 h after the transfection for 48 h with a time-lapse of 30 min (Plan10x, NA 0.4; Olympus). Each microarray contained 8 negative controls (scrambled: not targeting any gene) and 12 positive controls showing different phenotypes. 
22,612 protein-coding genes have been targeted by at least 2 siRNAs each, in total 51,767 siRNAs. For each siRNA, there is data from at least 3 technical replicates, which created 182,191 quality controlled time-lapse experiments in total. Due to updates in the genome annotation, some reagents could not be mapped to the current ENSEMBL version. In total, the data set contains data for 17,816 protein-coding genes in 144,909 quality controlled time-lapse
experiments.  

HeLa cells are epithelial cancer cells which were derived from the adenocarcinoma of Henrietta Lacks in 1951. Being the first human cell line to survive \textit{ex vivo} for more than a few days, they are very appreciated from cell biologists as they are easy to grow and transfect. Indeed, this cell line is mentioned in approximately 0.3\% of PubMed abstracts although 64\% of its genome has a copy number greater than three~\cite{pmid23925245}. It is not motile as can be the case of other cell lines which are widely used in migration studies such as the epithelial metastatic breast-cancer derived MDA-MB-231 cell line (ATCC\up{\textregistered} HTB-26\texttrademark), or the epithelial metastatic lung-cancer derived NCI-H1299 (ATCC\up{\textregistered} CRL-5803\texttrademark). Indeed, it was the 16\up{th} slowest out of 54 in the first World Cell Race~\cite{pmid22974990}. Gene silencing in this background therefore makes it easier to identify migration suppressors, that is, genes whose silencing will enhance cell motility, rather than migration enhancers, given that HeLa basal cell motility is rather low.

\subsection{PCNA data set}
\subsection{Drug screen}
\subsection{Xenobiotic screen}
\section{Software}

We use CellCognition
\cite{cellcognition} (\href{http://cellcognition.org}{cellcognition.org}) for segmentation and feature
extraction and CPlex \\ (\href{http://www-01.ibm.com/software/commerce/optimization/cplex-optimizer/}{http://www-01.ibm.com/software/commerce/optimization/cplex-optimizer/}) for optimization in the tracking procedure. To store, manage and access the
screening data, we use a previously published data format CellH5
\cite{Sommer2013}. All scripts are written in the programming language \href{http://www.python.org}{Python 2.7}
using scipy~(\cite{scipy}), statsmodels, scikit-learn, rpy2 and numpy, and all plots were generated by matplotlib~(\cite{matplotlib}).