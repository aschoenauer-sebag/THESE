\lhead{\emph{Conclusion}} 
\chapter{Conclusion}
\begin{table}[!ht]
\begin{tabular}{|l|}
\hline
~\\

\textbf{Résumé~}(see \textit{infra} for English text)\\
\parbox{15cm}{Grâce aux progrès dans les domaines de la robotique, de l'informatique, comme de la chimie organique et de la biologie moléculaire, les cribles biologiques à haut débit et haut contenu se sont multipliés ces dernières années. Parmi les techniques utilisées, la vidéomicroscopie l'est de plus en plus.

Elle produit de riches jeux de données, dont l'exploitation optimale reste une question ouverte. C'est ce à quoi nous nous sommes attachés à répondre dans cette thèse. Nous avons tout d'abord présenté le premier workflow pour l'étude de la motilité cellulaire individuelle dans de telles données, MotIW. Le chapitre~\ref{chap:reuse} démontre ensuite par trois exemples l'intérêt de la ré-utilisation des données de vidéomicroscopie produites dans le cadre de cribles à haut débit : l'application de MotIW aux données du projet Mitocheck~\cite{pmid20360735}, l'étude du cycle cellulaire dans ces mêmes données, et enfin leur utilisation pour l'inférence de cibles thérapeutiques par leur comparaison avec un crible pharmaceutique non-publié. Le chapitre~\ref{chap:xbsc} présente en dernier lieu une approche méthodologique globale pour l'utilisation de la vidéomicroscopie en toxicologie environnementale.

Cette thèse a amené à l'établissement de pistes sérieuses en ce qui concerne les gènes impliqués dans la motilité cellulaire, comme dans le cycle cellulaire. Elle a également abouti au développement d'une distance pour l'inférence de cibles thérapeutiques dans les données de vidéomicroscopie. Ces résultats gagneraient respectivement à être confirmés et utilisés dans d'autres systèmes. D'autre part, les expériences réalisées en toxicologie environnementale ont permis d'identifier les pierres d'achoppement de la procédure expérimentale. Une des perspectives de cette thèse serait par conséquent de reconduire les expériences en tenant compte des lignes directrices identifiées.}\\
~\\
\hline
\end{tabular}
\end{table}
\clearpage
In the last two decades, constant progress in the fields of molecular and cellular biology, laboratory hardware automation and computational methods for large scale data storage and data mining, have permitted high-throughput high-content experiments to presently be almost affordable and mainstream. As such, time-lapse microscopy has become more and more used. This has enabled a better understanding of complex dynamic biological processes such as cell division, which endpoint assays can more hardly help grasp. Not only are endpoint assays bound to miss rare and transient events, but they do not permit any assessment of the order in which displayed events happened.

Nevertheless, the question to know how to optimally develop computational methods for mining such large and complex datasets remains open. Indeed, time-lapse microscopy experiments produce three to five-dimensional datasets: 2 or 3 dimensions come from the images, time constitutes another one, and when a specific perturbation was studied (e.g. gene silencing, chemical exposure), it adds another dimension. The high-throughput quality of such experiments finally adds another difficulty: the size of the final dataset.

\paragraph*{Main highlights\\}
This is precisely the question which we have aimed at tackling in this thesis.

Given both that single cell motility is a particularly appropriate subject to be studied using time-lapse microscopy, and that there is currently no fully automated multivariate method for addressing such a question in this type of data, we have in the first place designed a generic methodological workflow for studying single cell motility in HT time-lapse microscopy experiments in \textbf{chapter~\ref{chap:motiw}}. This workflow was furthermore validated on a simulated screen, and applied to an existing dataset of approximately $150,000$ videos.

This leads to the second main contribution of this thesis, namely the proof that HT HC time-lapse microscopy datasets constitute a rich and much valuable good. We think in particular that, should they be easily accessible and re-mined, their content could lead to more than one or two high-impact discoveries. In \textbf{chapter~\ref{chap:reuse}}, we therefore attached ourselves to re-discovering the Mitocheck dataset~\cite{pmid20360735} from the perspectives of single cell motility, cell cycle and drug target inference. This permitted to discover an ontology of single cell motility behaviours as well as a list of putative cell cycle genes. Furthermore, it enabled us to develop various other methods: for studying cell cycle in time-lapse experiments, and for performing drug target inference using phenotypic profiling on parallel siRNA and drug screens.

Finally, drawing on this methodological development, we exported the technique of HT HC time-lapse microscopy to \textit{Environmental Toxicology} in \textbf{chapter~\ref{chap:xbsc}}. Although the results we observed in our newly generated dataset were not up to our expectations, all necessary methodological and practical tools have been developped, which are ready to be used on new data.

\paragraph*{Perspectives\\}
Application of our methodological workflow to the Mitocheck dataset produced a list of genes that might play a role in single cell motility. This list of genes was obtained in a specific model, HeLa cells, using a specific set of siRNAs. Therefore, confirmatory experiments in one (or more) different cell lines, using another set of siRNAs, should be performed in order to confirm our results. This would also permit to know if the ontology of single cell motility behaviour we obtained exists in other cell lines.

Similarly, we have developed a new distance for drug target inference by phenotypic profile comparison between parallel siRNA and drug screens. It would be interesting to confirm that this distance can apply to datasets using different markers and phenotypic classes. This distance could also benefit from further methodological development, which would take into account the temporal dimension of our data.

Finally, the reasons for the little success which was obtained in chapter~\ref{chap:xbsc} were at least partly identified. As the need for sophisticated and HT assays in Environmental Toxicology are rather increasing than diminishing, new data should be generated following the guidelines which this thesis permitted to identify. A simple and homogenous model organism should be used. It would be made fluorescent for one or two relevant markers, whose genetic insertion using modern techniques such as the CRISPR/Cas system would not alter any fundamental cellular processes. Last but not least, xenobiotics and doses should be chosen following robust preliminary experiments. HT use of such an assay in Environmental Toxicology would help us better understand our chemical environment.

