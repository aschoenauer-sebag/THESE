\documentclass[12pt]{article}
\usepackage[french]{babel}
\usepackage[T1]{fontenc}
\usepackage{amsmath}
\usepackage{multirow}

\usepackage[pdftex]{graphicx}
\usepackage{graphics}
\usepackage{color}
\usepackage{floatflt}
%usepackage[frenchb]{babel}
\usepackage{amsfonts}
\usepackage{multirow}
\usepackage{calc}
%\linespread{1.3}
\usepackage{setspace}
\usepackage{hyperref}
\hypersetup{bookmarks=true,
colorlinks,%
citecolor=black,%
filecolor=black,%
linkcolor=black,%
urlcolor=blue,%
pdftex}
\usepackage{array}
\newcolumntype{L}[1]{>{\raggedright\let\newline\\\arraybackslash\hspace{0pt}}m{#1}}
\DeclareMathOperator*{\argmax}{arg\,max}
\DeclareMathOperator*{\argmin}{arg\,min}

\setlength{\unitlength}{2mm}
\pagestyle{empty}
\textwidth 17truecm
\textheight 23truecm
\voffset -2.5truecm
\hoffset -2.0truecm
\sloppy
\hbadness = 10000
\vbadness = 10000
\def\RR{{\rm I\hspace{-0.50ex}R} }

\begin{document}
Things to check for PhD
\section{Title}
\begin{itemize}
\item English: Generic methods for data re-use and comparative analyses for High
Content Screening by live cell imaging
%\item English: The versatility of high-content high-throughput time-lapse screening data: developing generic methods for data re-use and comparative analyses
\item French: Développements méthodologiques pour données de cribles temporels à haut contenu et haut débit: versatilité et analyses comparatives
\end{itemize}
\section{Keywords}
\begin{enumerate}
\item Data mining
\item Machine Learning
\item Bioinformatics

\item Bioimage informatics
\item High-content screening
\item Toxicology

\end{enumerate}


\section{Summary (4,000 char and both must fit on the last page)}
\subsection{English}

%Biological screens are experiments in which a set of compounds is
%tested for a speci c biological e ect on a living entity. Recent
%technical and computational progresses have made it possible to
%biologically test hundreds to thousands of chemical compounds in
%parallel, such as small interfering RNAs or putative drugs and small
%molecules. 
%
%Time-lapse microscopy screening experiments enable to gather even more
%and more precise data regarding the consequences of chemical
%perturbation on a given biological process. Never- theless, the
%quantity and speci c structure of information which such experiments
%produce make them di cult to analyze. They demand the combination of
%robust computer vision methods, e cient statistical methods for signi
%cant e ect detection and reliable quality controls. 
%
%Through examples, this thesis answers the question to know how to
%optimally develop analy- tical methods for high throughput (HT)
%time-lapse microscopy screening data. It demonstrates the signi cant
%insights which one can gain by developing generic methods for
%analyzing and re-analyzing such screens. The  rst multivariate work ow
%for the study of single cell motility in such data is detailed in
%Chapter 2. Chapter 3 presents its application to published data, and
%the development of a new distance for drug target inference by in
%silico comparisons of parallel siRNA and drug screens. Finally,
%chapter 4 presents a complete methodological pipeline for performing
%HT time-lapse screens in Environmental Toxicology, together with its
%visualization Web-interface.


%Biological screens are experiments in which compounds,
%e.g. pollutants, are tested for a given biological e ect on a living
%entity. Recent progresses have made it possible to test more than
%hundreds of compounds in parallel.

%Time-lapse microscopy screening experiments enable to gather even more
%and more precise data regarding the consequences of chemical
%perturbation on a given biological process. Nevertheless, the quantity
%and specific structure of information which such experiments produce
%make them difficult to analyze. They demand the combination of robust
%computer vision methods, efficient statistical methods for significant
%effect detection and reliable quality controls. 

%Through examples, this thesis answers the question to know how to
%optimally develop analytical methods for high throughput (HT)
%time-lapse microscopy screening data. It demonstrates the significant
%insights which one can gain by developing generic methods for
%analyzing and re-analyzing such screens. The first multivariate
%workflow for the study of single cell motility in such data is
%detailed in Chapter 2. Chapter 3 presents its application to published
%data, and the development of a new distance for drug target inference
%by \textit{in silico} comparisons of parallel siRNA and drug screens.
%Finally, chapter 4 presents a complete methodological pipeline for
%performing HT time-lapse screens in Environmental Toxicology, together
%with its visualization Web-interface.


%Biological screens are experiments in which a large set of conditions is
%tested for a specific biological effect on a living system. Recent
%technical and computational progresses have made it possible to apply
%such approaches at a large scale involving tens to hundreds of
%thousands of experimental conditions, with applications in the field
%of functional genomics (RNA intereference screens) or drug screening. 
%
%Live cell imaging is an excellent tool to study in detail the consequences of chemical
%perturbation on a given biological process. 
%However, the quantity
%and specific structure of information produced by such experiments 
%make them difficult to analyze. They demand the combination of robust
%computer vision methods, efficient statistical methods for hit
%detection and robust procedures for quality control.
%  
%This thesis addresses these problems by the development of analytical methods
%for the analysis of High Content Screening (HCS) data. Application of these
%methods to publicly available HCS data, the thesis demonstrates
%applicability of these frameworks and the usefulness of these data as
%a scientific resource. The first multivariate
%workflow for the study of single cell motility in such large-scale data is
%detailed in Chapter 2. Chapter 3 presents application of this workflow
%to previously published
%data, and the development of a new distance for drug target inference
%by \textit{in silico} comparisons of parallel siRNA and drug screens.
%Finally, chapter 4 presents a complete methodological pipeline for
%performing HT time-lapse screens in Environmental Toxicology, together
%with its visualization Web-interface.
%
Biological screens test large sets of experimental conditions with respect to their specific biological effect on living systems. Technical and computational progresses have made it possible to perform such screens at a large scale - up to hundreds of thousands of experiments. % Such approaches have applications in the field of functional genomics (e.g. RNA interference (RNAi) screens) and in the field of pharmacology (drug screens). 
Live cell imaging is an excellent tool to study in detail the consequences of chemical perturbation on a given biological process. However, the analysis of live cell screens demands the combination of robust computer vision methods and quality control procedures, and efficient statistical approaches for the detection of significant effects. %the quantity and structural complexity of such experimental data make them difficult to analyze. They demand the combination of robust computer vision methods, efficient statistical methods for the detection of significant effects and robust procedures for quality control. 
This thesis addresses these challenges by developing analytical methods for High Throughput time-lapse microscopy screening data. The developed frameworks are applied to publicly available HCS data, demonstrating their applicability and the benefits of HCS data remining. % usefulness of HCS data as a scientific resource to gain new insights from existing data. 
The first multivariate workflow for the study of single cell motility in such large-scale data is detailed in Chapter 2. Chapter 3 presents this workflow application to previously published data, and the development of a new distance for drug target inference by in silico comparisons of parallel siRNA and drug screens. Finally, chapter 4 presents a complete methodological pipeline for performing HT time-lapse screens in Environmental Toxicology.%, together with its visualization Web-interface.
\subsection{French}
Un crible biologique a pour objectif de tester en parallèle l'impact de nombreuses conditions expérimentales sur un processus biologique d'un organisme modèle. Le progrès technique et computationnel a rendu possible la réalisation de tels cribles à grande échelle - jusqu'à des centaines de milliers de conditions.

L'imagerie sur cellules vivantes est un excellent outil pour étudier en détail les conséquences d'une perturbation chimique sur un processus biologique. L'analyse des cribles sur cellules vivantes demande toutefois la combinaison de méthodes robustes d'imagerie par ordinateur et de contr\^ole qualité, et d'approches statistiques efficaces pour la détection des effets significatifs.

La présente thèse répond à ces défis par le développement de méthodes analytiques pour les images de cribles temporels à haut débit. Les cadres qui y sont développés sont appliqués à des données publiées, démontrant par là leur applicabilité ainsi que les bénéfices d'une ré-analyse des données de cribles à haut contenu (HCS). Le premier workflow pour l'étude de la motilité cellulaire à l'échelle d'une cellule dans de telles données constitue le chapitre 2. Le chapitre 3 applique ce workflow à des données publiées et présente une nouvelle distance pour l'inférence de cible thérapeutique à partir d'images de cribles temporels. Enfin, le chapitre 4 présente une pipeline méthodologique complète pour la conduite de cribles temporels à haut débit en toxicologie environnementale.

\section{General public summary (1,000 char)}
\subsection{English}
%Biological screens are experiments in which compounds,
%e.g. pollutants, are tested for a given biological effect on a living
%entity. Recent progresses have made it possible to test more than
%hundreds of compounds in parallel. 
%
%Time-lapse microscopy screening experiments are screening experiments
%in which images are acquired over time. Temporal experiments enable to
%gather more and more precise data. But they constitute large
%structured datasets which are harder to analyze: one needs to combine
%robut image analysis algorithms with efficient statistical methods. 
%
%This thesis answers through examples the question to know how to
%optimally develop generic analytical methods for such data: it
%presents e.g. the first multivariate workflow for the study of single
%cell motility in such data, the development of a new approach for drug
%target inference, and a complete methodological pipeline for
%performing such screens in Environmental Toxicology, together with its
%visualization Web-interface. 
%

Biological screens test large sets of experimental conditions with
respect to their specific biological effect on living systems. Recent
progresses have made it possible to apply such approaches at a large
scale, involving thousands of experiments. 

In time-lapse microscopy, images are acquired over time. Such
experiments allow studying dynamic biological processes. But they
generate large complex datasets, which are more difficult to analyze:
one needs to combine robust image analysis algorithms with efficient
statistical methods.  

This thesis addresses these challenges and shows the applicability of
the developed frameworks. It presents e.g. the first multivariate
workflow for the study of single cell motility in such data, the
development of a new approach for drug target inference, and a
complete methodological pipeline for performing such screens in
Environmental Toxicology, together with its visualization
Web-interface. 

\subsection{French}
Un crible biologique a pour objectif de tester l'impact de nombreuses conditions expérimentales sur un organisme vivant. Le progrès technique et computationnel a rendu possible leur réalisation à haut débit - jusqu'à des centaines de milliers de conditions.

L'imagerie sur cellules vivantes est un excellent outil pour étudier les processus biologiques dynamiques. Mais elle génère des données larges et complexes. Leur analyse demande la combinaison de méthodes robustes d'imagerie par ordinateur et d'approches statistiques.

La présente thèse répond à ces défis par le développement et l'application de méthodes analytiques pour les images de cribles temporels à haut débit. Elle présente par exemple le premier workflow pour l'étude de la motilité cellulaire individuelle dans de telles données, une nouvelle approche pour l'inférence de cible thérapeutique, ainsi qu'une pipeline complète pour la conduite de cribles biologiques sur cellules vivantes en toxicologie environnementale. % Les cadres qui y sont développés sont appliqués à des données publiées, démontrant par là leur applicabilité ainsi que les bénéfices d'une ré-analyse des données de cribles à haut contenu (HCS). Le premier workflow pour l'étude de la motilité cellulaire à l'échelle d'une cellule dans de telles données constitue le chapitre 2. Le chapitre 3 applique ce workflow à des données publiées et présente une nouvelle distance pour l'inférence de cible thérapeutique à partir d'images de cribles temporels. Enfin, le chapitre 4 présente une pipeline méthodologique complète pour la conduite de cribles temporels à haut débit en toxicologie environnementale.

\section{Plan de la soutenance}
My idea is to have broad introduction and conclusions, and specific ones in parts 1 and 2, just like in the thesis.
\subsection*{Global introduction - 5min}
\begin{itemize}
\item Global subject introduction
\item The contributions of this PhD are... 
\item In this PhD defense we will talk about...
\end{itemize}
\subsection*{Part 1: MotIW and its application to Mitocheck - 20min}
Idea: do the same presentation as ISMB except be more detailed about the state-of-the-art
\subsection*{Part 2: Part on drug target inference and application to Mitocheck+unpublished data - 15 min}
\subsection*{General conclusion - 5 min}

%\[
%\forall ~ r,c \in \Sigma_d ~ d_M(r,c) = \min_{\substack{P\in U(r,c)}} <M,P>
%\]
%
%\[
%\Sigma_d = \{ x \in \mathbb{R}^d_+ | x^T \mathbf{1}_d = 1\}
%\]
%
%\[
%(\%_{k,t})_{k,t}
%\]

\end{document}